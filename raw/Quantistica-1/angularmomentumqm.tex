\documentclass[a4paper, 11pt]{article}
	\usepackage[a4paper, left=2.5cm, bottom=2.5cm]{geometry}
	\usepackage{beppe_package}
	\usepackage{bbm}
	
	\title{Il momento angolare in Meccanica Quantistica}
	
	\newcommand{\op}{\hat}
	\newcommand{\Op}[1]{\vec{\hat{#1}}}
	\def\bal#1\eal{\begin{align*}#1\end{align*}}
	\renewcommand{\H}{\mathcal{H}}
	\renewcommand{\L}{\mathcal{L}}
	\newcommand{\1}{\mathbbm{1}}
	\renewcommand{\l}{\ell}
	
\begin{document}
	\maketitle
	\section*{Introduzione}
	Sappiamo bene che l'omogeneità dello spazio euclideo in una teoria non relativistica è associata a un'operatore unitario della forma
	\[\op U=\exp\left(-\frac{i\Op{P}\cdot\vec{x}}{\hbar}\right)\]
	Nell'ambito delle teorie classiche è definito il prodotto interno tra vettori di $\R ^3$ 
	\[\vec{a}\cdot\vec{b}=a_ib_i\]
	ed è ben noto che tale prodotto è invariante sotto l'endomorfismo $\vec{a}\mapsto R\vec{a}$, dove $R\in\mathrm{O}(3)$. Limitiamoci ora al caso $R\in\mathrm{SO}(3)\subseteq\mathrm{O}(3)$ e usiamo la notazione $a_i'=R_{ij}a_j$.\footnote{In una teoria non relativisticamente invariante non è importante distinguere tra indici covarianti e indici controvarianti.}
	\section{Rotazioni}
	\subsection{Trasformazione degli autostati dell'impulso}
	Consideriamo gli autostati di $\Op{P}$, definiti da
	\[\Op{P}\ket{\vec{p}}=\vec{p}\ket{\vec{p}}\]
	Alla rotazione associata a $R$ facciamo corrisponde la mappa sullo spazio di Hilbert $\H$
	\[\ket{\vec{p}}\to\ket{R\vec{p}}\]
	L'interpretazione è chiara: ruotando nello spazio fisico, un autostato dell'impulso rimane autostato dell'impulso, ma il suo autovalore è il ruotato dell'autovalore originario. Si noti ora che 
	\bal 
	\braket{R\vec{q}}{R\vec{p}}&=(2\pi\hbar)^3\delta^3(R\vec{q}-R\vec{q})=\\&=\frac{(2\pi\hbar)^3}{|\det R|}\delta^3(\vec{q}-\vec{p})=\\&=\braket{\vec{q}}{\vec{p}}
	\eal
	dove si è usato il fatto che $\det R=1$. La trasformazione introdotta lascia quindi invariati i prodotti scalari dello spazio di Hilbert, dunque per il teorema di Wigner esiste un operatore unitario $\op U(R)$ tale che
	\[\op U(R)\ket{\vec{p}}=\ket{R\vec{p}}\]
	Sia ora $\rho\colon\SO(3)\to\L(\H)$ la mappa
	\[R\mapsto\op U(R)\]
	e dimostriamo che è una rappresentazione di $\SO(3)$. Per linearità, è sufficiente considerare un autostato dell'impulso, dato che questi formano una base di $\H$. Si ha allora
	\bal\op U(R_1R_2)\ket{\vec{p}}&=\ket{R_1R_2\vec{p}}=\\&=\op U(R_1)\ket{R_2\vec{p}}=\\&=\op U(R_1)\op U(R_2)\ket{\vec{p}} 
	\eal
	e quindi, come voluto, $\rho$ è una rappresentazione. Questo significa chiaramente che $\op U^{-1}(R)=\op U(R^{-1})$, che unito all'unitarietà porta a
	\[\op U^\dagger(R)=\op U(R)\]
	In particolare è
	\[\bra{\vec{p}}\op U(R)=\bra{R^{-1}\vec{p}}\]
	\subsubsection{Trasformazione delle funzioni d'onda}
	L'azione di $\op U(R)$ sugli autostati dell'impulso è chiaramente identica all'azione sugli autostati della posizione. Questo significa che la funzione d'onda $\braket{\vec{x}}{\psi}$ associata allo stato $\ket\psi$ trasforma come
	\bal
	\psi(\vec{x})&\mapsto\psi_R(\vec{x})=\\&=\braketop{\vec{x}}{\op U(R)}{\psi}=\\&=\braket{R^{-1}\vec{x}}{\psi}=\\&=\psi(R^{-1}\vec{x}) 
	\eal
	Nel seguito, indichiamo con $\mathcal{U}(R)$ l'operatore su $\mathbb L^2(\R ^3)$ che trasforma le funzioni d'onda, i.e.
	\[[\mathcal{U}(R)\psi](\vec{x})=\psi(R^{-1}\vec{x})\]
	\subsection{Rotazioni infinitesime}
	Associamo a una rotazione antioraria di un angolo $\theta$ intorno all'asse $\vec{n}$ il vettore $\bm{\theta}=\theta\vec{n}$. Si sa che se $\theta$ è un angolo generico, allora un vettore $\vec{v}$ trasforma come
	\[\vec v\mapsto\vec{v}'=(\vec{v}\cdot\vec{n})\vec{n}+(\vec{v}-(\vec{v}\cdot\vec{n})\vec{n})\cos\theta+(\vec{n}\times\vec{v})\sin\theta\]
	Per una rotazione infinitesima è $|\theta|\ll1$, dunque al primo ordine si trova
	\[\vec{v}'=\vec{v}+\bm{\theta}\times\vec{v}+o(\theta)\]
	Si trova così
	\[R^{-1}\vec{x}=\vec{x}-\bm{\theta}\times\vec{x}+o(\theta)\]
	e dunque sulle funzioni d'onda
	\bal
	[\mathcal{U}(R)\psi](\vec{x})&=\psi(R^{-1}\vec{x})=\\&=\psi(\vec{x}-\bm\theta\times\vec{x}+o(\theta))=\\&=\psi(\vec{x})-(\bm\theta\times\vec{x})\cdot\nabla\psi(\vec{x})+o(\theta)=\\&=[(\1-\bm\theta\cdot(\vec{x}\times\nabla))\psi](\vec{x})+o(\theta)
	\eal
	D'altro canto, per il teorema di Stone è
	\[\mathcal{U}(R)=\exp(-i\bm\theta\cdot\Op{L})\]
	dove $\Op{L}$ è il generatore infinitesimo delle rotazioni. Per confronto, è
	\[\hbar\Op{L}=-i\hbar\vec{x}\times\nabla=\vec{x}\times\Op{P}\]
	dunque $\hbar\Op{L}$ coincide con la definizione classica di momento angolare. In meccanica quantistica, tale operatore è noto come momento angolare orbitale, per distinguerlo dal momento angolare intrinseco o di spin. Si noti che la costante di Planck che compare nella definizione serve unicamente a rendere $\Op{L}$ adimensionale. In componenti, è
	\[\op L_i=-i\varepsilon_{ijk}\op x_j\partial_k=\hbar^{-1}\varepsilon_{ijk}\op x_j\op P_k\]
	Da cui si ricavano le regole commutazione
	\begin{align}
	[\op L_i,\op L_j]&=\hbar^{-2}\varepsilon_{ikl}\varepsilon_{jmn}[\op x_k\op P_l,\op x_m\op P_n]=\nonumber\\&=\hbar^{-2}\varepsilon_{ikl}\varepsilon_{jmn}\left(\op x_k[\op P_l,\op x_m]\op P_n+\op x_m[\op x_k,\op P_n]\op P_l\right)=\nonumber\\&=i\hbar^{-1}\varepsilon_{ikl}\varepsilon_{jmn}(-\op x_k\op P_n\delta_{lm}+\op x_m\op P_l\delta_{kn})=\nonumber\\&=i\hbar^{-1}(-\varepsilon_{ikl}\varepsilon_{jln}\op x_k\op P_n+\varepsilon_{ikl}\varepsilon_{jmk}\op x_m\op P_l)=\nonumber\\&=i\hbar^{-1}(-(\delta_{in}\delta_{kj}-\delta_{ij}\delta_{kn})\op x_k\op P_n+(\delta_{lj}\delta_{im}-\delta_{ij}\delta_{lm})\op x_m\op P_l)=\nonumber\\&=i\hbar^{-1}(\op x_i\op P_j-\op x_j\op P_i)=\nonumber\\&=i\varepsilon_{ijk}\op L_k\label{eq:commutation}
	\end{align}
	Si noti che, posto $\Op{L}^2=\op L_i\op L_i$, si ha
	\[[\op L_i,\Op{L}^2]=0\]
	dato che
	\[[\op L_i,\op L_j\op L_j]=i\varepsilon_{ijk}(\op L_j\op L_k+\op L_k\op L_j)=0\]
	\subsection{Trasformazione degli operatori}
	L'operatore $\op U(R)$ su $\H$ induce la mappa su $\L (\H)$ che manda
	\[\op A\mapsto \op A'=\op U^\dagger(R)\op A\op U(R)\]
	Per rotazioni infinitesime, è $\op U(R)=\op\1-i\bm\theta\cdot\Op L+o(\theta)$, quindi è
	\bal 
	\op A'&=(\op\1+i\bm\theta\cdot\Op{L}+o(\theta))\op A(\op\1-i\bm\theta\cdot\Op{L}+o(\theta))=\\&=\op A+i[\bm\theta\cdot\Op{L},\op A]+o(\theta)
	\eal
	Si deduce quindi che $\op A$ è invariante sotto rotazioni se e solo se commuta con tutte le componenti di $\Op L$. In tal caso diciamo che $\op A$ è uno scalare.
	
	Se invece abbiamo una tripletta $(\op A_1,\op A_2,\op A_3)$ di operatori con le regole di commutazione
	\[[\op L_i,\op A_j]=i\varepsilon_{ijk}\op A_k\]
	diciamo che $\Op A=(\op A_1,\op A_2,\op A_3)$ è un vettore. Si mostra facilmente che $\Op A^2$ è sempre uno scalare. In particolare, la posizione, l'impulso e il momento angolare sono vettori, come si dimostra usando la definizione di $\Op L$.
	\section{Autostati del momento angolare}
	Dalle regole di commutazione \ref{eq:commutation} è chiaro che non possiamo diagonalizzare simultaneamente le tre componenti di $\Op L$, al contrario di quanto siamo riusciti a fare con posizione e impulso. Possiamo però diagonalizzare simultaneamente una componente di $\Op L$ e $\Op L^2$. Scegliamo, come di consueto, di diagonalizzare $\op L_z$ e $\Op L^2$. Dato che $\op L_x$ e $\op L_y$ commutano con $\Op L^2$, ma non con $\op L_z$, è noto che esiste almeno un autovalore di $\Op{L}^2$ degenere. 
	\subsection{Autostati astratti}
	Vediamo ora gli autostati veri e propri: in astratto, vogliamo risolvere le equazioni agli autovalori
	\[\op L_z\ket{\l,m}=m\ket{j,m},\qquad\qquad\Op L^2\ket{\l,m}=\alpha_\l\ket{\l,m}\]
	Definiamo a tal proposito gli operatori di salita e discesa
	\[\op L_\pm=\op L_x\pm i\op L_y\]
	le cui relazioni di commutazione sono
	\[[\op L_+,\op L_-]=2\op L_z,\qquad\qquad[\op L_z,\op L_\pm]=\pm\op L_\pm\]
	Si noti che è
	\bal 
	\Op L^2&=\left(\frac{\op L_++\op L_-}{2}\right)^2+\left(\frac{\op L_+-\op L_-}{2i}\right)^2+\op L_z^2=\\&=\op L_+\op L_-+\op L_z^2-\op L_z=\\&=\op L_-\op L_++\op L_z^2+\op L_z
	\eal
	e dunque
	\bal 
	\op L_z\op L_\pm\ket{\l,m}&=(\pm\op L_\pm+\op L_\pm\op L_z)\ket{\l,m}=\\&=(m\pm 1)\op L_\pm\ket{\l,m}\\\Op L^2\op L_\pm\ket{\l,m}&=\alpha_\l\op L_\pm\ket{\l,m}\eal
	Quindi gli operatori di salita e di discesa permettono di passare da un autospazio di $\op L_z$ a un altro, rimando all'interno dello stesso autospazio di $\Op L^2$. Notiamo ora che $\Op L^2-\op L_z^2$ è un operatore positivo, quindi deve essere
	\begin{equation}\label{eq:eigenvaluesineq}\alpha_\l^2-m^2\geq0\end{equation}
	Sia $M$ l'autovalore massimo per $\op L_z$, a un dato $\l$. Deve essere
	\[\op L_+\ket{\l,M}=0\]
	e dunque agendo con $\Op L^2$ si ottiene
	\bal 
	\alpha_\l\ket{\l,M}&=\Op L^2\ket{\l,M}=\\&=(\op L_-\op L_++\op L_z^2+\op L_z)\ket{\l,M}=\\&=M(M+1)\ket{\l,M}
	\eal
	è quindi $\alpha_\l=M(M+1)$. Possiamo identificare $\l$ con $M$, e scrivere semplicemente
	\[\Op L^2\ket{\l,m}=\l(\l+1)\ket{\l,m},\qquad\qquad\op L_z\ket{\l,m}=m\ket{\l,m}\]
	Si noti che la \ref{eq:eigenvaluesineq} limita anche dal basso i valori di $m$. Questo significa che esiste $k\in \N$ tale che
	\[\op L_-\ket{\l,\l-k}=0\]
	Si deduce allora
	\bal 
	\Op L^2\ket{\l,\l-k}&=(\op L_+\op L_-+\op L_z^2-\op L_z)\ket{\l,\l-k}=\\&=((\l-k)^2-(\l-k))\ket{\l,\l-k}
	\eal
	deve quindi essere
	\[(\l-k)^2-(\l-k)=\l(\l+1)\]
	e dunque
	\[k=2\l\]
	Questo significa che $\l$ è intero o semi-intero. $m$ può quindi assumere i $2\l+1$ valori compresi tra $-\l$ e $\l$, quindi l'autovalore $\l(\l+1)$ di $\Op L^2$ è degenere $2\l+1$ volte.
	\subsection{Elementi di matrice del momento angolare}
	Gli elementi di matrice di $\Op L^2$ e $\op L_z$ sono chiaramente
	\[\braketop{\l,m}{\Op L^2}{\l',m'}=\l(\l+1)\delta_{\l,\l'}\delta_{m,m'},\qquad\qquad\braketop{\l,m}{\op L_z}{\l',m'}=m\delta_{\l,\l'}\delta_{m,m'}\]
	Costruiamo anche gli elementi di matrice di $\op L_\pm$. Sappiamo che
	\[\op L_\pm\ket{\l,m}=k_\pm(\l,m)\ket{\l,m\pm1}\]
	ma dobbiamo determinare la costante moltiplicativa. Notiamo che
	\bal 
	\l(\l+1)&=\braketop{\l,m}{\Op L^2}{\l,m}=\\&=\braketop{\l,m}{\op L_+\op L_-+\op L_z^2-\op L_z}{\l,m}=\\&=m^2-m+\sum_{\l',m'}\braketop{\l,m}{\op L_+}{\l',m'}\braketop{\l',m'}{\op L_-}{\l,m}=\\&=m^2-m+\sum_{\l',m'}|\braketop{\l,m}{\op L_+}{\l',m'}|^2=\\&=m^2-m+|k_+(\l,m-1)|^2
	\eal
	Usando la seconda scrittura di $\Op L^2$ in termini di $\op L_\pm $ e $\L_z$, si mostra allo stesso modo
	\[\l(\l+1)=m^2+m+|k_-(\l,m-1)|^2\]
	Le fasi dei due coefficienti vanno fissate. Seguiamo la convenzione di Condon-Shortley, che prende la radice positiva. In altre parole, è
	\[\braketop{\l,m-1}{\op L_-}{\l,m}=\braketop{\l,m}{\op L_+}{\l,m-1}=\sqrt{(\l+m)(\l-m+1)}\]
	o, in termini delle azioni degli operatori,
	\bal 
	\op L_+\ket{\l,m}&=\sqrt{\l(\l+1)-m(m+1)}\ket{\l,m+1}\\\op L_-\ket{\l,m}&=\sqrt{\l(\l+1)-m(m-1)}\ket{\l,m-1}
	\eal
	\subsection{Armoniche sferiche}
	Come spazio di Hilbert, prendiamo $\H=\mathbb L^2(\S ^2)$, con il prodotto interno
	\[(f,g)=\int \d \Omega\,f^*(\Omega)g(\Omega)\]
	dove $\d \Omega=\sin\theta\,\d \theta\,\d \phi$ è l'usuale elemento di angolo solido. Si noti che $\d \Omega$ è invariante sotto rotazioni, come deve essere. Infatti, è
	\[\d ^3x=r^2 \,\d r\, \d \Omega\]
	ma $\d ^3x$ e $r$ sono invarianti sotto rotazioni, dunque anche $\d \Omega$ lo è. In tale spazio, gli operatori $\op L_\pm$, $\op L_z$ e $\Op L^2$ sono dati rispettivamente da
	\bal 
	\op L_\pm&=e^{\pm i\phi}\left(\pm\pder{}{\theta}+i\cot\theta\pder{}{\phi}\right)\\\op L_z&=\frac{1}{i}\pder{}{\phi}\\\Op L^2&=-\left[\frac{1}{\sin\theta}\pder{}{\theta}\left(\sin\theta\pder{}{\theta}\right)+\frac{1}{\sin^2\theta}\pder[2]{}{\phi}\right]
	\eal
	Poniamo ora $\braket{\theta,\phi}{\l,m}=Y_{\l, m}(\theta,\phi)$. Deve essere
	\[\frac{1}{i}\pder{Y_{\l,m}}{\phi}=mY_{\l,m}\]
	ossia
	\[Y_{\l,m}(\theta,\phi)=f_{\l,m}(\theta)e^{im\phi}\]
	Fisicamente è sensato richiedere $Y_{\l,m}(\theta,\phi+2\pi)=Y_{\l,m}(\theta,\phi)$, ma questo implica $m\in\Z$. Ricordando che $m$ assume i valori $-\l,-\l+1,\dots,\l-1,\l$, deve essere anche $\l\in\N$. Vedremo dopo come conciliare questo fatto con il risultato $2\l\in\N$, ottenuto nella sezione precedente.
	
	L'equazione per $\Op{L}^2$ diventa
	\[\frac{1}{\sin\theta}\pder{}{\theta}\left(\sin\theta\pder{f_{\l,m}}{\theta}\right)-\frac{m^2}{\sin^2\theta}f_{\l,m}+\l(\l+1)f_{\l,m}=0\]
	e, facendo il cambio di variabile $\xi=\sin\theta$, si trova
	\[\der{}{\xi}\left((1-\xi^2)\der{f_{\l,m}}{\xi}\right)-\frac{m^2}{1-\xi^2}f_{\l,m}+\l(\l+1)f_{\l,m}=0\]
	La soluzione di questa equazione è ben nota. La soluzione completa è l'armonica sferica
	\[Y_{\l,m}(\theta,\phi)=\sqrt{\frac{2\l+1}{4\pi}\frac{(\l+m)!}{(\l-m)!}}e^{im\phi}P^m_\l(\cos\theta)\]
	dove
	\bal P^m_\l(x)&=(-)^\l(1-x^2)^{|m|/2}\der[m]{}{x}P_\l(x)\\P_\l(x)&=\frac{1}{2^\l\l!}\der[\l]{}{x}(x^2-1)^\l\eal
	sono, rispettivamente, i polinomi di Legendre generalizzati e i polinomi di Legendre. 
	
	Le armoniche sferiche formano un insieme completo e ortonormale per $\H$ e godono delle proprietà
	\bal 
	Y^*_{\l,m}(\theta,\phi)&=(-)^m Y_{\l,-m}(\theta,\phi)\\
	Y_{\l,m}(\pi-\theta,\phi+\pi)&=(-)^\l Y_{\l,m}(\theta,\phi)\\
	\eal
	Valgono inoltre l'espansione del potenziale coulombiano
	\[\frac{1}{|\vec{r}-\vec{r}'|}=\sum_{n=0}^{+\infty}\frac{(\min\left\{|\vec{r}|,|\vec{r}'|\right\})^n}{(\max\left\{|\vec{r}|,|\vec{r}'|\right\})^{n+1}}P_n\left(\frac{\vec{r}\cdot\vec{r}'}{|\vec{r}||\vec{r}'|}\right)\]
	e la formula di somma
	\[P_\l(\cos\alpha)=\frac{4\pi}{2\l+1}\sum_{m=-\l}^{\l}Y_{\l,m}(\theta,\phi)Y^*_{\l,m}(\theta',\phi')\]
	dove $\alpha$ è l'angolo tra $(\sin\theta\cos\phi,\sin\theta\sin\phi,\cos\theta)$ e $(\sin\theta'\cos\phi',\sin\theta'\sin\phi',\cos\theta')$.
	Si noti infine che il laplaciano in coordinate sferiche si scrive nella forma
	\[\lap f=\frac{1}{r}\pder[2]{}{r}\left(r\pder{f}{r}\right)+\frac{\Op L^2}{r^2}f\]
	\section{Gruppo delle rotazioni}
	Studiamo più in dettaglio $\SO(3)$. Consideriamo una rotazione infinitesima $R$
	\[R_{ij}=\delta_{ij}+\varepsilon M_{ij}\]
	La condizione $R^t=R^{-1}$ implica
	\[M_{ij}=-M_{ji}\]
	Dato che le matrice $3\times3$ antisimmetriche formano un sottospazio delle matrici $3\times3$ di dimensione 3, avremo tre generatori. Questi sono convenzionalmente
	\bal
	i\Sigma_1&=\left(\begin{array}{c c c}
		0&0&0\\0&0&1\\0&-1&0
	\end{array}\right)\\
	i\Sigma_2&=\left(\begin{array}{c c c}
	0&0&-1\\0&0&0\\1&0&0
\end{array}\right)\\
	i\Sigma_3&=\left(\begin{array}{c c c}
	0&1&0\\-1&0&0\\0&0&0
\end{array}\right)
	\eal
	ossia $i(\Sigma_i)_{jk}=\varepsilon_{ijk}$. Le regole di commutazione per tali generatori sono
	\[[\Sigma_i,\Sigma_j]=i\varepsilon_{ijk}\Sigma_k\]
	Dunque le costanti di struttura di $\textrm{so}(3)$ nella base delle matrici $\Sigma$ sono $\varepsilon_{ijk}$. In termini dei generatori, una rotazione di $\bm\theta=\theta\vec{n}$ è
	\[\exp(i\bm\theta\cdot\bm\Sigma)\]
	Infatti, preso un vettore $\vec{v}$ siano $\vec{v}_\perp$ e $\vec{v}_\parallel$ le componenti di $\vec{v}$ ortogonali e parallele a $\vec n$
	\[\vec{v}_\perp=\vec{v}-(\vec{v}\cdot\vec{n})\vec{n},\qquad\qquad\vec{v}_\parallel=(\vec{v}\cdot{\vec{n}})\vec{n}\]
	Si noti ora che
	\begin{align*}
		(i\bm\theta\cdot\bm\Sigma)\vec{v}&=-\theta n_i\varepsilon_{ijk}v_k\hat{x}_j=\\&=\bm\theta\times\vec{v}\\
		(i\bm\theta\cdot\bm\Sigma)(i\bm\theta\cdot\bm\Sigma)\vec{v}&=\bm\theta\times(\bm\theta\times\vec{v})=\\&=-\theta^2\vec{v}_\perp
	\end{align*}
	Notando che $\vec{n}\times\vec{v}=\vec{n}\times\vec{v}_\perp$, è chiaro che
	\[(i\bm\theta\cdot\bm\Sigma)^{2k}\vec{v}=(-)^k\theta^{2k}\vec{v}_\perp,\qquad\qquad(i\bm\theta\cdot\bm\Sigma)^{2k+1}\vec{v}=(-)^k\theta^{2k+1}\vec{n}\times\vec{v}\]
	e dunque
	\bal
	\exp(i\bm\theta\cdot\bm\Sigma)\vec{v}&=\vec{v}+\sum_{k=1}^{+\infty}\frac{(-)^k}{(2k)!}\theta^{2k}\vec{v}_\perp+\sum_{k=0}^{+\infty}\frac{(-)^k}{(2k+1)!}\theta^{2k+1}\vec{n}\times\vec{v}=\\&=\vec{v}+\vec{v}_\perp\cos\theta+\vec{n}\times\vec{v}\sin\theta
	\eal
	Abbiamo quindi una mappa suriettiva $\exp\colon\mathrm{so}(3)\to\SO(3)$.
	\subsection{Rappresentazioni di algebre}
	Premettiamo due definizioni.
	\begin{definition}
		Sia $V$ un $\C$-spazio. L'algebra di Lie $\mathcal{GL}(V)$ degli operatori lineari su $V$ è lo spazio $\GL(V)$, dotato del prodotto di Lie
		\[[\op A,\op B]=\op A\op B-\op B\op A\]
	\end{definition}
	\begin{definition}
		Una rappresentazione $\rho$ di un'algebra di Lie $\mathcal{L}$ su un $\C$-spazio $V$ è un omomorfismo di algebre di Lie $\rho\colon\L\to\mathcal{GL}(V)$.
	\end{definition}
	Da qui in poi le nozioni usuali della teoria delle rappresentazioni, come l'irriducibilità, non subiscono modifiche. Sappiamo ora che possiamo mappare tutto $\SO(3)$ a partire da $\mathrm{so}(3)$. Sembra quindi ragionevole cercare le rappresentazioni irriducibili di $\mathrm{so}(3)$ e ricavarne quelle di $\SO(3)$. Più precisamente, se $\rho\colon\mathrm{so}(3)\to\mathcal{GL}(V)$ è una rappresentazione irriducibile di $\mathrm{so}(3)$, vorremmo dire che esite una rappresentazione $\tilde{\rho}\colon\SO(3)\to\mathcal{GL}(V)$ che faccia commutare il diagramma
	\begin{figure}[h!]
		\centering
	\begin{tikzcd}
		\mathrm{so}(3)\arrow [r, "\rho"]\arrow [d, "\exp" left]&\mathcal{GL}(V)\\\SO(3)\arrow[ru, "\tilde{\rho}" right]&
	\end{tikzcd}
\end{figure}


	Iniziamo notando che abbiamo in realtà già trovato le rappresentazioni irriducibili di $\mathrm{so}(3)$: i suoi generatori soddisfano le stesse regole di commutazione del momento angolare, dunque ha anche le stesse rappresentazioni. Queste sono le rappresentazioni sugli spazi
	\[V_\l=\langle\left\{\ket{\l,m}:m=-\l,-\l+1,\dots,\l-1,\l\right\}\rangle\]
	definite sui generatori da
	\[\rho\colon\Sigma_i\mapsto\op L_i\]
	Verifichiamo l'irriducibilità: sia $W\subseteq V_\l$ un sottospazio $\mathrm{so}(3)$-invariante e non vuoto. Se $\ket{w}\in W$, sicuramente è
	\[\ket{w}=\sum_{m=-\l}^{\l}\alpha_m\ket{\l,m}\]
	Sia $\overline{m}=\min\left\{m:\alpha_m\neq0\right\}$. Deve essere
	\[(\rho(\Sigma_1)+i\rho(\Sigma_2))^{\l-\overline{m}}\ket{w}\in W\]
	ma questo significa ovviamente $\ket{\l,\l}\in W$. Preso $k\in1,\dots,2\l$, è allora
	\[(\rho(\Sigma_1)-i\rho(\Sigma_2))^k\ket{\l,\l}=\ket{\l,\l-k}\in W\]
	e quindi $W=V_\l$.
	
	Prendiamo ora come candidata per $\tilde{\rho}_\l\colon\SO(3)\to\mathcal{GL}(V_\l)$ la mappa
	\[\tilde{\rho}_\l\colon\exp(i\bm\theta\cdot\bm\Sigma)\mapsto\exp(i\bm\theta\cdot\Op{L})\]
	dove, si intende, le matrici associate a $\Op L$ sono quelle a fissato $\l$. C'è un problema: non tutte le $\tilde{\rho}_\l$ appena definite sono rappresentazioni. Infatti, prendiamo $\bm\theta=2\pi\hat{z}$: è $\exp(i\bm\theta\cdot\bm\Sigma)=\mathrm{Id}$, ma è
	\[\exp(i\bm\theta\cdot\Op L)\ket{\l,m}=e^{2\pi im}\ket{\l, m}\]
	che \emph{non} è l'identità se $m$ (e dunque $\l$) è semi-intero. Vediamo come si risolve questo dilemma
	\section{$\SU(2)$}
	$\SU(2)$ è il gruppo delle matrici unitarie $2\times2$ con determinante 1. La più generica matrice di tale forma è
	\[\left(\begin{array}{c c}a&b\\-b^*&a^*\end{array}\right)\]
	con la condizione aggiuntiva
	\[|a|^2+|b|^2=1\]
	Si deduce quindi che $\SU(2)$ è uno spazio vettoriale reale di dimensione 3 isomorfo a $\S^3\subseteq \R ^4$, tramite la mappa $\pi\colon\S ^3\to\SU(2)$ definita da
	\[\pi\colon (x_1,x_2,x_3,x_4)\mapsto\left(\begin{array}{c c}
	x_1+ix_2&x_3+ix_4\\x_3-ix_4&x_1-ix_2
	\end{array}\right)\]
	che è anche un isomorfismo di gruppi. Questo implica un'importante proprietà: $\SU(2)$ è semplicemente connesso. Vediamone i generatori: se $M\in\SU(2)$ e
	\[M=1+i\varepsilon A+o(\varepsilon)\]
	deve essere
	\[1=M^\dagger M=(1-i\varepsilon A^\dagger)(1+i\varepsilon A)+o(t)=1+i\varepsilon(A-A^\dagger)+o(\varepsilon)\]
	I generatori sono quindi matrici Hermitiane. Dato che è $M=\exp(i\varepsilon A)$, se $U$ diagonalizza $A$ e questa ha autovalori $\lambda_1$ e $\lambda_2$, è
	\bal\det M&=\det(UMU^\dagger)=\\&=\det\exp(i\varepsilon UAU^{-1})=\\&=e^{i\varepsilon(\lambda_1+\lambda_2)}
	\eal
	otteniamo quindi la condizione $\mathrm{tr} A=0$. Un set di generatori è dato dalle matrici di Pauli
	\bal 
	\sigma_1&=\left(\begin{array}{c c}
	0&1\\1&0	
	\end{array}\right)\\
	\sigma_2&=\left(\begin{array}{c c}
	0&-i\\i&0
	\end{array}\right)\\
	\sigma_3&=\left(\begin{array}{c c}
	1&0\\0&-1
	\end{array}\right)\\
	\eal
	Chiamiamo $\mathrm{su}(2)$ l'algebra di Lie generata da queste matrici. Dato che
	\[\sigma_i\sigma_j=\delta_{ij}+i\varepsilon_{ijk}\sigma_k\]
	si deduce che, per $s_i=\sigma_i/2$, è
	\[[s_i,s_2]=i\varepsilon_{ijk}s_k\]
	Ma allora la mappa $\pi\colon\mathrm{su}(2)\to\mathrm{so}(3)$ definita da
	\[\rho\colon s_i\mapsto\Sigma_i\]
	è un isomorfismo di algebre di Lie. Le rappresentazioni di $\mathrm{su}(2)$ sono quelle già viste. Questo \emph{non} implica in alcun modo un isomorfismo tra $\SU(2)$ e $\SO(3)$. Piuttosto, si consideri l'omomorfismo di gruppi $\rho\colon\SU(2)\to\SO(3)$ dato da
	\[\rho\colon\exp\left(i\bm\theta\cdot\frac{\bm\sigma}{2}\right)\mapsto \exp(i\bm\theta\cdot\bm\Sigma)\]
	Questa mappa è chiaramente suriettiva, ma non è iniettiva. Il suo nucleo è $\ker\rho=\left\{\pm1\right\}$, quindi per il primo teorema di omomorfismo è $\SO(3)\cong\SU(2)/\Z _2$. Le rappresentazioni di $\mathrm{su}(2)$ sono estendibili a rappresentazioni di $\SU(2)$, ma solo quelle con $\l$ intero sono effettivamente rappresentazioni di $\SO(3)$.
	
	Mostriamo incidentalmente che esiste una formula chiusa per $\exp\left(i\bm\theta\cdot\frac{\bm\sigma}{2}\right)$. Usando le regole di moltiplicazione delle matrici di Pauli è
	\bal 
	(\bm\theta\cdot\bm\sigma)^2&=\theta^2n_in_j(\delta_{ij}+i\varepsilon_{ijk}\sigma_k)=\\&=\theta^2
	\eal
	e quindi
	\bal 
	(\bm\theta\cdot\bm\sigma)^{2k}&=\theta^{2k}\\
	(\bm\theta\cdot\bm\sigma)^{2k+1}&=\theta^{2k+1}\vec{n}\cdot\bm\sigma\\\exp\left(i\bm\theta\cdot\frac{\bm\sigma}{2}\right)&=\sum_{k=0}^{+\infty}\frac{(-)^k\theta^{2k}}{2^{2k}(2k)!}+i\vec{n}\cdot\bm\sigma\sum_{k=0}^{+\infty}\frac{(-)^k\theta^{2k+1}}{2^{2k+1}(2k+1)!}=\\&=\cos\frac{\theta}{2}+i\vec{n}\cdot\bm\sigma\sin\frac{\theta}{2}
	\eal
	\section{Spin}
	Definiamo lo spin di un certo sistema come il momento angolare intrinseco, ovvero come il momento angolare nel sistema del centro di massa. Indichiamo lo spin con $\Op{s}$. Le sue regole di commutazione sono quelle usuali per un momento angolare
	\[[\op s_i,\op s_j]=i\varepsilon_{ijk}\op s_k\]
	Dobbiamo capire su quali variabili agisce $\Op{s}$. Avevamo descritto l'azione  del momento angolare orbitale $\hbar\Op L=\Op x\times\Op p$ sugli autostati della posizione (o dell'impulso). Per $\Op s$ invece le variabili naturali sono gli autovalori di una delle sue componenti, ad esempio $\op s_z$. Per una particella libera, il momento angolare intrinseco è una sua caratteristica, quindi ha un valore ben definito. Questo significa che ci troviamo in una qualche rappresentazione di $\SU(2)$, ovvero che
	\[\Op s^2=s(s+1)\]
	per un qualche $s$. Al solito, gli autovalori di $\op s_z$ (indichiamoli con $\sigma$) assumono i valori $|\sigma|\leq s$. La funzione d'onda di uno stato generico $\ket{\psi}$ necessita allora di due coordinate, ossia la posizione e il valore di $\sigma$. Abbiamo cioè
	\[\psi(\vec{x},\sigma)=\braket{\vec{x},\sigma}{\psi}\]
	A seconda dei gusti, si può anche dire che la funzione d'onda del sistema è in realtà un vettore di funzioni d'onda con $2s+1$ componenti, ossia la funzione d'onda è
	\[\bm\psi(\vec{x})=\left(\begin{array}{c}
	\psi(\vec{x},-s)\\\psi(\vec{x},-s+1)\\\vdots\\\psi(\vec{x},s-1)\\\psi(\vec{x},s)
	\end{array}\right)\]
	Il prodotto scalare fra stati è definito in maniera naturale come
	\[\braket{\varphi}{\psi}=\int\d ^3\vec{x}\sum_{\sigma=-s}^s\varphi^*(\vec{x},\sigma)\psi(\vec{x},\sigma)=\int\d ^3\,\vec{x}\bm\varphi^*(\vec{x})\cdot\bm\psi(\vec{x})\]
	e, al solito, $|\psi(\vec{x},\sigma)|^2$ si interpreta come la densità di probabilità di trovare la particella in $\vec{x}$ con componente $z$ dello spin pari a $\sigma$.
	\subsection{Trasformazione delle funzioni d'onda}
	Introduciamo le matrici $(2s+1)\times(2s+1)$ $D^{(s)}(R)$, definite da
	\[D^{(s)}_{\sigma,\sigma'}(R)=\braketop{s,\sigma'}{\op U(R)}{s,\sigma}\]
	(Si noti l'ordine inverso in $\sigma$ e $\sigma'$ tra primo e secondo membro. In tal modo l'azione di $\op U(R)$ su un ket si ottiene in forma matriciale contraendo il secondo indice di $D^{(\l)}(R)$, come si fa usualmente nel prodotto matrice-vettore colonna). Per tali matrici vale
	\[[D^{(s)}(R)]^\dagger=D^{(s)}(R^{-1})\]
	In tal modo, è
	\bal 
	\op U(R)\ket{\vec{x},s,\sigma}&=D^{(s)}_{\sigma,\sigma'}(R)\ket{R\vec{x},s,\sigma'}	\\\bra {\vec{x},s,\sigma}\op U(R)&=D^{(s)}_{\sigma,\sigma'}(R)\bra {R^{-1}\vec{x},s,\sigma'}
	\eal
	Quindi una funzione d'onda per una particella di spin $s$ trasforma come
	\[\psi_R(\vec{x},\sigma)=\braketop{\vec{x},s,\sigma}{\op U(R)}{\psi}=\sum_{\sigma'}D^{(s)}_{\sigma,\sigma'}(R)\psi(R^{-1}\vec{x},s,\sigma')\]
	\section{Momento angolare totale}
	Consideriamo ora il caso generale di una particella con spin e momento angolare orbitale. Sia $\Op J$ il generatore delle rotazioni, che chiamiamo momento angolare totale. Per definizione è
	\[\op U(R)=\exp(i\bm\theta\cdot\Op J)\]
	Abbiamo visto nella sezione precedente l'azione di $\op U(R)$ su una funzione d'onda. Per una rotazione infinitesima è
	\bal \exp(i\bm\theta\cdot\Op J)&=1+i\bm\theta\cdot\Op J+o(\theta)\\D^{(s)}(R)&=1+i\bm\theta\cdot\Op s+o(\theta)\\ R\vec{x}&=\vec{x}-i\bm\theta\cdot\Op L\vec{x}+o(\theta)\eal
	Deve quindi essere
	\bal \psi(\vec{x},\sigma)+i\bm\theta\cdot\Op J\psi(\vec{x},\sigma)&=\sum_{\sigma'}D^{(s)}_{\sigma,\sigma'}(R)\left[\psi(\vec{x},s,\sigma')+i\bm\theta\cdot\Op L\psi(\vec{x},\sigma')\right]=\\&=\psi(\vec{x},\sigma)+i\bm\theta\Op L\psi(\vec{x},\sigma)+i\bm\theta\cdot\op s_{\sigma,\sigma'}\psi(\vec{x},\sigma')\eal
	Il momento angolare totale è quindi
	\[\Op J=\Op L+\Op s\]
	Sotto rotazioni, la grandezza conservata è $\Op J$, mentre in generale non si conservano separatamente $\Op L$ e $\Op s$. Questo è dovuto al fatto che
	\[[\op J_i,\op L_j]=[\op L_i,\op L_j]\neq0\]
	e analogamente per lo spin. Si noti infatti che $[\op L_i,\op s_j]=0$. Dimostriamo ora un'importante proprietà sulla composizione di momenti angolari: sia $\rho_n\colon\SU(2)\to\mathcal{GL}(V_n)$ la rappresentazione di $\SU(2)$ di dimensione $2n+1$. Allora è
	\[V_{\l_1}\otimes V_{\l_2}\cong\bigoplus_{\l=|\l_1-\l_2|}^{\l_1+\l_2}V_\l\]
	In un certo senso, questa decomposizione è analoga alle disuguaglianze che abbiamo classicamente per due vettori qualunque $\vec{a}$ e $\vec{b}$:
	\[\left||\vec{a}|-|\vec{b}|\right|\leq\left|\vec{a}+\vec{b}\right|\leq|\vec{a}|+|\vec{b}|\]
	Per la dimostrazione, supponiamo wlog $\l_1\leq\l_2$ e sia 
	\[\mathcal{B}=\left\{\ket{\l_1,m_1}\otimes\ket{\l_2,m_2}:m_1=-\l_1,-\l_1+1,\dots,\l_1-1,\l_1,\,m_2=-\l_2,-\l_2+1,\dots,\l_2-1,\l_2\right\}\]
	la base naturale di $V_{\l_1}\otimes V_{\l_2}$. La componente $z$ del momento angolare $\Op L=\Op L_1\otimes\op\1_2+\op\1_1\otimes\Op L_2$ agisce sui vettori di base come
	\[\op L_z\ket{\l_1,m_1}\otimes\ket{\l_2,m_2}=(m_1+m_2)\ket{\l_1,m1}\otimes\ket{\l_2,m_2}\]
	Dunque l'autovalore massimo di $\op L_z$ è $\l_1+\l_2$, ed ha come unico autovettore $\ket{\l_1,\l_1}\otimes\ket{\l_2,\l_2}$. Più in generale, poniamo
	\[\mathcal{B}_k=\left\{\ket{\l_1,m_1}\otimes\ket{\l_2,m_2}\in\mathcal{B}:m_1+m_2=k\right\}\]
	In tal modo, $|\mathcal{B}_k|$ è il numero di vettori di base con momento angolare lungo $z$ pari a $k$. Mostriamo alla fine della sezione che per $\l_2-\l_1\leq k\leq \l_1+\l_2$ è $|\mathcal{B}_k|=\l_1+\l_2+1-k$. Con questo risultato abbiamo concluso: abbiamo infatti $k$ stati con componente $\op L_z$ pari a $\l_1+\l_2+1-k$: partendo da $\ket{\l_1,\l_1}\otimes\ket{\l_2,\l_2}$, possiamo trovare una copia di $V_{\l_1+\l_2}$ all'interno di $V_{\l_1}\otimes V_{\l_2}$ tramite l'operatore di discesa $\op L_-$. Ci rimane così un vettore $\ket{u}$ indipendente da $\op L_-\ket{\l_1,\l_1}\otimes\ket{\l_2,\l_2}$ con $\op L_z$ pari a $\l_1+\l_2-1$. Questo vettore appartiene necessariamente a una copia $V_\l$ con $\l\geq\l_1+\l_2-1$, ma non può essere $\l\geq\l_1+\l_2$ perchè altrimenti agendo con $\op L_+$ su $\ket{u}$ troveremmo altri autostati di $\op L_z$, che abbiamo già mostrato che non esistono. Quindi è $\l=\l_1+\l_2-1$ e tramite $\op L_-$ costruiamo una copia di $V_{\l_1+\l_2-1}$ all'interno di $V_{\l_1}\otimes V_{\l_2}$. Il processo continua diminuendo di 1 l'autovalore di $\op L_z$. Ad ogni passaggio troviamo \emph{una} copia di $V_{\l_1+\l_2-k}$ all'interno del prodotto tensoriale. Per trovare l'indice a cui fermarsi, basta ragionare per dimensioni: dato che $\dim V_\l=2\l+1$, è
	\bal(2\l_1+1)(2\l_2+1)&=\dim\bigoplus_{\l=n}^{\l_1+\l_2}V_\l=\\&=\sum_{k=n}^{\l_1+\l_2}(2\l+1)=\\&=\l_1+\l_2+1-n+2\sum_{\l=0}^{\l_1+\l_2}\l-2\sum_{\l=0}^{n-1}\l=\\&=\l_1+\l_2+1-n+(\l_1+\l_2)(\l_1+\l_2+1)-n(n-1)=\\&=(\l_1+\l_2+1)^2-n^2
	\eal
	Si trova così $n=|\l_2-\l_1|$, come preannunciato. Resta da mostrare la parte più importante della dimostrazione, ossia il calcolo di $|\mathcal{B}_k|$. Procediamo induttivamente su $j=\l_1+\l_2-k$: dobbiamo così mostrare $|\mathcal{B}_{\l_1+\l_2-j}|=j+1$. Per $j=0$ la tesi è vera ed è riportata sopra. Per $j\leq 0$, consideriamo l'insieme delle coppie $(m_1,m_2)$ che risolvono $m_1+m_2=\l_1+\l_2-j$. Vogliamo contare le coppie che risolvono $m_1+m_2=\l_1+\l_2-j-1$. FINISCI.
	\subsection{Coefficienti di Clebsch-Gordan}
	Dalla decomposizione precedente
	\[V_{\l_1}\otimes V_{\l_2}\cong\bigoplus_{\l=|\l_1-\l_2|}^{\l_1+\l_2}V_\l\]
	segue che abbiamo due basi naturali di $V_{\l_1}\otimes V_{\l_2}$: una è $\mathcal{B}$, l'altra è chiaramente
	\[\tilde{\mathcal{B}}=\left\{\ket{\l_1,\l_2;L,M}:|\l_1-\l_2|\leq L\leq \l_1+\l_2,\,M=-L,-L+1,\dots,L-1,L\right\}\]
	I coefficienti del cambio di base tra $\mathcal{B}$ e $\tilde{\mathcal{B}}$ sono noti come coefficienti di Clebsch-Gordan. Convenzionalmente si pone
	\[\ket{\l_1,\l_2;L,M}=\tensor*{C}{^{L\,\l_1\,\l_2}_{M\,m_1\,m_2}}\ket{\l_1,m_1}\otimes\ket{\l_2,m_2}\]
	Ovviamente è
	\[\tensor*{C}{^{L\,\l_1\,\l_2}_{M\,m_1\,m_2}}=(\bra {\l_1,m_1}\otimes\bra{\l_2,m_2})\ket{\l_1,\l_2;L,M}\]
	Vediamo qualche proprietà: abbiamo due regole di selezione, ossia due condizioni necessarie affinché $\tensor*{C}{^{L\,\l_1\,\l_2}_{M\,m_1\,m_2}}\neq0$. Esse sono, per quanto visto nella sezione precedente,
	\[|\l_1-\l_2|\leq L\leq \l_1+\l_2,\qquad\qquad M=m_1+m_2\]
	Inoltre, con la convenzione scelta per le fasi degli autostati del momento angolare si hanno dei coefficienti di Clebsch-Gordan reali. Infine, sotto scambio dei due spazi prodotto si ha
	\[(\bra{\l_1,m_1}\otimes\bra {\l_2,m_2})\ket{\l_1,\l_2;L,M}=(-)^{-L+\l_1+\l_2}(\bra{\l_2,m_2}\otimes\bra {\l_1,m_1})\ket{\l_2,\l_1;L,M}\]
	ossia
	\[\tensor*{C}{^{L\,\l_1\,\l_2}_{M\,m_1\,m_2}}=(-)^{-L+\l_1+\l_2}\tensor*{C}{^{L\,\l_2\,\l_1}_{M\,m_2\,m_1}}\]
	\subsection{Angoli di Eulero}
	Riprendiamo le matrici di rotazione
	\[D^{(j)}_{m,m'}(R)=\braketop{j,m}{\op U(R)}{j,m'}\]
	e cerchiamo un modo conveniente per calcolarle che non sia il calcolo diretto dell'esponenziale $\exp(i\bm\theta\cdot\Op J)$. Sappiamo dai corsi precedenti che una rotazione si può esprimere in termini di angoli di Eulero. In particolare, se abbiamo un sistema $S$ con assi $x,y,z$ che viene ruotato nel sistema $S'$ con assi $x',y',z'$, possiamo
	\begin{itemize}
		\item ruotare di un angolo $\alpha$ intorno a $z$ per portare il sistema $S_1=S$ nel sistema $S_2$. L'asse $y$ viene portato nell'asse $y_2$ e scegliamo $\alpha$ in modo che $y_2$ sia ortogonale al piano passante per $z$ e $z'$.
		\item Ruotare di un angolo $\beta$ intorno a $y_2$ per portare il sistema $S_2$ nel sistema $S_3$. Scegliamo $\beta$ in modo che $z$ venga portato in $z'$.
		\item Ruotare di un angolo $\gamma$ intorno a $z'$ per portare il sistema in $S_4=S'$. 
	\end{itemize}

	Se indichiamo con $\rho(\theta,\xi,S)$ una rotazione di un angolo $\theta$ intorno all'asse $\xi$ del sistema $S$ e con $\overline{\rho}(R,S)$ una rotazione $R$ del sistema $S$, la nostra rotazione si scrive come si scrive come
	\[\overline{\rho}(R,S)=\rho(\gamma,z',S_3)\rho(\beta,y_2,S_2)\rho(\alpha,z,S_1)\]
	Abbiamo un'importante proprietà: se un sistema $K$ viene ruotato in un sistema $K'$ tramite una rotazione $R_1$ e in questo facciamo una rotazione $R_2$, abbiamo
	\[\overline{\rho}(R_2,K')=\overline{\rho}(R_1,K)\overline{\rho}(R_2,K)\overline{\rho}(R_1^{-1},K)\]
	Da questo si deduce
	\[\overline{\rho}(R,K)\rho(\alpha,z,S_1)\rho(\beta,y,S_1)\rho(\gamma,z,S_1)\]
	e dunque
	\[\exp(i\bm\theta\cdot\Op J)=e^{i\alpha \op J_z}e^{i\beta\Op J_y}e^{i\gamma\Op J_z}\]
	Si ha dunque
	\[D^{(j)}_{m,m'}(R)=e^{i\alpha m}d^{(j)}_{m,m'}(\beta)e^{im'\gamma}\]
	dove
	\[d^{(j)}_{m,m'}(\beta)=\braketop{j,m}{e^{i\beta \op J_y}}{j,m'}\]
	Dobbiamo quindi solo calcolare l'elemento di matrice di una rotazione intorno all'asse $y$ per trovare quello di una rotazione generica. Esplicitamente è
	\[d^{(j)}_{m,m'}(\beta)=\sqrt{\frac{(j+m)!(j-m)!}{(j+m')!(j-m')!}}\left(\cos\frac{\beta}{2}\right)^{2j}\sum_{k}(-)^k\binom{j+m'}{k}\binom{j-m'}{j-m-\mu}\left(\tan\frac{\beta}{2}\right)^{m-m'+2k}\]
	dove la somma su $k$ è estesa a tutti i valori per cui sono definiti i binomiali. Queste matrici sono tabulate nella tabelle dei coefficienti di Clebsch-Gordan.
	\section{Rotazione degli stati}
	Consideriamo ancora una volta l'azione di $\op U(R)$ su uno stato $\ket{j,m}$. Intuitivamente, se $R\hat{z}=\vec{a}$, lo stato ruotato $\op U(R)\ket{j,m}$ è autostato di $\Op J\cdot\vec{a}$. Dimostriamolo formalmente: è
	\bal 
	(\Op J\cdot\vec{a})\op U(R)\ket{j,m}&=\op U(R)\op U^\dagger(R)(\Op J\cdot\vec{a})\op U(R)\ket{j,m}=\\&=a_i\op U(R)(\op U^\dagger(R)\op J_i\op U(R))\ket{j,m}=\\&=a_i\op U(R)R_{im}\op J_m\ket{j,m}=\\&=\op U(R)(R\Op J\cdot\vec{a})\ket{j,m}=\\&=\op U(R)(\Op J\cdot R^{-1}\vec{a})\ket{j,m}=\\&=\op U(R)\op J_z\ket{j,m}=\\&=m\op U(R)\ket{j,m}
	\eal
	Consideriamo ora un versore $\vec{n}$. A questo possiamo associare un autostato della posizione su $\S ^2$, che indichiamo indifferentemente con le notazioni $\ket{\vec{n}}$ o $\ket{\theta,\phi}$. Le armoniche sferiche sono allora
	\[Y_{\l,m}(\theta,\phi)=\braket{\theta,\phi}{\l,m}\]
	Sotto rotazioni si ha, per quanto detto
	\bal 
	\braketop{\vec{n}}{\op U(R)}{\l,m}=\\&=\braket{R^{-1}\vec{n}}{\l,m}=\\&=Y_{\l,m}(R^{-1}n)
	\eal
	D'altro canto, ricordando la decomposizione in rappresentazioni irriducibili di $\op U(R)$, è
	\[\op U(R)\ket{\l,m}=D^{(\l)}_{m,m'}(R)\ket{\l,m'}\]
	e quindi troviamo la legge di trasformazione
	\[Y_{\l,m}(R^{-1}\vec{n})=D^{(\l)}_{m,m'}Y_{\l,m'}(\vec{n})\]
	\section{Tensori sferici}
	Definiamo un tensore sferico di rango $j$ come un insieme di $2j+1$ operatori $\left\{\op T^{j}_{-j},\op T^j_{-j+1},\dots,\op T^j_{j-1},\op T^j_j\right\}$ per cui valga
	\[\op U(R)\op T^j_m\op U^\dagger(R)=D^{(j)}_{m,m'}(R)\op T^j_{m'}\]
	ossia
	\[\op U(R)\left(\begin{array}{c}
	\op T^j_{-j}\\\op T^j_{-j+1}\\\vdots\\\op T^j_{j-1}\\\op T^j_j
	\end{array}\right)\op U^\dagger(R)=D^{(j)}(R)\left(\begin{array}{c}
	\op T^j_{-j}\\\op T^j_{-j+1}\\\vdots\\\op T^j_{j-1}\\\op T^j_j
	\end{array}\right)\]
	Per una rotazione infinitesima si trova
	\[[\bm\theta\cdot\Op J,\op T^j_m]=(\bm\theta\cdot\Op J)_{m,m'}\op T^j_{m'}\]
	e prendendo le varie componenti
	\bal 
	[\op J_z,\op T^j_m]&=m\op T^j_m\\ [\op J_\pm,\op T^j_m]&=\sqrt{j(j+1)-m(m\pm 1)}\op T^j_{m\pm 1}
	\eal
	In particolare, dalla prima si deduce
	\bal 
	(m_1-m_2)\braketop{j,m_1,\alpha}{\op T^{j}_m}{j,m_2,\beta}&=\braketop{j,m_1,\alpha}{\op J_z\op T^j_m-\op T^j_m\op J_z}{j,m_2,\beta}=\\&=m\braketop{j,m_1,\alpha}{\op T^j_m}{j,m_2,\beta}
	\eal
	Se l'elemento di matrice è non nullo, deve quindi essere
	\[m_1=m_2+m\]
	Questo è un esempio di regola di selezione, ossia di condizione necessaria tra due stati affinché possa esserci una transizione da uno all'altro.
	
	Vediamone un'altra: uno stato $\op T^j_m\ket{j_2,m_2}$ trasforma sotto rotazioni come
	\bal 
	\op U(R)\op T^j_m\ket{j_2,m_2}=&=\op U(R)\op T^j_m\op U^\dagger(R)\op U(R)\ket{j_2,m_2}=\\&=D^{(j)}_{m,m'}(R)D^{(j_2)}_{m_2,m_2'}\op T^j_{m'}\ket{j_2,m_2'}
	\eal
	Lo stato trasforma quindi come $j\otimes j_2$. Ma allora se
	\[\braketop{j_1,m_1}{\op T^j_m}{j_2,m_2}\neq0\]
	deve essere $|j-j_2|\leq j_1\leq j+j_2$.
	\subsection{Teorema di Wigner-Eckart}
	C'è una stretta connessione tra i tensori sferici e i coefficienti di Clebsch-Gordan. Questa connessione è data dal teorema di Wigner-Eckart: dato un tensore sferico $\op T^{j}$ di rango $j$, allora è
	\[\braketop{j_1,m_1,\alpha}{\op T^j_m}{j_2,m_2,\beta}=\frac{\Lambda(j;j_1,\alpha;j_2,\beta)}{\sqrt{2j+1}}\tensor*{C}{^{j\,j_1\,j_2}_{m\,m_1\,m_2}}\]
	In altre parole, per l'elemento di matrice di una componente di un tensore sferico è
	\[\braketop{j_1,m_1,\alpha}{\op T^j_m}{j_2,m_2,\beta}\propto(\bra{j_1,m_1}\otimes\bra{j_2,m_2})\ket{j_1,j_2;j,m}\]
	dove la costante di proporzionalità \emph{non} dipende da $m$, $m_1$ e $m_2$. Tale costante si chiama elemento di matrice ridotto ed è usualmente scritta nella fuorviante forma
	\[\Lambda(j;j_1,\alpha;j_2,\beta)=\braketop{j_1,\alpha}{|\op T^j|}{j_2,\beta}\]
	Il teorema è particolarmente importante perchè
	\begin{itemize}
		\item Tiene automaticamente conto delle regole di selezione sui tensori sferici, poichè già implementate nei coefficienti di Clebsch-Gordan.
		\item Permette di calcolare tutti gli elementi di matrice di tutti gli operatori del tensore considerato, una volta che conosciamo un certo elemento di matrice.
		\item Tutti gli operatori hanno elementi di matrice proporzionali tra loro (con la costante di proporzionalità eventualmente nulla).
	\end{itemize}
	DIMOSTRAZIONE: DA FARE.
\end{document}
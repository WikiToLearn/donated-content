\begin{es}
  Per ogni $ m $ si $ P_m(n) $ definito per ricorsione: $ P_m(0) = 0 $ e $ P_m(n + 1) = P_m(n) + m $. Dimostrare che
  \begin{enumerate}
  \item $ P_{m + 1}(n) = P_m(n) + n $ per ogni $ n, m \in \N $;
  \item commutativa
  \item distributiva
  \item associativa
  \end{enumerate}
\end{es}

\begin{es}
  Siano $ X $ e $ Y $ due insiemi infiniti tali che $ |X| \leq |Y| $. Mostare che $ |X \cup Y| = |Y| $.
\end{es}

\begin{es}
  Siano $ X $ e $ Y $ due insiemi infiniti tali che $ |X| \leq |Y| $. Mostare che $ |X \times Y| = |Y| $.
\end{es}

\begin{es}
  Siano $ X $ e $ Y $ due insiemi infiniti tali che $ |X| \leq |Y| $. Mostare che $ |X^Y| = |2^Y| $
\end{es}

\begin{es}
  Siano $ X $ e $ Y $ due insiemi infiniti. Mostrare che se $ |X| = |2^Z| $ per qualche $ Z $ allora $ |X^Y| = \max\{|X|, |2^Y|\} $. Trovare, se possibile, un esempio di un insieme $ X $ infinito con $ |X| \geq |\R| $ tale che $ |X^\N| > |X| $.
\end{es}

\begin{es}
  Dato $ X $ insieme infinito, mostrare che esiste una partizione di $ X $ fatta di sottoinsiemi numerabili. Che cardinalità ha questa partizione? Mostrare che esiste una partizione numerabile di insiemi di cardinalità uguale a $ X $.
\end{es}

\begin{es}
  Dato uno spazio vettoriale $ V $, si mostri che data una base di Hamel $ B $ e un sottoinsieme $ A $ di vettori linearmente indipendenti, esiste $ B' \subset B $ tale che $ A \cup B' $ è ancora una base. Si dimostri inoltre che due basi $ B_1 $ e $ B_2 $ hanno la stessa cardinalità.
\end{es}

\begin{es}
  Se $ X $ è un insieme infinito, qual è la cardinalità dell'insieme delle partizioni di $ X $? Per insieme delle partizioni si intende tutte le collezioni $ \mathscr{C} \subset \P(X) $ tali che $ A, B \in \mathscr{C} $ implichi $ A = B $ o $ A \cap B = \emptyset $ e inoltre si abbia $ \bigcup_{A \in \mathscr{C}} A = X $.
\end{es}

\begin{es}
  Se $ X $ è un insieme infinito, qual è la cardinalità dell'insieme delle relazioni di equivalenza su $ X $?
\end{es}

\begin{es}
  Se $ X $ è un insieme infinito, qual è la cardinalità dell'insieme delle relazioni d'ordine parziale su $ X $? La cardinalità di quelle di ordine totale?
\end{es}
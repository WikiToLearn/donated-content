\documentclass[a4paper,11pt]{book}
	 \usepackage[a4paper, left=2.5cm, bottom=2.5cm]{geometry}
     \usepackage[italian]{babel}
     \usepackage[utf8]{inputenc}
     \usepackage{enumitem}
     \usepackage{siunitx}
     \usepackage{tikz}
     \usepackage{tensor}
	 \usetikzlibrary{circuits}
     \usepackage{circuitikz}
     \usepackage{graphicx}
     \usepackage{amsfonts}
     \usepackage{amsthm}
     \usepackage{amsmath}
     \usepackage{mathrsfs}
     \usepackage{bm}
     
     \usepackage{hyperref}
     \hypersetup{linktoc=all}     
     
     \title{Elettrodinamica classica}
     \author{Appunti aggiornati al:}
	 \date{\today}
	 
	 \newcommand{\dif}{\mathrm{d}}
	 \renewcommand{\d}{\mathrm{d}}
	 \newcommand{\der}[3][]{\frac{\partial ^{#1}#2}{\partial {#3}^{#1}}}
	 \newcommand{\R}{\mathbb{R}}
	 \newcommand{\C}{\mathbb{C}}
	 \let\oldnabla\nabla
	 \renewcommand{\nabla}{\vec{\oldnabla}}
	 \newcommand{\lap}{\oldnabla^2}
	 \newcommand{\norm}[1]{\left\Vert#1\right\Vert}
	 \newcommand{\scal}[2]{\left\langle#1\left|\right.#2\right\rangle}
	 \renewcommand{\vec}[1]{\mathbf{#1}}
	 \newcommand{\fourier}[2]{\frac{1}{\sqrt{2\pi}}\int_{-\infty}^{+\infty}#1\,\dif#2}
	 \newcommand{\ext}{\textrm{ext}}
	 \newcommand{\m}{\textrm{m}}
	 \newcommand{\p}{\textrm{p}}
	 
	 \theoremstyle{theorem}
	 \newtheorem{teorema}{Teorema}[section]
	 \newtheorem{lemma}{Lemma}[section]
	 \newtheorem{corollario}{Corollario}[section]
	 \newtheorem{proposizione}{Proposizione}[section]
	 
	 \theoremstyle{definition}
	 \newtheorem{problema}{Problema}[section]
	 \newtheorem{soluzione}{Soluzione}[section]
	 \newtheorem{osservazione}{Osservazione}[section]
	 \newtheorem{definizione}{Definizione}[section]
	 
\begin{document}
	\maketitle
	\tableofcontents
\chapter{Elettromagnetismo nel vuoto}
\section{28 settembre 2017}
\subsection{Libri consigliati}
\begin{itemize}
	\item Dispense del corso
	\item Jackson
	\item Landau II
	\item Landau VIII
\end{itemize}
\subsection{Unità di misura}
Il corso utilizzerà il sistema CGS. In particolare, la carica elettrica si misura in ues (unità elettrostatica). La conversione con il sistema MKS si ottiene considerando che la carica del protone è
\[e=4,8\cdot10^{-10}\textrm{ ues}=1,6\cdot10^{-19}\textrm{ C}\]
Tale scelta è dovuta soprattutto a motivi di eleganza, nella legge di Coulomb, e al fatto che in questo sistema il campo elettrico e il campo magnetico hanno la stessa unità di misura.
\subsection{Legge di Coulomb}
In natura esistono corpi dotati di una certa proprietà, detta carica elettrica. Esistono due tipi di carica elettrica, che per convenzione si identificano come carica positiva e carica negativa. Si osserva sperimentalmente che due corpi carichi, posti a grande distanza rispetto alle proprie dimensioni (in modo che possano essere trattati a tutti gli effetti come puntiformi) interagiscono tramite una forza. In particolare, se si pone una carica $q_1$ nella posizione $\vec{r}_1$ e una carica $q_2$ nella posizione $\vec{r_2}$, la forza di cui risente la carica $q_2$ è
\begin{equation}
	\label{coulomb}
	\vec{F}_{2,1}=k\frac{q_1q_2}{r_{2,1}^2}\hat{r}_{2,1}
\end{equation}
dove si è posto $\vec{r}_{2,1}=\vec{r}_2-\vec{r}_1$ e $k$ è una costante positiva, che nel sistema CGS è 1. Dalla equazione (\ref{coulomb}) si possono trarre considerazioni interessanti:
\begin{itemize}
	\item $\vec{F}_{2,1}$ è diretta lungo la congiungente tra le due cariche, come si poteva intuire dalla simmetria del sistema
	\item il modulo della forza è inversamente proporzionale al quadrato della distanza tra le due cariche
	\item la forza è repulsiva se le cariche sono concordi, attrattiva altrimenti
\end{itemize}
\subsection{Principio di sovrapposizione}
Sperimentalmente, si osserva che in una configurazione di 3 cariche $q_1,q_2,q_3$, si ha
\[
\vec{F}_{3,12}=\vec{F}_{3,1}+\vec{F}_{3,2}
\]
In particolare, non compaiono altri termini di interazione, come ad esempio un termine $\vec{\mathcal F}(q_1q_2q_3)$ dipendente dal prodotto delle cariche. Da ciò deriva anche che se consideriamo una collezione di cariche $q_1,\dots,q_N$, posizionate rispettivamente in $\vec{r}_1,\dots,\vec{r}_N$, allora la forza di cui risente una carica $q$ posta in $\vec{r}$ (con $r\neq\vec{r}_j$ per ogni $j=1,\dots,N$) è
\begin{equation}
\label{campoelettrico}
\vec{F}(\vec{r})=q\sum_{j=1}^{N}\frac{q_j}{|\vec{r}-\vec{r}_j|^3}\left(\vec{r}-\vec{r}_j\right)\equiv q\vec{E}(\vec{r})
\end{equation}
Il vettore $\vec{E}(\vec{r})$, chiamato campo elettrico, definisce una proprietà del punto $\vec{r}$ che dipende solo dalla geometria del sistema di cariche $q_1,\dots,q_N$ ed è indipendente da $q$. Si deve ovviamente richiedere che $q$ sia sufficientemente piccola da non perturbare significativamente il sistema. Questa definizione, che per il momento è una semplice riscrittura in forma compatta della legge di Coulomb, sarà fondamentale nell'interpretazione dell'elettrodinamica come teoria di campo. In particolare, la legge di Coulomb predice un'azione a distanza tra due cariche, mentre la (\ref{campoelettrico}) può essere vista come un'interazione tra la carica $q$ e il campo elettrico, che a sua volta interagisce con la collezione di cariche $q_1,\dots,q_N$. Parliamo quindi di $\vec{E}$ come un mediatore dell'interazione tra cariche.

Possiamo ovviamente considerare distribuzioni di cariche continue, oltre che puntiformi, e quindi introdurre il concetto di densità di carica. Usualmente vengono utilizzate le lettere $\rho,\sigma,\lambda$ per la densità rispettivamente volumica, superficiale e lineare. In particolare, il campo elettrico generato da una densità volumica di carica è
\[\vec{E}(\vec{r})=\int\frac{\rho(\vec{r}')\left(\vec{r}-\vec{r}'\right)}{\left|\vec{r}-\vec{r}'\right|^3}\dif^3x'\]
L'integrale è chiaramente esteso a tutti i punti in cui $\rho$ è non nulla, eventualmente tutto lo spazio. Una distribuzione di carica si dice localizzata se è nulla al di fuori di una sfera sufficientemente grande, oppure se si annulla all'infinito in maniera sufficientemente veloce (ad esempio, esponenzialmente, o comunque più velocemente di una qualunque potenza).
\subsection{Principio di simmetria}
Se una distribuzione di carica gode di una certa simmetria (ad esempio sotto riflessione o rotazione), allora anche il campo elettrico da essa generato deve godere della stessa simmetria.
\subsection{Filo infinito uniformemente carico}
Consideriamo un filo con densità di carica uniforme $\lambda$, e supponiamo per semplicità che sia infinito. Fissiamo un sistema di riferimento cartesiano in cui l'asse $x$ coincide con il filo. Consideriamo un punto $P$ sull'asse $z$ a distanza $D$ dal filo. Se facciamo una riflessione rispetto al piano contenente gli assi $x$ e $z$, sia il filo che $P$ rimangono fissi. Ciò significa che il campo in $P$ è contenuto nel piano $xz$. Inoltre, se consideriamo una riflessione rispetto al piano ortogonale al piano $xz$ e contenente l'asse $z$, sia il filo che $P$ rimangono fissi. Allora il campo elettrico in $P$ deve essere della forma $\vec{E}=E_z(D)\hat{z}$. In un qualunque altro punto $Q$ del piano $xz$ valgono le medesime considerazioni, poichè il filo è invariante per traslazioni lungo l'asse $x$ (dato che è infinito). Inoltre, la simmetria cilindrica del sistema ci permette di ripetere questo ragionamento per un qualunque piano contenente l'asse $x$. Procediamo ora con il calcolo. Si ha
\[
	E_z(D)=\int_{-\infty}^{+\infty}\frac{\lambda}{D^2}\cos^3\theta\dif x=\frac{\lambda}{D}\int_{-\pi/2}^{\pi/2}\cos\theta\dif\theta=\frac{2\lambda}{D}
\]
dove si è posto $x=D\tan\theta$.
\subsection{Piano uniformemente carico}
Conseriamo un piano con densità di carica uniforme $\sigma$, e supponiamo a meno di rotazioni che sia il piano $xy$. Se $P$ è un punto dell'asse $z$ a distanza $D$ dal piano. Anche in questo caso, la distribuzione di carica è invariante per rotazioni intorno all'asse $z$, quindi $\vec{E}=E_z(D)\hat{z}$. Tale ragionamento si può adattare a un qualunque altro punto dello spazio, grazie all'invarianza della distribuzione di carica rispetto a traslazioni su rette parallele al piano $xy$. Allora si ha
\[
	E_z(D)=\int_{0}^{2\pi}\int_{0}^{\infty}\frac{\sigma}{D^2}\cos^3\theta r\dif r\dif\theta=2\pi\sigma
\]
Il risultato precedente non dipende da $D$. Infatti, fissato un cono con vertice in $P$, asse coincidente con l'asse $z$ e semiampiezza $\theta$, la carica intercettata dal cono è proporzionale a $qr^2\propto qD^2$, quindi il contributo al campo elettrico di tale porzione di piano è proporzionale a $\sigma$.
\subsection{Flusso del campo elettrico e legge di Gauss}
Siano $\vec{v}$ un campo vettoriale, $S$ una superficie. Definiamo il flusso di $\vec{v}$ attraverso $S$ come
\[\Phi_S(\vec{v})=\int_S\vec{v}\cdot\dif\vec{A}\]
Il flusso gode delle ovvie proprietà:
\begin{itemize}
	\item linearità rispetto ai vettori, ovvero \[\Phi_S(\alpha\vec{v}+\beta\vec{w})=\alpha\Phi_s(\vec{v})+\beta\Phi_S(\vec{w})\]
	\item additività sulle superfici, ovvero se $S=S_1\sqcup S_2$ allora
	\[
		\Phi_S(\vec{v})=\Phi_{S_1}(\vec{v})+\Phi_{S_2}(\vec{v})
	\]
\end{itemize}
Notiamo inoltre che se $S$ è aperta, il versore normale non ha un verso preferenziale. Se $S$ è chiusa, per convenzione si considera positivo il verso uscente.

Consideriamo ora una superficie chiusa e convessa $S$ e una carica $q$ al suo interno. Se $\dif\Omega$ è un angolo solido infinitesimo, il contributo al flusso del campo elettrico di $q$ della superficie infinitesima $\dif A$ intercettata da $\dif\Omega$ è
\[
	\dif\Phi_S(\vec{E})=E(R)\cos\theta\dif A=E(R)R^2\dif\Omega=q\dif\Omega
\]
Dunque integrando, e ricordando che l'angolo solido completo è $4\pi$, si ottiene
\[\Phi_S(\vec{E})=4\pi q\]
Se $S$ non è convessa, basta osservare che l'angolo solido intercetta la superficie un numero dispari di volte se $q$ è interna e quindi i contributi al flusso si elidono tutti tranne uno, mentre l'angolo solido intercetta la superficie un numero pari di volte se $q$ è esterna, e quindi il campo generato da $q$ non contribuisce al flusso. Possiamo quindi enunciare la legge di Gauss
\begin{equation}
\label{integralgauss}
	\oint_S\vec{E}\cdot\dif\vec{A}=4\pi Q_{\textrm{int}}
\end{equation}
dove $Q_{\textrm{int}}$ è la carica che si trova nel volume delimitato da $S$. Notiamo inoltre che è fondamentale la dipendenza di $\vec{E}$ dall'inverso del quadrato della distanza, affinchè il teorema sia valido.
\newpage
\section{5 ottobre 2017}
\subsection{Operatori differenziali}
Consideriamo una funzione scalare $u\colon\R^3\to\R$ e una funzione vettoriale $\vec{v}\colon\R^3\to\R^3$. Allora possiamo definire, in coordinate cartesiane, i seguenti operatori
\begin{itemize}
	\item Gradiente di uno scalare
	\[\nabla u=\left(\der{u}{x},\der{u}{y},\der{u}{z}\right)\]
	\item Divergenza di un vettore
	\[\nabla\cdot\vec{v}=\der{v_x}{x}+\der{v_y}{y}+\der{v_z}{z}\]
	\item Rotore di un vettore
	\[\nabla\times\vec{v}=\left(\der{v_z}{y}-\der{v_y}{z},\der{v_x}{z}-\der{v_z}{x},\der{v_y}{x}-\der{v_x}{y}\right)\]
	\item Laplaciano di uno scalare
	\[\lap u=\nabla\cdot\left(\nabla u\right)=\der[2]{u}{x}+\der[2]{u}{y}+\der[2]{u}{z}\]
	\item Laplaciano di un vettore
	\[\lap\vec{v}=\left(\lap v_x,\lap v_y,\lap v_z\right)\]
\end{itemize}
Valgono inoltre i seguenti teoremi
\begin{teorema}[della divergenza o di Ostrogradskij] Consideriamo una superficie chiusa $S$ e sia $V_S$ il volume da essa racchiuso. Allora si ha
	\[\oint_S\vec{v}\cdot\dif\vec{A}=\int_{V_S}\nabla\cdot\vec{v}\dif^3r\]
\end{teorema}
\begin{proof}
	Consideriamo una partizione di $V_S$ in $n$ cubetti. Se $V_i$ è il volume dell'$i$-esimo cubetto si ha ovviamente
	\[\int_{V_S}\nabla\cdot\vec{v}d^3r=\sum_{i=1}^{n}\int_{V_i}\nabla\cdot\vec{v}\dif^3r\]
	Detta $S_i$ la superficie dell'$i$-esimo cubetto, si ha anche
	\[\oint_S\vec{v}\cdot\dif\vec{A}=\sum_{i=1}^{n}\oint_{S_i}\vec{v}\cdot\dif\vec{A}\]
	Infatti, se un cubetto ha un faccia interna a $V_S$, allora esiste un secondo cubetto che ha tale faccia in comune con il primo. Allora i due contributi al flusso si elidono, perchè uguali e opposti. Se invece un cubetto ha una faccia appartenente alla superficie $S$, il suo contributo al flusso non si elide. Se ora consideriamo una suddivisione sufficientemente fitta e un cubetto con un vertice in $(x,y,z)$ e con i lati lunghi $\Delta x,\Delta y$ e $\Delta z$, si ha al primo ordine
	\[\oint_{S_i}\vec{v}\cdot\dif\vec{A}=\left(v_x(x+\Delta x,y,z)-v_x(x,y,z)\right)\Delta y\Delta z+\left(v_x(x,y+\Delta y,z)-v_x(x,y,z)\right)\Delta x\Delta z+\]\[+\left(v_x(x,y,z+\Delta z)-v_x(x,y,z)\right)\Delta x\Delta y=\left(\der{v_x}{x}+\der{v_y}{y}+\der{v_z}{z}\right)\Delta x\Delta y\Delta z=\]\[=\nabla\cdot\vec{v}\Delta x\Delta y\Delta z\]
	Sommando tutti i contributi si ottiene la tesi
\end{proof}
\begin{teorema}[del rotore o di Stokes]
	Consideriamo una curva chiusa $\gamma$ e una superficie qualunque $S_\gamma$ che ha per bordo $\gamma$, con il versore normale scelto con la regola della mano destra. Allora si ha
	\[\oint_\gamma\vec{v}\cdot\dif\vec{l}=\int_{S_\gamma}\nabla\times\vec{v}\cdot\dif\vec{A}\]
\end{teorema}
\begin{proof}
	Partizioniamo $S_\gamma$ in $n$ quadratini. Allora si ha ovviamente
	\[\int_{S_\gamma}\nabla\times\vec{v}\cdot\dif\vec{A}=\sum_{i=1}^{n}\int_{S_i}\nabla\times\vec{v}\cdot\dif\vec{A}\]
	Si ha anche
	\[\oint_\gamma\vec{v}\cdot\dif\vec{l}=\sum_{i=1}^{n}\oint_{\gamma_i}\vec{v}\cdot\dif\vec{l}\]
	Infatti, se un quadratino ha un lato interno a $S_\gamma$, allora esiste un altro quadratino con tale lato in comune. I versi di percorrenza sono opposti, quindi i contributi del lato alla circuitazione si elidono. Invece i contributi dovuti ai lati esterni non si elidono. Consideriamo ora una suddivisione in quadratini sufficientemente fitta e supponiamo ad esempio che $S_i$ giaccia su un piano ortogonale all'asse $z$, abbia un vertice in $(x,y,z)$ e lati lunghi $\Delta x$ e $\Delta y$ . Allora al primo ordine si ha
	\[\oint_{\gamma_i}\vec{v}\cdot\dif\vec{l}=\left(v_x(x,y,z)-v_x(x,y+\Delta y,z)\right)\Delta x+\left(v_y(x+\Delta x,y,z)-v_y(x,y,z)\right)\Delta y=\]\[=\left(\der{v_y}{x}-\der{v_x}{y}\right)\Delta x\Delta y=\left(\nabla\times\vec{v}\right)_z\Delta x\Delta y\]
	Più in generale, se $S_i$ è ortogonale a $\vec{n}$, si ha
	\[\oint_{\gamma_i}\vec{v}\cdot\dif\vec{l}=\nabla\times\vec{v}\cdot\vec{n}S_i\]
	E sommando su $i$ si ottiene la tesi. Tale risultato non dipende dalla particolare $S_\gamma$. Infatti, se $S'_\gamma$ è un'altra superficie che ha per bordo $\gamma$, $S_\gamma\cup S'_\gamma$ è chiusa, quindi per il teorema della divergenza
	\[\oint_{S\gamma\cup S'_\gamma}\nabla\times\vec{v}\cdot\dif\vec{A}=\int_V\nabla\cdot\left(\nabla\times\vec{v}\right)\dif^3r=0\]
	Dato che l'orientazione di $S'_\gamma$ va invertita per rispettare la regola della mano destra, si ha infine
	\[\int_{S_\gamma}\nabla\times\vec{v}\cdot\dif\vec{A}=\int_{S'_\gamma}\nabla\times\vec{v}\cdot\dif\vec{A}\]
\end{proof}
\subsection{Prima equazione di Maxwell}
Riprendiamo la legge di Gauss e scriviamo la carica come integrale di volume
\[
	\oint_S\vec{E}\cdot\dif\vec{A}=4\pi\int_V\rho(\vec{r})\dif^3r
\]
Usando il teorema della divergenza e l'arbitrarietà di $S$ e $V$ si ottiene la legge di Gauss in forma locale:
\[
	\nabla\cdot\vec{E}(\vec{r})=4\pi\rho(\vec{r})
\]
\subsection{Potenziale elettrostatico e seconda equazione di Maxwell}
Consideriamo una carica puntiforme $Q$ posta nell'origine. Sia $\vec{E}(\vec{r})$ il campo elettrostatico da essa generato. Allora, presi due punti $A$ e $B$ posti in $\vec{r}_A$ e $\vec{r}_B$ e una curva $\gamma$ che li congiunge, si ha
\[\int_{\gamma}\vec{E}\cdot\dif\vec{l}=\int_{\gamma}\frac{Q}{r^2}\hat{r}\cdot\dif\vec{l}=\int_{r_A}^{r_B}\frac{Q}{r^2}\dif r=Q\left(\frac{1}{r_A}-\frac{1}{r_B}\right)\]
Il risultato precedente non dipende da $\gamma$, dunque è possibile definire un potenziale $V$ tale che
\[-\int_{\vec{r}_A}^{\vec{r}_B}\vec{E}\cdot\dif\vec{l}=V(\vec{r}_B)-V(\vec{r}_A)\]
Equivalentemente, se non ci sono problemi di convergenza all'infinito possiamo definire il potenziale in un punto $\vec{r}$ come
\[
	V(\vec{r})=-\int_{\infty}^{\vec{r}}\vec{E}\cdot\dif\vec{l}
\]
dove si è implicitamente assunto che $\lim\limits_{|\vec{r}|\to\infty}V(\vec{r})=0$. In forma locale, si ha
\[\vec{E}=-\nabla V\]
Il potenziale di una carica puntiforme $Q$ nell'origine è
\[
V(\vec{r})=\frac{Q}{r}\]
Di conseguenza, per una collezione di cariche puntiformi
\[V(\vec{r})=\sum_{i=1}^{n}\frac{q_i}{|\vec{r}-\vec{r}_i|}\]
Per una distribuzione continua di carica
\[V(\vec{r})=\int\frac{\rho(\vec{r}')\dif^3r'}{|\vec{r}-\vec{r}'|}\]
Infine, la conservatività è equivalente a
\[\oint_\gamma\vec{E}\cdot\dif\vec{l}=0\]
per ogni curva chiusa $\gamma$. Ma questo è equivalente per il teorema di Stokes a
\[\nabla\times\vec{E}=0\]
che è la seconda equazione di Maxwell.
\subsection{Delta di Dirac}
Introduciamo la funzione impropria $\delta$ che gode delle seguenti proprietà:
\begin{itemize}
	\item $\delta(x)=0$ per ogni $x\neq0$
	\item $\int_a^b\delta(x)\dif x=1$ se l'intervallo $(a,b)$ contiene lo zero
	\item $\int_a^b\delta(x)f(x)\dif x=f(0)$ se l'intervallo $(a,b)$ contiene lo zero
	\item $\int_{0}^{b}\delta(x)f(x)\dif x=\frac{1}{2}f(0)$
\end{itemize}
La $\delta$ può essere vista come il limite di una successione di funzioni pari con integrale costante (ad esempio, step functions o gaussiane). Inoltre, se $\vec{r}=x\hat{x}+y\hat{y}+z\hat{z}$ definiamo la $\delta$ tridimensionale come
\[\delta(\vec{r})=\delta(x)\delta(y)\delta(z)\]
\newpage
\section{12 ottobre 2017}
\subsection{Energia elettrostatica}
Consideriamo una distribuzione di carica localizzata $\rho(\vec{r})$ e sia $V(\vec{r})$ il potenziale generato:
\[V(\vec{r})=\int\frac{\rho(\vec{r}')}{|\vec{r}-\vec{r}'|}\dif^3r'\]
Dove l'integrale è esteso a tutto il supporto di $\rho(\vec{r})$. Il lavoro compiuto per portare una carica puntiforme dall'infinito nel punto $\vec{R}$ è
\[L=-\int_{-\infty}^{\vec{R}}\vec{F}(\vec{r})\cdot\dif\vec{l}=qV(\vec{R})\]
Dunque l'energia potenziale della carica $q$ è
\[U(\vec{R})=qV(\vec{R})\]

Consideriamo ora una collezione di $N$ cariche puntiformi $q_1,\dots,q_N$ poste in $\vec{r}_1,\dots,\vec{r}_N$ e, per ogni $1\leq i,j\leq N$, sia $r_{ij}=|\vec{r}_i-\vec{r}_j|$. L'energia del sistema può essere calcolata nel seguente modo: la carica $q_1$ viene portata dall'infinito a $\vec{r}_1$ senza compiere lavoro. La seconda carica viene portata dall'infinito a $\vec{r}_2$ compiendo un lavoro
\[L_2=\frac{q_1q_2}{r_{12}}\]
Il lavoro compiuto per spostare $q_3$ dall'infinito a $\vec{r}_3$ è
\[L_3=\frac{q_1q_3}{r_{13}}+\frac{q_2q_3}{r_{23}}\]
Generalizzando, l'energia del sistema, che è uguale alla somma dei lavori precedenti, è
\begin{equation}\label{energiacariche}U=\sum_{i=1}\sum_{j<i}\frac{q_{i}q_j}{r_{ij}}=\frac{1}{2}\sum_{i\neq j}\frac{q_iq_j}{r_{ij}}\end{equation}
Per distribuzioni continue di carica si ha ovviamente
\begin{equation}
\label{energiadistribuzione1}U=\frac{1}{2}\iint\frac{\rho(\vec{r})\rho(\vec{r}')}{|\vec{r}-\vec{r}'|}\dif^3r\dif^3r'=\frac{1}{2}\int\rho(\vec{r})V(\vec{r})\dif^3r\end{equation}
Vogliamo ora riscrivere l'ultima relazione in funzione del solo campo elettrico $\vec{E}$. Notiamo preliminarmente che, prese $A$ funzione scalare $\vec{B}$ funzione vettoriale, si ha
\[
	\nabla\cdot(A\vec{B})=\der{(AB_i)}{x_i}=B_i\der{A}{x_i}+A\der{B_i}{x_i}=\vec{B}\cdot\nabla A+A\nabla\cdot\vec{B}
\]
Usando la prima equazione di Maxwell e il teorema della divergenza si ottiene
\[
	U=\frac{1}{8\pi}\oint_S V\vec{E}\cdot\dif\vec{A}+\frac{1}{8\pi}\int_{V_S}|\vec{E}|^2\dif^3r
\]
Per l'ipotesi sulla localizzazione, possiamo supporre che $S$ sia una sfera di raggio $R$ sufficientemente elevato. Allora su tale superficie si ha $V|\vec{E}|\propto R^{-3}$, quindi nel limite $R\to\infty$ il primo integrale si annulla. Pertanto abbiamo
\begin{equation}
	\label{energiadistribuzione}
	U=\frac{1}{8\pi}\int|\vec{E}|^2\dif^3r
\end{equation}
L'integrale è esteso a tutto lo spazio. La (\ref{energiadistribuzione}) permette anche di interpretare $U$ come l'energia necessaria a creare il campo elettrico stesso. Inoltre, non è del tutto equivalente alla (\ref{energiacariche}), come si deduce subito considerando una coppia di cariche di segno opposto. Infatti, la (\ref{energiadistribuzione1}), da cui deriva la (\ref{energiadistribuzione}), è stata ottenuta immaginando di portare dei volumetti dall'infinito contenenti una carica infinitesima, e non finita. In altre parole, la (\ref{energiadistribuzione}) tiene conto anche dell'energia spesa per creare effettivamente la carica, mentre nella (\ref{energiacariche}) si suppone che le cariche puntiformi siano preesistenti, e tale equazione esprime solamente l'energia di interazione tra coppie di cariche. Per verificare questa affermazione, consideriamo due cariche puntiformi (il caso generico segue immediatamente per il principio di sovrapposizione). Siano $q_1$ e $q_2$ tali cariche, poste rispettivamente in $\vec{r}_1$ e $\vec{r}_2$. Allora per il campo elettrico si ha, con ovvio significato dei simboli
\[
	\vec{E}=\vec{E}_1+\vec{E}_2
\]
Di conseguenza usando la (\ref{energiadistribuzione}) si ottiene
\[
	U=\frac{1}{8\pi}\int_{V_S}\left(|\vec{E}_1|^2+|\vec{E}_2|^2+2\vec{E}_1\cdot\vec{E}_2\right)\dif^3r
\]
Dove $V_S$ è un opportuno volume che specificheremo in seguito. Trascurando gli integrali in $|\vec{E}_1|^2$ e $|\vec{E}_2|^2$, che divergono, si ha
\[
	U=-\frac{1}{4\pi}\int_{V_S}\vec{E}_1\cdot\nabla V_2\dif^3r=\frac{1}{4\pi}\int_{V_S} V_2\nabla\cdot\vec{E}_1\dif^3x-\frac{1}{4\pi}\oint_SV_2\vec{E}_1\cdot\dif\vec{A}
\]
Scegliamo ora come $V_S$ una sfera di raggio $R$ che contiene le due cariche, privata di due sfere di raggio $R_1$ e $R_2$ centrate rispettivamente nelle due cariche. In tal modo, il primo integrale all'ultimo membro è nullo per la prima equazione di Maxwell. Per il secondo, tenuto conto dei versi dei vettori superficie, si ha con ovvio significato dei simboli
\[
	U=-\frac{1}{4\pi}\oint_{S_R}V_2\vec{E}_1\cdot\dif\vec{A}+\frac{1}{4\pi}\oint_{S_{R_1}}V_2\vec{E}_1\cdot\dif\vec{A}+\frac{1}{4\pi}\oint_{S_{R_2}}V_2\vec{E}_1\cdot\dif\vec{A}
\]
Il primo integrale si annulla quando $R\to\infty$. Su $S_{R_2}$ il potenziale $V_2$ è costante, dunque il terzo integrale è nullo per la legge di Gauss. Infine, se $R_1$ è sufficientemente piccolo $V_2$ è pressochè costante sulla sfera $S_{R_1}$, quindi si ha per la legge di Gauss
\[
	U=\frac{1}{4\pi}\oint_{S_{R_1}}V_2\vec{E}_1\cdot\dif\vec{A}=q_1V_2=\frac{q_1q_2}{r_{12}}
\]
Quindi la (\ref{energiadistribuzione}) e la (\ref{energiacariche}) danno lo stesso valore per l'energia di mutua interazione.
\subsection{Conduttori}
Un conduttore perfetto è un corpo tale che il campo elettrico al suo interno, in condizioni elettrostatiche, è sempre nullo. Ciò significa che un conduttore ideale può generare una distribuzione di carica arbitrariamente grande per annullare un eventuale campo esterno. Nella pratica, non esistono conduttori perfetti, ma alcuni corpi (specialmente metallici) sono buoni conduttori, ossia si comportano a grandi linee come conduttori ideali, una volta superato un transiente iniziale. La definizione di conduttore permette di dedurre alcune proprietà:
\begin{enumerate}
	\item ogni coppia di punti del conduttore si trova allo stesso potenziale
	\item un'eventuale carica netta del conduttore deve essere distribuita sulla superficie
	\item in prossimità della superficie, subito all'esterno del conduttore, la componenente normale del campo elettrico è $E_\perp=4\pi\sigma$, dove $\sigma$ è la distribuzione superficiale di carica
	\item il campo elettrico in prossimità del conduttore è ortogonale alla superficie, ossia $\vec{E}=4\pi\sigma\hat{n}$
	\item una carica $q$ posta in prossimità di un conduttore genera una certa distribuzione di carica, che annulla il campo di $q$ all'interno
	\item se il conduttore ha una cavità e in essa viene posta una carica $q$, dall'esterno non si hanno informazioni sulla posizione di $q$
\end{enumerate}
\subsection{Cariche immagine per il piano}
Consideriamo un conduttore che occupa il semispazio $z<0$ e mettiamo una carica puntiforme $q$ in $(0,0,d)$, con $d>0$. Possiamo supporre che il piano $z=0$ sia a potenziale nullo. Allora il campo delle cariche indotte può essere simulato, per $z>0$, al campo prodotto da una carica $-q$ in $(0,0,-d)$. Tale carica, detta carica immagine, è chiaramente fittizia, ma simula gli effetti reali della distribuzione di cariche indotte.
\subsection{Buona positura dei problemi di elettrostatica, funzioni armoniche}
Consideriamo un certo volume $V_S$ racchiuso dalla superficie $S$, in cui viene posta una densità di carica $\rho$. Vogliamo risolvere in $V$ l'equazione di Poisson
\[
	\lap V=-4\pi\rho
\]
Inoltre, vogliamo capire quali condizioni dare su $S$ e $V_S$ affinché tale soluzione sia unica. Non ci preoccupiamo invece dell'esistenza di $V$, dato che supporremo di trattare sempre casi fisicamente possibili. Dimostriamo un lemma preliminare:
\begin{lemma}[Prima identità di Green]
	Siano $S,V_S$ come sopra e $\phi,\psi$ due funzioni sufficientemente regolari. Allora si ha
	\[\int_{V_S}\left(\phi\lap\psi+\nabla\phi\cdot\nabla\psi\right)\dif^3r=\oint_S\phi\der{\psi}{n}\dif A\]
\end{lemma}
\begin{proof}
	Segue immediatamente dal teorema della divergenza.
\end{proof}
\begin{teorema}[Unicità della soluzione del problema di Poisson]
	Siano $S,V_S,\rho$ come sopra e supponiamo di avere la condizione al bordo $V_{|S}(\vec{r})=u(\vec{r})$ (condizione di Dirichlet) oppure $\der{V_{|S}}{n}(\vec{r})=v(\vec{r})$ (condizione di Neumann), dove $u,v$ sono funzioni assegnate sufficientemente regolari. Allora la soluzione del problema di Poisson è unica.
\end{teorema}
\begin{proof}
	Siano $V_1,V_2$ due soluzioni. Allora, posto $U=V_2-V_1$ e utilizzando la prima identità di Green con $\phi=\psi=U$, si ha
	\[\int_{V_S}\left(U\lap U+|\nabla U|^2\right)\dif^3r=\oint_SU\der{U}{n}\dif A\]
	Chiaramente $U$ è armonica in $V_S$, inoltre $U$ oppure $\der{U}{n}$ sono identicamente nulle su $S$, quindi si deduce $\nabla{U}=0$ in $V$, da cui $U$ costante. Allora $V_1$ e $V_2$ differiscono per una costante, e dunque sono lo stesso potenziale.
\end{proof}
\begin{osservazione}
	Si possono dare condizioni miste su $S$, ossia partizionare $S$ in $S_1,S_2$ e dare una condizione di Dirichlet su $S_1$ e una condizione di Neumann su $S_2$. In generale, se si assegnano contemporaneamente $V$ e $\der{V}{n}$ la soluzione non esiste.
\end{osservazione}
Valgono inoltre i seguenti teoremi per le funzioni armoniche:
\begin{teorema}[della media]
	Siano $V$ una funzione armonica su $V_S$, $x\in V_S$, $a>0$ tale che la sfera $S_a$ di raggio $a$ centrata in $x$ sia interamente contenuta in $V_S$. Allora si ha
	\[
		V(x)=\frac{1}{4\pi a^2}\oint_{S_a}V\dif A
	\]
\end{teorema} 
\begin{proof}
	A meno di traslazioni, supponiamo $x=0$. Allora si ha
	\[
		\langle V\rangle(a)=\frac{1}{4\pi a^2}\oint_{S_a}V\dif A=\int\frac{\dif\Omega}{4\pi}V(a,\theta,\phi)
	\]
	Poichè $V$ è almeno $C^1$, si può applicare il teorema di derivazione sotto il segno di integrale, ottenendo
	\[\der{\langle V\rangle}{a}=\int\frac{\dif\Omega}{4\pi}\der{V}{a}=\frac{1}{4\pi a^2}\oint_{S_a}\nabla V\cdot\dif\vec{A}=\frac{1}{4\pi a^2}\int\lap V\dif^3r=0\]
	Da cui si conclude.
\end{proof}
\begin{osservazione}
	Il teorema della media vale anche in dimensione più alta.
\end{osservazione}
\begin{teorema}[Principio del massimo per funzioni armoniche]
	Sia $V$ armonica su $V_S$ non costante. Allora $V$ ammette massimo e minimo su $S$.
\end{teorema}
\begin{proof}
	Supponiamo che $V$ abbia un massimo interno $x$ (il caso del minimo è analogo se si considera $-V$). Sia $a>0$ tale che la sfera $S_a$ centrata in $x$ di raggio $a$ sia interna a $V_S$. Allora si ha
	\[
		V(x)=\frac{1}{4\pi a^2}\oint_{S_a}V\dif A<V(x)
	\]
	Da cui l'assurdo.
\end{proof}
\begin{corollario}[Teorema di Earnshaw]
	Il campo elettrico generato da collezione di cariche puntiformi non ammette punti di equilibrio stabile.
\end{corollario}
\begin{proof}
	Infatti, i punti di equilibrio stabile sono dei minimi per il potenziale $V$, e nel caso di cariche puntiformi possono solo essere all'infinito.
\end{proof}
\newpage
\section{19 ottobre 2017}
\subsection{Funzioni di Green}
\begin{lemma}[Seconda identità di Green]
	Consideriamo una superficie chiusa $S$ e sia $V_S$ il volume da essa delimitato. Se $\phi$ e $\psi$ sono due funzioni sufficientemente regolari, allora si ha
	\[\int_{V_S}\left(\phi\lap\psi-\psi\lap\phi\right)\dif^3r=\oint_S\left(\phi\der{\psi}{n}-\psi\der{\phi}{n}\right)\dif A\]
\end{lemma}
\begin{proof}
	L'identità si ottiene banalmente dalla prima identità di Green, scambiando $\phi$ e $\psi$ e sommando membro a membro.
\end{proof}
Vogliamo ora usare la seconda identità di Green per risolvere l'equazione di Poisson, una volta assegnati il volume $V_S$, la superficie $S$, la distribuzione di carica $\rho$ e una condizione al bordo di Dirichlet o di Neumann. Scegliamo $\phi=V$ e $\psi=\frac{1}{|\vec{r}-\vec{r}'|}$. Allora si ottiene
\begin{equation}
\label{solpoisson}
	V(\vec{r})=\int_{V_S}\frac{\rho(\vec{r}')}{|\vec{r}-\vec{r}'|}\dif^3r'+\frac{1}{4\pi}\oint_S\left(\frac{1}{|\vec{r}-\vec{r}'|}\der{V}{n'}(\vec{r}')-V(\vec{r}')\der{}{n'}\frac{1}{|\vec{r}-\vec{r}'|}\right)\dif A'
\end{equation}
In generale, su $S$ non conosciamo contemporaneamente $V$ e $\partial V/\partial n$, e anche se li conoscessimo il problema sarebbe sovradeterminato e, in generale, senza soluzione. Osserviamo però che abbiamo scelto una tale $\psi$ per il semplice fatto che $\lap\psi=-4\pi\delta$. Allora, riscriviamo la (\ref{solpoisson}) sostituendo $\frac{1}{|\vec{r}-\vec{r}'|}$ con una funzione $G(\vec{r},\vec{r}')$, detta funzione di Green, tale che $\oldnabla'^2G(\vec{r},\vec{r}')=-4\pi\delta(\vec{r}-\vec{r}')$. Si ottiene
\[
	V(\vec{r})=\int_{V_S}\rho(\vec{r}')G(\vec{r},\vec{r}')\dif^3r'+\frac{1}{4\pi}\oint_S\left(G(\vec{r},\vec{r}')\der{V}{n'}(\vec{r}')-V(\vec{r}')\der{G}{n'}(\vec{r},\vec{r}')\right)\dif A'
\]
Distinguiamo ora due casi, in base alle condizioni al contorno:
\begin{enumerate}
	\item Se le condizioni al bordo sono di Dirichlet, ossia se è noto $V_{|S}$, possiamo richiedere $G_D(\vec{r},\vec{r}')=0$ per ogni $\vec{r}'$ su $S$. In tal modo si ottiene la soluzione
	\begin{equation}
	\label{greendirichlet}
	V(\vec{r})=\int_{V_S}\rho(\vec{r}')G(\vec{r},\vec{r}')\dif^3r'-\frac{1}{4\pi}\oint_SV(\vec{r}')\der{G}{n'}(\vec{r},\vec{r}')\dif A'
	\end{equation}
	La funzione di Green $G_D(\vec{r},\vec{r}')$ dipende unicamente da $S$ e non dalle condizioni al bordo o dalla distribuzione di carica in $V_S$, dunque è sufficiente trovare la funzione di Green per una data superficie per poter (almeno teoricamente) risolvere un qualunque problema di Poisson con condizioni di Dirichlet.
	\item Se le condizioni al bordo sono di Neumann, vorremmo richiedere $\partial G_N/\partial n'=0$ su $S$. Ciò non è possibile, infatti si ha
	\[
		-4\pi=\int_{V_S}\oldnabla'^2G_N(\vec{r},\vec{r}')\dif^3r'=\oint_S\der{G_N}{n'}(\vec{r},\vec{r}')\dif A
	\]
	Dove si è usato il teorema della divergenza. Cerchiamo allora una funzione di Green con derivata normale costante su $S$, ossia
	\[
		\der{G_{|S}}{n'}(\vec{r},\vec{r}')=-\frac{4\pi}{S}
	\]
	In tal caso, detta $\langle V\rangle$ la media di $V$ su $S$, si ha la soluzione
	\begin{equation}
	\label{grennneumann}
		V(\vec{r})=\langle V\rangle+\int_{V_S}\rho(\vec{r}')G(\vec{r},\vec{r}')\dif^3r'+\frac{1}{4\pi}\oint_SG_N(\vec{r},\vec{r}')\der{V}{n'}(\vec{r}')\dif A'
	\end{equation}
	Ossia $V$ è definito a meno di una costante, che per quanto ci riguarda è ininfluente. Anche in questo caso $G_N(\vec{r},\vec{r}')$ dipende unicamente dalla superficie.
\end{enumerate}
La funzione di Green $G_D(\vec{r},\vec{r}')$ è simmetrica nei suoi argomenti, ovvero $G_D(\vec{r},\vec{r}')=G_D(\vec{r}',\vec{r})$. Infatti, usando la seconda identità di Green con $\phi(\vec{r}')=G_D(\vec{r},\vec{r}')$ e $\psi(\vec{r}')=G_D(\vec{r}_0,\vec{r}')$ si ottiene
\[
	\int_{V_S}\left(G_D(\vec{r},\vec{r}')\oldnabla'^2G_D(\vec{r}_0,\vec{r}')-G_D(\vec{r}_0,\vec{r}')\oldnabla'^2G_D(\vec{r},\vec{r}')\right)\dif^3r'=0
\]
Da cui la simmetria.

Supponiamo ora di avere un certo volume $V_S$ racchiuso dalla superficie $S$. Supponiamo anche di conoscere il potenziale $V(\vec{r})$ generato da una carica puntiforme $q$ posta in $\vec{r}_0$, con la condizione al bordo $V_{|S}=0$. Allora dalla (\ref{greendirichlet}) si ottiene
\[
	G_D(\vec{r},\vec{r}_0)=\frac{V(\vec{r})}{q}
\]
Verifichiamo che effettivamente è soluzione. Si ha
\[
	\oldnabla'^2G_D(\vec{r}_0,\vec{r}')=\oldnabla'^2G_D(\vec{r}',\vec{r}_0)=-4\pi\delta(\vec{r}'-\vec{r}_0)
\]
Inoltre, $G_D(\vec{r}_0,\vec{r}')$ è chiaramente nulla se $\vec{r}'$ è sul bordo $S$.
\subsection{Funzione di Green per il piano}
Vogliamo utilizzare il metodo precedente per calcolare la funzione di Green con condizioni di Dirichlet per un piano conduttore. Supponiamo che tale piano sia $z=0$ e immaginiamo di porre una carica $q$ in $(x',y',z')$, con $z'>0$. Allora utilizzando le cariche immagine si trova la soluzione per il potenziale
\[V(x,y,z)=q\left(\frac{1}{\sqrt{(x-x')^2+(y-y')^2+(z-z')^2}}-\frac{1}{\sqrt{(x-x')^2+(y-y')^2+(z+z')^2}}\right)\]
Di conseguenza la funzione di Green è
\begin{equation}\label{greenpiano}
	G_D(x,y,z,x',y',z')=\frac{1}{\sqrt{(x-x')^2+(y-y')^2+(z-z')^2}}-\frac{1}{\sqrt{(x-x')^2+(y-y')^2+(z+z')^2}}
\end{equation}
Ora possiamo risolvere un qualunque problema nel semispazio $z>0$, una volta assegnato il valore del potenziale sul piano $z=0$ e all'infinito (a rigore, tutte le superfici vanno intese come chiuse, quindi per spazi illimitati si considera la superficie all'infinito). Ad esempio, supponiamo di non avere carica e che il potenziale sia $V_0$ su un quadrato di lato $a$ centrato in $(0,0,0)$ e nullo altrove. Allora il potenziale in tutto il semispazio è
\[
	V(x,y,z)=\frac{V_0z}{2\pi}\int_{-a/2}^{a/2}\dif x'\int_{-a/2}^{a/2}\dif y'\frac{1}{\left((x-x')^2+(y-y')^2+z^2\right)^{3/2}}
\]
Osserviamo che, posto $r=|\vec{r}|$ e detto $\theta$ l'angolo tra $\vec{r}$ e l'asse $z$, a grandi distanze dall'origine (con grandi si intende grandi rispetto ad $a$) il potenziale è
\[
	V(r,\theta)=\frac{V_0a^2\cos\theta}{2\pi r^2}
\]
Ossia al primo ordine coincide con il campo di un dipolo di momento $\vec{p}=(V_0a^2\hat{z})/(2\pi)$.
\newpage
\section{26 ottobre 2017}
\subsection{Importanza dell'equazione di Laplace}
Supponiamo di voler risolvere l'equazione di Poisson $\lap V=-4\pi\rho$ in un certo volume $V_S$ delimitato da una superficie $S$, con una certa condizione al bordo di Dirichlet $V_{|S}=\varphi$. Definiamo
\[\tilde{V}(\vec{r})=\int\frac{\rho(\vec{r}')}{|\vec{r}-\vec{r}'|}\dif^3r'\]
Ovviamente $\tilde{V}$ soddisfa l'equazione di Poisson, ma in generale non rispetta le condizioni al bordo. Possiamo allora risolvere l'equazione di Laplace $\lap\overline{V}=0$, con la condizione al bordo $\overline{V}_{|S}=\varphi-\tilde{V}_{|S}$. In tal modo, abbiamo la soluzione $\tilde{V}+\overline{V}$. Le prossime sezioni saranno dedicate a metodi per la soluzione dell'equazione di Laplace.
\subsection{Separazione delle variabili}
Cominciamo con un esempio. Vogliamo risolvere l'equazione di Laplace nella striscia $[0,a]\times[0,+\infty)$, con le condizioni al bordo $V(x=0,y)=V(x=a,y)=\lim\limits_{y\to+\infty}V(x,y)=0$ e $V(x,y=a)=\varphi(x)$, con $\varphi$ funzione assegnata. Immaginiamo di poter scrivere $V(x,y)=X(x)Y(y)$. Allora l'equazione di Laplace è semplicemente
\[Y\frac{\dif^2X}{\dif x^2}+X\frac{\dif^2Y}{\dif y^2}=0\]
Dividendo per $XY$ si ha
\[\frac{1}{X}\frac{\dif^2X}{\dif x^2}=-\frac{1}{Y}\frac{\dif^2Y}{\dif y^2}\]
I due membri dipendono rispettivamente da una sola delle due variabili, dunque devono essere entrambi uguali a una certa costante $-k^2$. Non è restrittivo supporre la costante negativa, come vedremo in seguito. Le soluzioni che rispettano le condizioni al bordo richiedono $k=n\pi/a$, per un certo $n\in\mathbb{N}^*$, e sono
\[X_n(x)=\sin \frac{n\pi}{a}x\]
\[Y_n(y)=e^{-\frac{n\pi}{a}y}\]
Una generica soluzione sarà allora della forma
\[V(x,y)=\sum_{n=1}^{\infty}c_n\sin \frac{n\pi}{a}xe^{-\frac{n\pi}{a}y}\]
Imponendo l'ultima condizione al bordo, si ottiene
\[\varphi(x)=\sum_{n=1}^{\infty}c_n\sin \frac{n\pi}{a}x\]
A questo punto i $c_n$ sono noti
\[c_n=\frac{2}{a}\int_{0}^{a}\sin\frac{n\pi}{a}t\varphi(t)\dif t\]
\newpage
\section{2 novembre 2017}
\subsection{Equazione di Laplace in coordinate sferiche}
Nella lezione precedente abbiamo visto la soluzione dell'equazione di Laplace in coordinate cartesiane su una striscia $[0,a]\times[0,+\infty)$, ossia
\[V(x,y)=\sum_{n=0}^{\infty}c_n\sin\frac{n\pi x}{a}e^{-\frac{n\pi y}{a}}\]
Adesso vogliamo risolvere l'equazione di Laplace in coordinate sferiche. In tali coordinate si ha
\[\lap V=\frac{1}{r}\der[2]{}{r}\left(rV\right)+\frac{1}{r^2\sin\theta}\der{}{\theta}\left(\sin\theta\der{V}{\theta}\right)+\frac{1}{r^2\sin^2\theta}\der[2]{V}{\varphi}\]
Posto allora
\[V(r,\theta,\varphi)=\frac{U(r)}{r}P(\theta)Q(\varphi)\]
L'equazione di Laplace si scrive come
\[\frac{r^2\sin^2\theta}{U}\frac{\dif^2U}{\dif r^2}+\frac{\sin\theta}{P}\frac{\dif}{\dif\theta}\left(\sin\theta\frac{\dif P}{\dif\theta}\right)+\frac{1}{Q}\frac{\dif^2Q}{\dif\varphi^2}=0\]
Notiamo che possiamo separare l'equazione in due parti, una dipendente solo da $r$ e $\theta$, e una dipendente solo da $\varphi$. Si ha di conseguenza
\[Q(\varphi)=e^{im\varphi}\]
\[\frac{r^2}{U}\frac{\dif^2U}{\dif r^2}+\frac{1}{P\sin\theta}\frac{\dif}{\dif\theta}\left(\sin\theta\frac{\dif P}{\dif \theta}\right)-\frac{m^2}{\sin^2\theta}=0\]
Abbiamo scelto opportunamente la costante in modo che $Q(\varphi)$ sia perioda con periodo minore o uguale a $2\pi$. Notiamo che per $m=0$ a rigore si ha $Q(\varphi)=\alpha\varphi+\beta$. Se siamo interessati a una soluzione in cui $\varphi\in[0,2\pi]$, allora necessariamente $\alpha=0$, ma ci sono casi in cui $\alpha$ potrebbe essere non nulla. Limitiamoci per il momento ai casi in cui il sistema gode di simmetria azimutale, ossia $V$ è indipendente da $\varphi$. Di conseguenza $m=0$, quindi separando le variabili si ha
\[\frac{r^2}{U}\frac{\dif^2U}{\dif r^2}=l(l+1)\]
\[\frac{1}{P\sin\theta}\frac{\dif}{\dif\theta}\left(\sin\theta\frac{\dif P}{\dif \theta}\right)=-l(l+1)\]
La prima equazione si risolve cercando soluzioni della forma $r^\gamma$. Si trova $\gamma=l+1$ oppure $\gamma=-l$. Per la seconda, dato che $\theta\in[0,\pi]$ possiamo porre $x=\cos\theta$. L'equazione si riscrive allora come
\[\frac{\dif}{\dif x}\left((1-x^2)\frac{\dif P}{\dif x}\right)+l(l+1)P=0\]
Cerchiamo per $P$ una soluzione in serie di potenze:
\[P(x)=\sum_{n=0}^{\infty}a_nx^n\]
Svolgendo i calcoli si trova la relazione
\[a_{n+2}=\frac{n(n+1)-l(l+1)}{(n+2)(n+1)}a_n\]
Si può mostrare che se gli $a_n$ non si annullano mai allora $P(1)$ diverge. Di conseguenza, richiediamo che $l$ sia un intero non negativo e che sia $a_0=0$ (se $l$ è dispari) oppure $a_1=0$ (se $l$ è pari). In tal modo abbiamo solo un numero finito di termini non nulli, dunque $P$ è un polinomio. Questi polinomi sono chiamati polinomi di Legendre e, generalmente, sono scelti in modo che $P_l(1)=1$ (tale scelta è possibile per la linearità del laplaciano). I primi polinomi sono
\[P_0(x)=1\]
\[P_1(x)=x\]
\[P_2(x)=\frac{3x^2-1}{2}\]
In generale, $P_l$ è un polinomio di grado $l$ e le potenze che compaiono in $P_l$ hanno tutte la stessa parità di $l$. Inoltre, si può mostrare che i polinomi di Legendre sono un insieme ortogonale e completo per $L^2([-1,1])$. In particolare, si ha
\[\int_{-1}^{1}P_n(x)P_m(x)\dif x=\delta_{mn}\frac{2}{2n+1}\]
Ritornando all'equazione di Laplace, se abbiamo simmetria azimutale si ha
\[V(r,\theta)=\sum_{l=0}^{\infty}\left(A_lr^l+\frac{B_l}{r^{l+1}}\right)P_l(\cos\theta)\]
\subsection{Sfera conduttrice in campo esterno}
Consideriamo una sfera conduttrice di raggio $a$ scarica, posta in un campo esterno uniforme $\vec{E}_0$. Possiamo supporre che la sfera sia centrata nell'origine e che il campo sia diretto lungo l'asse $z$. Inoltre, possiamo anche supporre che il potenziale sulla sfera sia nullo. Il sistema ha un'ovvia simmetria azimutale e le condizioni al contorno sono
\[V(a,\theta)=0\]
\[V(r\gg a,\theta)=-Er\cos\theta\]
Inoltre, il potenziale deve essere dispari rispetto a $z$, in particolare nullo sul piano $z=0$. In generale avremo
\[V(r,\theta)=\sum_{l=0}^{\infty}\left(A_lr^l+\frac{B_l}{r^{l+1}}\right)P_l(\cos\theta)\]
La condizione all'infinito impone $A_1=-E$, $A_l=0$ se $l\geq 2$. Inoltre, dato che sul piano $z=0$ il potenziale è nullo si avrà anche $A_0=0$. I coefficienti $B_l$ si determinano imponendo che il potenziale sulla sfera sia nullo. In particolare, si ha $B_1=E_0a^3$ e $B_l=0$ se $l\neq 1$. La soluzione per il potenziale sarà quindi
\[V(r,\theta)=-E_0r\cos\theta+\frac{E_0a^3\cos\theta}{r^2}\]
Notiamo inoltre che il secondo termine è il potenziale di un dipolo posto nell'origine, parallelo a $\vec{E}_0$ e di modulo $p=E_0a^3$. Tale soluzione può essere ottenuta anche con le cariche immagine o con MasterCard.
\subsection{Espansione fondamentale}
Fissiamo un punto $\vec{r}'$. Vogliamo espandere nella base dei polinomi di Legendre la funzione
\[\frac{1}{|\vec{r}-\vec{r}'|}\]
Supponiamo che i due vettori siano paralleli e $|\vec{r}|>|\vec{r}'|$. Allora si ha
\[\frac{1}{|\vec{r}-\vec{r}'|}=\frac{1}{r-r'}=\frac{1}{r}\frac{1}{1-r'/r}=\frac{1}{r}\sum_{k=0}^{\infty}\left(\frac{r'}{r}^k\right)\]
Nel caso generale, se $\gamma$ è l'angolo tra $\vec{r}$ e $\vec{r}'$ avremo
\[\frac{1}{|\vec{r}-\vec{r}'|}=\sum_{l=0}^{\infty}\left(A_lr^l+\frac{B_l}{r^{l+1}}\right)P_l(\cos\gamma)\]
Dato che la funzione tende a zero quando $r$ diventa grande, si ha $A_l=0$. Inoltre, dato che i $B_l$ non dipendono da $\gamma$ è sufficiente calcolarli per $\gamma=0$ (ossia $\vec{r}$ e $\vec{r}'$ sono paralleli). In tal caso $P_l(1)=1$ per ogni $l$, quindi si conclude
\[\frac{1}{|\vec{r}-\vec{r}'|}=\sum_{l=0}^{\infty}\frac{r'^l}{r^{l+1}}P_l(\cos\gamma)\]
Se invece $|\vec{r}|<|\vec{r}'|$, basta invertire i ruoli di $\vec{r}$ e $\vec{r}'$. In generale quindi, posto $r_<=\min\left\{|\vec{r}|,|\vec{r}'|\right\}$ e $r_>=\max\left\{|\vec{r}|,|\vec{r}'|\right\}$, si ha
\[\frac{1}{|\vec{r}-\vec{r}'|}=\sum_{l=0}^{\infty}\frac{r_<^l}{r_>^{l+1}}P_l(\cos\gamma)\]
Ci sono (almeno) due casi patologici. Se $\vec{r}=\vec{r}'$ allora la serie diverge (come fa d'altronde la funzione). In tal caso si può mostrare che, negli integrali, la serie si comporta come una $\delta$. Se invece $\vec{r}$ e $\vec{r}'$ hanno stessa direzione e modulo, ma verso opposto, si ha
\[\frac{1}{2r}=\frac{1}{r}\sum_{l=0}^{\infty}(-1)^l\]
Il che è assai brutto, ma secondo La Rocca "basta guardare le somme parziali e torna tutto".
\newpage
\section{9 novembre 2017}
\subsection{Armoniche sferiche}
Riprendiamo la parte angolare dell'equazione di Laplace in assenza di simmetria azimutale. In generale, $P$ è soluzione dell'equazione differenziale
\[\frac{\dif}{\dif x}\left[(1-x^2)\frac{\dif P}{\dif x}(x)\right]+l(l+1)P(x)-\frac{m^2}{1-x^2}P(x)=0\]
Fissati $l$ e $m$, sia $P_{l,m}(x)$ la soluzione di tale equazione e poniamo
\[Y_{l,m}(\theta,\varphi)=P_{l,m}(\cos\theta)e^{im\varphi}\]
Si può mostrare che per avere regolarità su tutto l'intervallo considerato si deve richiedere $-l\leq m\leq l$. Inoltre, $P_{l,m}(\cos\theta)$ è un polinomio in $\cos\theta$ e $\sin\theta$ di grado $l$. Le armoniche sferiche godono di tante proprietà:
\begin{itemize}
	\item Sono un sistema ortogonale completo per $L^2([0,\pi]\times[0,2\pi])$ e sono opportunamente normalizzate sull'angolo solido:
	\[\int Y^*_{l,m}(\theta,\varphi)Y_{l',m'}(\theta,\varphi)\dif\Omega=\int_{0}^{2\pi}\dif\varphi\int_{0}^{\pi}Y^*_{l,m}(\theta,\varphi)Y_{l',m'}(\theta,\varphi)\sin\theta\dif\theta=\delta_{ll'}\delta_{mm'}\]
	\item Sono le rappresentazioni irriducibili di $SO(3)$.
	\item Fissata la parità di $l$, $Y_{l,m}$ ha la stessa parità sotto inversione degli assi, ossia
	\[Y_{l,m}(\pi-\theta,\varphi+\pi)=(-1)^lY_{l,m}(\theta,\varphi)\] 
	\item Si ha
	\[Y_{l,-m}(\theta,\varphi)=(-1)^mY^*_{l,m}(\theta,\varphi)\]
\end{itemize}
Le prime armoniche sferiche sono
\[\begin{array}{l l l}
Y_{0,0}(\theta,\varphi)&=&\sqrt{\frac{1}{4\pi}}\\
Y_{1,0}(\theta,\varphi)&=&\sqrt{\frac{3}{4\pi}}\cos\theta\\
Y_{1,1}(\theta,\varphi)&=&-\sqrt{\frac{3}{8\pi}}\sin\theta e^{i\varphi}\\
Y_{2,0}(\theta,\varphi)&=&\sqrt{\frac{5}{16\pi}}(3\cos^2\theta-1)\\
Y_{2,1}(\theta,\varphi)&=&-\sqrt{\frac{15}{8\pi}}\sin\theta\cos\theta e^{i\varphi}\\
Y_{2,2}(\theta,\varphi)&=&\sqrt{\frac{15}{32\pi}}\sin^2\theta e^{2i\varphi}
\end{array}\]
I segni meno sparsi qua e là sono dovuti per lo più a motivi storici, non hanno una qualche interpretazione fisica.

Tornando all'equazione di Laplace, in generale avremo la soluzione
\[V(r,\theta,\varphi)=\sum_{l=0}^{\infty}\sum_{m=-l}^{l}\left(A_{l,m}r^l+\frac{B_{l,m}}{r^{l+1}}\right)Y_{l,m}(\theta,\varphi)\]
\subsection{Sviluppo in multipoli}
Vogliamo ora utilizzare le armoniche sferiche per trovare il potenziale elettrostatico di una certa distribuzione di carica localizzata $\rho$, quando siamo a grandi distanze rispetto alle dimensioni caratteristiche della distribuzione. Dato che la distribuzione è localizzata, possiamo assumere che il potenziale sia nullo all'infinito. Allora avremo
\[V(r,\theta,\varphi)=\sum_{l=0}^{\infty}\sum_{m=-l}^{l}\frac{B_{l,m}}{r^{l+1}}Y_{l,m}(\theta,\varphi)\]
Sappiamo anche che il potenziale è
\[V(r,\theta,\varphi)=\int\frac{\rho(\vec{r}')}{|\vec{r}-\vec{r}'|}\dif^3r'\]
Possiamo espandere l'espressione precedente in polinomi di Legendre, ricordando che siamo interessati a grandi distanze (e in particolare $r$ è sempre maggiore di $r'$):
\[V(r,\theta,\varphi)=\sum_{l=0}^{\infty}\frac{1}{r^{l+1}}\int\rho(\vec{r}')r'^lP_l(\cos\gamma)\dif^3r'\]
Dove $\gamma$ è l'angolo tra $\vec{r}$ e $\vec{r}'$. Se i due versori associati $\hat{r}$ e $\hat{r}'$ hanno coordinate angolari $(\theta,\varphi)$ e $(\theta',\varphi')$, si trova facilmente
\[\cos\gamma=\hat{r}\cdot\hat{r}'=\cos\theta\cos\theta'+\sin\theta\sin\theta'\cos(\varphi-\varphi')\]
A questo punto ci piacerebbe tantissimo fattorizzare $P_l(\cos\gamma)$ nel prodotto (o eventualmente nella somma di prodotti) di due termini, uno dipendente solo da $\theta$ e $\varphi$, l'altro solo da $\theta'$ e $\varphi'$. Possiamo utilizzare il teorema di addizione, secondo cui
\[P_l(\cos\gamma)=\frac{4\pi}{2l+1}\sum_{m=-l}^{l}Y_{l,m}(\theta,\varphi)Y^*_{l,m}(\theta',\varphi')\]
Di conseguenza il potenziale può essere scritto nella forma
\[V(r,\theta,\varphi)=\sum_{l=0}^{\infty}\sum_{m=-l}^{l}\frac{4\pi}{2l+1}\frac{Y_{l,m}(\theta,\varphi)}{r^{l+1}}\int\rho(\vec{r}')r'^lY^*_{l,m}(\theta',\varphi')\dif^3r'\]
I coefficienti $B_{l,m}$ si ricavano allora da un semplice confronto con l'espressione precedente. Fissato $l$, l'insieme dei coefficienti $B_{l,m}$ è il termine di $2^l$-polo. Ad esempio, per $l=0,1,2$ si hanno rispettivamente i termini di monopolo, dipolo e quadrupolo, che si possono ottenere anche espandendo l'espressione di $V(r,\theta,\varphi)$ in serie di Taylor in un intorno di $\vec{r}'=0$:
\[\frac{Q}{r}\]
\[\frac{\vec{p}\cdot\hat{r}}{r^2}\]
\[\frac{Q_{ij}x_ix_j}{r^5}\]
Il tensore di quadrupolo $Q_{ij}$ è simmetrico e a traccia nulla. Infatti, se non fosse a traccia nulla l'espansione del potenziale avrebbe un termine proporzionale a $r^{-3}$, mentre nell'espansione in armoniche sferiche tale termine ha sempre una dipendenza angolare non banale. In tal modo, gli elementi indipendenti del tensore di quadrupolo sono 5. Con l'espansione in armoniche sferiche, gli elementi indipendenti sono $B_{2,0}$, $B_{2,1}$ e $B_{2,2}$. Il primo è reale, gli altri due complessi, quindi anche in questo caso abbiamo cinque gradi di libertà.

Vediamo ora un esempio: consideriamo un quadrato la cui diagonale misura $2a$ e poniamo sui vertici due cariche puntiformi $q$ e due cariche puntiformi $-q$, in modo che le cariche con stesso segno siano su vertici opposti. Il sistema è globalmente neutro, quindi il termine di monopolo è nullo. Inoltre, è invariante sotto riflessioni intorno ai piani passanti per due cariche di stesso segni ortogonali alle rette per le rimanenti cariche, dunque tutti i termini con $l$ dispari sono nulli. A questo punto, fissiamo un sistema di coordinate in cui le cariche positive giacciono sull'asse $x$ e le cariche negative sull'asse $y$, con il centro del quadrato nell'origine. Prendiamo l'asse $z$ come asse polare e consideriamo gli eventuali termini di quadrupolo $B_{2,0}$, $B_{2,1}$ e $B_{2,2}$. Dato che $Y_{2,0}(\theta,\varphi)$ non dipende da $\varphi$, mentre il sistema è dispari sotto rotazioni di $\pi/2$ intorno all'asse $z$, ne deduciamo $B_{2,0}=0$. Analogamente, $Y_{2,1}(\theta,\varphi)\propto e^{i\varphi}$, quindi $B_{2,1}=0$ per mantenere la simmetria. $Y_{2,2}(\theta,\varphi)$ rispetta effettivamente le simmetrie del problema, dunque calcoliamolo. La densità di carica in tale sistema di riferimento è
\[\rho(\vec{r})=q\left(\delta(\vec{r}-a\hat{x})+\delta(\vec{r}+a\hat{x})-\delta(\vec{r}-a\hat{y})-\delta(\vec{r}+a\hat{y})\right)\]
Di conseguenza si ha
\[B_{2,2}=\frac{4\pi}{5}\int\rho(\vec{r})r^2Y^*_{2,2}(\theta,\varphi)\dif^3r=\frac{16}{5}\sqrt{\frac{15}{32\pi}}qa^2\]
Si ha chiaramente $B_{2,-2}=B_{2,2}$, quindi il potenziale a grandi distanze è
\[V(r,\theta,\varphi)=\frac{1}{r^3}B_{2,2}\left(Y_{2,2}(\theta,\varphi)+Y_{2,-2}(\theta,\varphi)\right)=\frac{3qa^2}{r^3}\sin^2\theta\cos2\varphi\]
Facciamo i controlli finali: il risultato è dimensionalmente corretto, le costanti di proporzionalità sono dell'ordine dell'unità, il potenziale è identicamente nullo sull'asse $z$, rispetta le simmetrie del sistema ed è massimo sul piano $xy$, come deve essere.
\newpage
\section{16 novembre 2017}
\subsection{Corrente elettrica e modello di Drude}
Definiamo la corrente attraverso una certa superficie (o un certo volume) come la quantità di carica che attraversa tale superficie nell'unità di tempo
\[I=\frac{\Delta Q}{\Delta t}\]
In alcuni materiali, detti materiali ohmici, la corrente è direttamente proporzionale alla differenziale di potenziale. Pertanto, in tali materiali si avrà
\[V=RI\]
La costante $R$ è detta resistenza e, se il conduttore ha lunghezza $l$ e sezione $A$, è pari a
\[R=\rho\frac{l}{A}\]
Dove $\rho$ è una costante tipica del materiale, detta resistività. L'inverso della resistività, comunemente indicata con $\sigma$, è la conduttività.

Se consideriamo ora un cilindro percorso da corrente, ci aspettiamo che $I$ sia distribuita uniformemente nel conduttore. Per un corpo di forma arbitraria in genere ciò non accade, quindi è utile introdurre la densità di corrente $\vec{J}$. Questo campo vettoriale è definito in modo tale che il suo flusso attraverso una certa superficie $S$ è uguale alla corrente che scorre attraverso $S$, ossia
\[I=\int\vec{J}\cdot\dif\vec{A}\]
Tornando all'esempio del cilindro, è facile mostrare che
\[\vec{J}=\sigma\vec{E}\]
Questa relazione è vera in generale, ed è chiamata legge di Ohm locale. 

Da un punto di vista microscopico, possiamo supporre che all'interno di un conduttore siano presenti $n$ cariche per unità di volume, ciascuna con carica $q$ e velocità media $\vec{v}$. In tal caso la densità di corrente è semplicemente
\[\vec{J}=nq\vec{v}\]
Ciò significa che la velocità è direttamente proporzionale al campo elettrico
\[nq\vec{v}=\sigma\vec{E}\]
Se invece risolviamo l'equazione del moto $m\dot{\vec{v}}=q\vec{E}$, ci aspettiamo $\vec{v}\propto\vec{E}t$. Ciò significa che devono essere presenti dei termini dissipativi, pertanto l'equazione del moto è
\[m\dot{\vec{v}}=q\vec{E}-\gamma\vec{v}\]
Per una certa costante $\gamma>0$. Allora i portatori di carica raggiungono una velocità limite
\[\vec{v}=\frac{q\vec{E}}{\gamma}\]
Possiamo quindi ricavare la conducibilità
\[\sigma=\frac{nq^2}{\gamma}\]
In particolare, $\sigma$ non dipende dal segno di $q$. La potenza dissipata per unità di volume è allora
\[P=n\gamma|\vec{v}|^2=\sigma |\vec{E}|^2=|\vec{J}||\vec{E}|\]
In accordo con l'espressione classica per la potenza dissipata per effetto Joule
\[P=\frac{VI}{Al}=|\vec{J}||\vec{E}|\]

Il principio di sovrapposizione lineare può essere esteso ai circuiti, a patto che tutti gli elementi circuitali siano lineari.
\subsection{Equazione di continuità}
Dal principio di conservazione della carica, sappiamo che la corrente attraverso una superficie chiusa è, a meno del segno, la derivata della carica contenuta all'interno della superficie, ossia
\[I=-\dot{Q}\]
Allora, utilizzando il teorema della divergenza, si ottiene facilmente l'equazione di continuità
\[\der{\rho}{t}+\nabla\cdot\vec{J}=0\]
\subsection{Magnetostatica}
Restringiamoci ora al caso di correnti stazionarie. Dall'equazione di continuità segue che la densità di corrente è indivergente. Sperimentalmente, si osserva che la forza agente su una carica $q$ in moto a velocità $\vec{v}$ in assenza di campo elettrico è della forma
\[\vec{F}=q\frac{\vec{v}}{c}\times\vec{B}\]
Dove $c$ è una costante con le dimensioni di una velocità. La relazione precedente può essere presa come definizione operativa del campo magnetico $\vec{B}$. Notiamo che, dato che $\vec{F}$ e $\vec{v}$ sono vettori, $\vec{B}$ deve essere uno pseudovettore. Inoltre, il campo magnetico e il campo elettrico hanno la stessa unità di misura. Se invece consideriamo un tratto $\Delta\vec{l}$ di filo percorso da una corrente $I$, la forza è
\[\Delta\vec{F}=\frac{I}{c}\Delta\vec{l}\times\vec{B}\]
La relazione precedente è equivalente all'espressione della forza di Lorentz, ma è più scomoda da un punto di vista operativo (i fili si devono per forza chiudere su un generatore).

Vogliamo ora studiare le sorgenti di campo magnetico. Oersted ha osservato che due fili paralleli percorsi dalle correnti $I_1$ e $I_2$ e posti a distanza $d$ interagiscono con una forza per unità di lunghezza $F$ pari a
\[F=\frac{2I_1I_2}{c^2d}\]
Inoltre, tale forza è attrattiva se le correnti sono concordi, repulsiva altrimenti. Sia $\vec{B}_1$ il campo generato dal filo percorso dalla corrente $I_1$. Dato che $\vec{B}$ è uno pseudovettore, le linee di campo possono solo essere circonferenze centrate nel filo. Allora la forza agente su un tratto $\Delta l$ del secondo filo è
\[\Delta F=I_2\Delta l B_1\]
Il risultato sperimentale ci dà allora
\[B_1=\frac{2I_1}{cd}\]
Notiamo che il campo generato da tale filo è indivergente, inoltre la sua circuitazione è proporzionale alla corrente concatenata con la curva lungo cui si integra:
\[\oint\vec{B}_1\cdot\dif\vec{l}=\frac{4\pi}{c}I\]
Dove il verso positivo di percorrenza è coerente con la regola della mano destra. Mostreremo che queste proprietà valgono in generale.
\subsection{Legge di Biot-Savart}
Biot e Savart hanno determinato sperimentalmente il campo magnetico generato da una certa distribuzione di corrente
\[\vec{B}(\vec{r})=\frac{1}{c}\int\frac{\vec{J}(\vec{r}')\times(\vec{r}-\vec{r}')}{|\vec{r}-\vec{r}'|^3}\dif^3r'\]
L'espressione può essere applicata al caso del filo e otteniamo lo stesso risultato precedente. Se invece consideriamo una spira di raggio $a$ e calcoliamo il campo magnetico in un punto sull'asse della spira, a distanza $z$ dal suo centro, si trova
\[\vec{B}(z)=\frac{2\pi Ia^2}{c(a^2+z^2)^{3/2}}\hat{z}\]
\newpage
\section{23 novembre 2017}
\subsection{Equazioni di Maxwell per la magnetostatica}
Vogliamo mostrare che la legge di Biot-Savart è equivalente al sistema di equazioni
\[\nabla\times\vec{B}=\frac{4\pi}{c}\vec{J}\]
\[\nabla\cdot\vec{B}=0\]
Assumiamo che Biot-Savart valga. Allora si ha
\[\vec{B}(\vec{r})=\frac{1}{c}\int\vec{J}(\vec{r}')\times\frac{\vec{r}-\vec{r}'}{|\vec{r}-\vec{r}'|^3}\dif^3r'=-\frac{1}{c}\int\vec{J}(\vec{r}')\times\nabla\left(\frac{1}{|\vec{r}-\vec{r}'|}\right)\dif^3r'\]
Notiamo ora che, presi $\psi$ campo scalare e $\vec{v}$ campo vettoriale, si ha
\[\nabla\times(\psi\vec{v})=\nabla\psi\times\vec{v}+\psi\nabla\times\vec{v}\]
Di conseguenza, notando che la densità di corrente dipende solo dalle coordinate primate, si ottiene
\[\vec{B}(\vec{r})=\frac{1}{c}\int\nabla\times\frac{\vec{J}(\vec{r}')}{|\vec{r}-\vec{r}'|}\dif^3r'=\nabla\times\vec{A}(\vec{r})\]
Dove si è introdotto il potenziale vettore $\vec{A}(\vec{r})$
\[\vec{A}(\vec{r})=\frac{1}{c}\int\frac{\vec{J}(\vec{r}')}{|\vec{r}-\vec{r}'|}\dif^3r'\]
In particolare, segue che $\vec{B}$ è indivergente. Mostriamo che anche il potenziale vettore è indivergente, nell'ipotesi che la densità di corrente sia localizzata. Per prima cosa, notiamo che se $\psi$ e $\vec{v}$ sono come sopra, si ha
\[\nabla\cdot(\psi\vec{v})=\nabla\psi\cdot\vec{v}+\psi\nabla\cdot\vec{v}\]
Di conseguenza, otteniamo
\[\nabla\cdot\vec{A}=\frac{1}{c}\int\nabla\cdot\frac{\vec{J}(\vec{r}')}{|\vec{r}-\vec{r}'|}\dif^3r'=\frac{1}{c}\int\vec{J}(\vec{r}')\cdot\nabla\left(\frac{1}{|\vec{r}-\vec{r}'|}\right)\dif^3r'=-\frac{1}{c}\int\vec{J}(\vec{r}')\cdot\nabla'\left(\frac{1}{|\vec{r}-\vec{r}'|}\right)\dif^3r'\]
Dove, al solito, abbiamo usato il fatto che la densità di corrente non dipende dalle coordinate non primate. Riapplicando la stessa relazione, otteniamo
\[\nabla\cdot\vec{A}=\frac{1}{c}\int\left[\frac{\nabla'\cdot\vec{J}(\vec{r}')}{|\vec{r}-\vec{r}'|}-\nabla'\cdot\left(\frac{\vec{J}(\vec{r}')}{|\vec{r}-\vec{r}'|}\right)\right]\dif^3r'\]
Il primo termine è nullo perchè siamo in condizioni stazionarie. Il secondo termine si tratta con il teorema della divergenza
\[\nabla\cdot\vec{A}=-\frac{1}{c}\int\frac{\vec{J}(\vec{r}')}{|\vec{r}-\vec{r}'|}\cdot\dif\vec{S}\]
In particolare, tende a zero quando la superficie diventa sufficientemente grande, grazie all'ipotesi di localizzazione. 

Ritorniamo per un'ultima volta a $\vec{v}$. Abbiamo
\[\nabla\times(\nabla\times\vec{v})=\hat{x}_i\varepsilon_{ijk}\der{}{x_j}\varepsilon_{klm}\der{v_m}{x_l}=\hat{x}_i\varepsilon_{kij}\varepsilon_{klm}\frac{\partial^2v_m}{\partial x_j\partial x_l}=\hat{x}_i(\delta_{il}\delta_{jm}-\delta_{im}\delta_{jl})\frac{\partial^2v_m}{\partial x_j\partial x_l}=\]\[=\hat{x}_i\frac{\partial^2 v_j}{\partial x_i\partial x_j}-\hat{x}_i\frac{\partial v_i}{\partial x_j\partial x_j}=\nabla(\nabla\cdot\vec{v})-\lap\vec{v}\]
Di conseguenza, abbiamo
\[\nabla\times\vec{B}=\nabla\times(\nabla\times\vec{A})=-\lap\vec{A}=-\frac{1}{c}\int\vec{J}(\vec{r}')\lap\frac{1}{|\vec{r}-\vec{r}'|}\dif^3r'=\frac{4\pi}{c}\int\vec{J}(\vec{r}')\delta(\vec{r}-\vec{r}')\dif^3r'=\frac{4\pi}{c}\vec{J}\]
Quindi la legge di Biot-Savart implica le equazioni di Maxwell per la magnetostatica. Per l'implicazione inversa, sappiamo che l'indivergenza di $\vec{B}$ ci assicura l'esistenza di un potenziale vettore $\vec{A}$ tale che
\[\vec{B}=\nabla\times\vec{A}\]
Chiaramente $\vec{A}$ non è unico. Possiamo, ad esempio, sommare al potenziale vettore il gradiente di una funzione scalare, senza modificare il campo magnetico. Scegliamo quindi il gauge di Coulomb, ossia scegliamo un certo potenziale vettore indivergente. In tal modo, otteniamo
\[\lap\vec{A}=-\frac{4\pi}{c}\vec{J}\]
Chiaramente, la soluzione per l'equazione precedente è
\[\vec{A}(\vec{r})=\frac{1}{c}\int\frac{\vec{J}(\vec{r}')}{|\vec{r}-\vec{r}'|}\dif^3r'\]
Che sappiamo essere indivergente. Di conseguenza ritroviamo il potenziale vettore che avevamo ottenuto partendo dalla legge di Biot-Savart. Ma allora anche $\vec{B}$ è lo stesso campo precedente, quindi abbiamo la seconda implicazione. Notiamo per inciso che neanche sotto gauge di Coulomb il potenziale vettore è univocamente determinato: possiamo, ad esempio, sommare il gradiente di una funzione armonica.
\subsection{Dipoli magnetici}
Consideriamo una distribuzione di densità di corrente localizzata. Vogliamo trovare il potenziale vettore a grande distanza dalla distribuzione. Abbiamo al primo ordine
\[\vec{A}(\vec{r})\approx\frac{1}{cr}\int\vec{J}(\vec{r}')\dif^3r'+\frac{1}{cr^3}\int\vec{J}(\vec{r}')\vec{r}\cdot\vec{r}'\dif^3r'\]
Mostriamo che il primo termine è nullo. Prendiamo, ad esempio, la componente lungo $\hat{x}$. Si ha
\[\int \hat{x}\cdot\vec{J}(\vec{r}')\dif^3r'=\int\nabla' x'\cdot\vec{J}(\vec{r}')\dif^3r'=\int\left(\nabla'\cdot(x'\vec{J}(\vec{r}'))-x'\nabla'\cdot\vec{J}(\vec{r}')\right)\dif^3r'\]
Il secondo termine è nullo perchè siamo in condizioni stazionarie. Il primo, in maniera analoga a quanto visto prima, è nullo a causa della localizzazione.
Il secondo termine dello sviluppo del potenziale vettore è invece
\[\frac{1}{cr^3}\int\vec{J}(\vec{r}')(xx'+yy'+zz')\dif^3r'\]
Limitiamoci ora alla componente lungo $\hat{x}$ di tale termine. Il primo integrale si annulla. Infatti, si ha
\[\int x'\hat{x}\cdot\vec{J}(\vec{r}')\dif^3r'=\frac{1}{2}\int\nabla' x'^2\cdot\vec{J}(\vec{r}')\dif^3r'=\frac{1}{2}\int\left(\nabla'\cdot(x'^2\vec{J}(\vec{r}'))-x'^2\nabla'\cdot\vec{J}(\vec{r}')\right)\dif^3r'\]
E si conclude come nel caso del termine di monopolo. Per il secondo termine, abbiamo invece
\[\int y'\hat{x}\cdot\vec{J}(\vec{r}')\dif^3r'=\int\left(\nabla'\cdot(x'y'\vec{J}(\vec{r}'))-x'\hat{y}\cdot\vec{J}(\vec{r}')-x'y'\nabla'\cdot\vec{J}(\vec{r}')\right)\dif^3r'=-\int x'\hat{y}\cdot\vec{J}(\vec{r}')\dif^3r'\]
L'integrale su $z'$ si fa in maniera analoga. Allora la componente lungo $\hat{x}$ del potenziale vettore è
\[A_x=\frac{1}{2cr^3}\int \left[y\left(y'J_x(\vec{r}')-x'J_y(\vec{r}')\right)+z\left(z'J_x(\vec{r}')-x'J_z(\vec{r}')\right)\right]\dif^3r'=\frac{(\vec{m}\times\vec{r})\cdot\hat{x}}{r^3}\]
Dove abbiamo posto
\[\vec{m}=\frac{1}{2c}\int\vec{r}'\times\vec{J}(\vec{r}')\dif^3r'\]
Chiaramente avremo
\[\vec{A}=\frac{\vec{m}\times\vec{r}}{r^3}\]
$\vec{m}$ è detto momento di dipolo magnetico ed è il termine dominante nello sviluppo in multipoli. Inoltre, ha le stesse dimensioni di un dipolo elettrico. Come esempio, per una spira piana di area $S$ in cui circola la corrente $I$, il momento di dipolo è
\[\vec{m}=\frac{IS}{c}\hat{n}\]
Dove $\hat{n}$ è il versore ortogonale alla spira, con verso preso coerentemente con la regola della mano destra.
\newpage
\section{30 novembre 2017}
\subsection{Forza e momento torcente su una distribuzione di corrente}
Sappiamo che la forza agente su una distribuzione di corrente immersa in un campo magnetico $\vec{B}$ è
\[\vec{F}=\frac{1}{c}\int\vec{J}(\vec{r}')\times\vec{B}(\vec{r}')\dif^3r'\]
Supponiamo ora che $\vec{B}$ sia uniforme. Allora si ha
\[\vec{F}=-\frac{1}{c}\vec{B}\times\int\vec{J}(\vec{r}')\dif^3r'=0\]
Ovvero la forza agente su una distribuzione di corrente può dipendere solo dalle derivate del campo magnetico. Questo fatto è una semplice conseguenza dell'assenza di monopoli magnetici.

Il momento torcente è 
\[\vec{\tau}=\frac{1}{c}\int\vec{r}'\times(\vec{J}(\vec{r}')\times\vec{B}(\vec{r}'))\dif^3r'\]
Se supponiamo di nuovo che $\vec{B}$ sia uniforme, allora si ha
\[\vec{\tau}=\frac{1}{c}\int\vec{J}(\vec{r}')\vec{r}'\cdot\vec{B}-\frac{\vec{B}}{c}\int\vec{J}(\vec{r}')\cdot\vec{r}'\dif^3r'\]
Sotto opportune ipotesi di localizzazione, il secondo termine è nullo. Per il primo invece, con calcoli del tutto analoghi a quelli per lo sviluppo del potenziale vettore, si ottiene
\[\vec{\tau}=\vec{m}\times\vec{B}\]
In alcuni testi, la relazione precedente è data come definizione operativa di campo magnetico.
\subsection{Rapporto giromagnetico}
Consideriamo una collezione di $N$ cariche puntiformi in moto. A tale moto possiamo associare la densità di corrente
\[\vec{J}=\sum_{i=1}^{N}q_i\vec{v}_i\delta(\vec{r}-\vec{r}_i)\]
Dove $q_i$, $\vec{v}_i$ e $\vec{r}_i$ sono rispettivamente carica, velocità e posizione della particella $i$-esima. Ovviamente questa densità di corrente non è stazionaria, ma a patto di prendere un numero sufficientemente alto di particelle può essere considerata ragionevolmente continua. Il momento magnetico associato a $\vec{J}$ è
\[\vec{m}=\frac{1}{2c}\int\vec{r}\times\vec{J}\dif^3r=\frac{1}{2c}\sum_{i=1}^{N}q_i\int\vec{r}\times\vec{v}_i\delta(\vec{r}-\vec{r}_i)\dif^3r=\frac{1}{2c}\sum_{i=1}^{N}q_i\vec{r}_i\times\vec{v}_i \]
Indicando con $\mu_i$ e $\vec{L}_i$ la massa e il momento angolare della particella $i$-esima, si ottiene
\[\vec{m}=\frac{1}{2c}\sum_{i=1}^{N}\frac{q_i}{\mu_i}\vec{L}_i\]
Il termine $q_i/(2c\mu_i)$ è detto rapporto giromagnetico. Infine, se tutte le particelle hanno stesso rapporto carica-massa, si ha
\[\vec{m}=\frac{q}{2c\mu}\vec{L}\]
\newpage
\section{11 gennaio 2018}
\subsection{Legge di Faraday}
Storicamente, Faraday ha osservato che un magnete in movimento rispetto a una spira chiusa genera una corrente elettrica all'interno di quest'ultima (e, ovviamente, la corrente era generata anche in una spira in movimento nei pressi di un magnete). Inoltre, anche una corrente variabile in una spira poteva generare corrente in un'altra spira. Se una spira ha resistenza $R$ ed è percorsa da una corrente indotta $I$, definiamo la forza elettromotrice come
\[\mathcal{V}=RI\]
Supponendo che tale "forza" sia dovuta a interazioni elettromagnetiche, allora si deve avere
\[RI=\oint_{\gamma}\left(\vec{E}+\frac{\vec{v}}{c}\times\vec{B}\right)\cdot\dif\vec{l}\]
dove l'integrale è esteso a tutto il circuito. Faraday propose inoltre la relazione
\[\mathcal{V}=-\frac{1}{c}\frac{\dif\Phi_B}{\dif t}=-\frac{1}{c}\frac{\dif }{\dif t}\int_{S_{\gamma(t)}}\vec{B}(t)\cdot\dif\vec{A}\]
dove $S_{\gamma(t)}$ è una superficie che ha per bordo il circuito stesso. Tale superficie può dipendere dal tempo, nel senso che non assumiamo che il circuito rimanga fermo (né che sia rigido). Abbiamo indicato esplicitamente la dipendenza temporale di $\vec{B}$, ma in generale esso dipende anche dalla posizione. Infine, il segno negativo può essere interpretato come segue: una corrente indotta genera un campo magnetico indotto la cui variazione di flusso si oppone alla variazione di flusso di campo magnetico che ha indotto la corrente stessa. Studiamo ora alcuni casi:
\begin{enumerate}
	\item Consideriamo $\gamma$ variabile e $\vec{B}$ uniforme e costante, ad esempio la classica rotaia che si muove a velocità $\vec{v}_0$. Se la lunghezza della rotaia è $w$, allora a meno di segni
	\[RI=\frac{Bwv_0}{c}\]
	Una carica $q$ nel tratto mobile, ferma rispetto a quest'ultimo, risente di una forza
	\[F=\frac{qv_0B}{c}\]
	La cui circuitazione lungo il circuito è proprio la forza elettromotrice. In realtà, a rigore dovremmo integrare
	\[\mathcal{V}=\oint_\gamma\frac{\vec{v}}{c}\times\vec{B}\cdot\dif\vec{l}\]
	con $\vec{v}=\vec{v}_0+\vec{v}_1$, dove $\vec{v}_1$ è la velocità della carica rispetto al filo. D'altro canto, $\vec{v}_1$ è parallelo a $\dif\vec{l}$, quindi non dà contributo alla forza elettromotrice. Infine, la forza agente sulla sbarra è
	\[\vec{F}=-\frac{IwB}{c}\hat{v}_0\]
	La potenza dissipata è allora, come ci aspettiamo
	\[P=\frac{IwBv_0}{c}=RI^2\]
	In tutta la discussione precedente, non abbiamo avuto necessità di considerare un campo elettrico non nullo: di fatto, non ci sono differenze sostanziali dal caso statico. Inoltre, in tale caso si parla di flusso tagliato: il circuito si muove "tagliando" le linee di flusso del campo magnetico.
	\item Consideriamo ora $\gamma$ fissa e $\vec{B}$ non costante. In tal caso, possiamo scegliere $S_{\gamma}$ indipendente dal tempo, dunque
	\[\mathcal{V}=-\frac{1}{c}\frac{\dif }{\dif t}\int_{S_\gamma}\vec{B}\cdot\dif\vec{A}=-\frac{1}{c}\int_{S_\gamma}\der{\vec{B}}{t}\cdot\dif\vec{A}\]
	Deduciamo, usando il teorema di Stokes
	\[-\frac{1}{c}\int_{S_\gamma}\der{\vec{B}}{t}(t)\cdot\dif\vec{A}=\oint_{\gamma}\left(\vec{E}+\frac{\vec{v}}{c}\times\vec{B}\right)\cdot\dif\vec{l}=\oint\vec{E}\cdot\dif\vec{l}=\int_{S_\gamma}\nabla\times\vec{E}\cdot\dif\vec{A}\]
	\[\nabla\times\vec{E}=-\frac{1}{c}\der{\vec{B}}{t}\]
	Infatti, se il circuito è fisso $\vec{v}$ è parallelo a $\dif\vec{l}$. Abbiamo quindi dei risultati nuovi, dato che nel caso statico il campo elettrico è sempre irrotazionale.
	\item Consideriamo il caso generale in cui sia $\gamma$ che $\vec{B}$ sono variabili nel tempo. Consideriamo il rapporto incrementale
	\[\frac{\Phi_B(t+\Delta t)-\Phi_B(t)}{\Delta t}=\frac{1}{\Delta t}\left(\int_{S_{\gamma(t+\Delta t)}}\vec{B}(t+\Delta t)\cdot\dif\vec{A}-\int_{S_{\gamma(t)}}\vec{B}(t)\cdot\dif\vec{A}\right)\simeq\]
	\[\simeq\frac{1}{\Delta t}\left[\int_{S_{\gamma(t+\Delta t)}}\left(\vec{B}(t)+\Delta t\der{\vec{B}}{t}(t)\right)\cdot\dif\vec{A}-\int_{S_{\gamma(t)}}\vec{B}(t)\cdot\dif\vec{A}\right]\simeq\]\[\simeq\int_{S_{\gamma(t+\Delta t)}}\der{\vec{B}}{t}(t)\cdot\dif\vec{A}+\frac{1}{\Delta t}\left(\int_{S_{\gamma(t+\Delta t)}}\vec{B}(t)\cdot\dif\vec{A}-\int_{S_{\gamma(t)}}\vec{B}(t)\cdot\dif\vec{A}\right)\]
	A meno di introdurre termini non del primo ordine, il primo integrale può essere fatto su $S_{\gamma(t)}$ (e quindi possiamo utilizzare il caso precedente con circuito fisso). Per il secondo, se $\sigma(t)$ è la superficie "laterale" del "cilindro" che ha per basi le superfici individuate dal circuito nei due istanti, abbiamo (attenzione all'orientazione delle superfici)
	\[\int_{S_{\gamma(t+\Delta t)}}\vec{B}(t)\cdot\dif\vec{A}-\int_{S_{\gamma(t)}}\vec{B}(t)\cdot\dif\vec{A}=-\int_{\sigma(t)}\vec{B}(t)\cdot\dif\vec{A}\]
	Abbiamo ovviamente $\dif\vec{A}=\dif\vec{l}\cdot\vec{v}\Delta t$, dove $\dif\vec{l}$ è lungo $\gamma(t)$. Allora effettivamente
	\[-\frac{1}{c}\frac{\dif\Phi_B}{\dif t}=\oint_{\gamma}\dif\vec{l}\cdot\left(\vec{E}+\frac{\vec{v}}{c}\times\vec{B}\right)\]
\end{enumerate}
\newpage
\section{18 gennaio 2018}
\subsection{Equazioni di Maxwell complete}
La legge di Ampère nel caso statico è, come sappiamo
\[\nabla\times\vec{B}=\frac{4\pi}{c}\vec{J}\]
L'equazione va sicuramente rivista nel caso dinamico: infatti, essa implica che $\vec{J}$ è indivergente, ma per l'equazione di continuità (che è basata su un principio fisico a cui non vorremo rinunciare) sappiamo che nel caso generale $\nabla\cdot\vec{J}\neq0$. Cerchiamo quindi di modificare la legge di Ampère come segue
\[\nabla\times\vec{B}=\frac{4\pi}{c}\vec{J}+\vec{\Lambda}\]
Prendendo la divergenza ambo i membri e usando l'equazione di continuità e la prima equazione di Maxwell otteniamo
\[\nabla\cdot\vec{\Lambda}=\frac{1}{c}\nabla\cdot\der{\vec{E}}{t}\]
Maxwell propose, con un lieve abuso
\[\vec{\Lambda}=\frac{1}{c}\der{\vec{E}}{t}\]
Otteniamo quindi le quattro equazioni di Maxwell
\begin{subequations}
	\begin{align}
		\nabla\cdot\vec{E}&=4\pi\rho\nonumber\\
		\nabla\times\vec{E}&=-\frac{1}{c}\der{\vec{B}}{t}\nonumber\\
		\nabla\cdot\vec{B}&=0\nonumber\\
		\nabla\times\vec{B}&=\frac{4\pi}{c}\vec{J}+\frac{1}{c}\der{\vec{E}}{t}\nonumber	
	\end{align}
\end{subequations}
Che, insieme alla legge di forza
\[\vec{F}=q\left(\vec{E}+\frac{\vec{v}}{c}\times\vec{B}\right)\]
descrivono interamente l'elettrodinamica. Osserviamo che tali equazioni sono invarianti sotto inversioni temporali (in particolare, $\vec{B}$ cambia di segno sotto tale trasformazione) e inversioni spaziali.

Per motivi storici, il termine aggiuntivo nella quarta equazione di Maxwell si chiama corrente di spostamento. Come noto, tale termine è necessario per mantenere la consistenza della legge di Ampère nella carica di un condensatore.
\subsection{Equazione delle onde per una corda vibrante}
Consideriamo una corda con estremità fissate di massa per unità di lunghezza $\mu$ e tensione $T$. Consideriamo le piccole oscillazioni trasversali della corda, in particolare indichiamo con $u(x,t)$ la distanza del punto di ascissa $x$ al tempo $t$ dalla posizione di equilibrio. Se consideriamo il tratto di corda compreso tra $x$ e $x+\Delta x$, la seconda legge di Newton si scrive come
\[\mu\Delta x\der[2]{u}{t}=T\left[\sin(\theta(x+\Delta x))-\sin(\theta(x))\right]\]
dove $\theta(x)$ è la pendenza della tangente alla corda in $x$. Se siamo nel regime di piccole oscillazioni, possiamo confondere seno e tangente, da cui
\[\mu\Delta x\der[2]{u}{t}=T\left[\tan(\theta(x+\Delta x))-\tan(\theta(x))\right]=T\left(\left.\der{u}{x}\right|_{x+\Delta x}-\left.\der{u}{x}\right|_{x}\right)\]
Di conseguenza, posto $v^2=T/\mu$
\[\der[2]{u}{x}-\frac{1}{v}\der[2]{u}{t}=0\]
\'E ben noto che le soluzioni di tale equazione sono tutte e sole le funzioni della forma
\[u(x,t)=f(x+vt)+g(x-vt)\]
Le funzioni $f$ e $g$ rappresentano rispettivamente onde che si propagano nel verso negativo e nel verso positivo dell'asse $x$ a velocità $v$. Si può anche risolvere l'equazione d'onda in funzione delle condizioni iniziali $u_0(x)=u(x,0)$ e $\dot{u}_0(x)=\partial u/\partial t(x,0)$. Infatti, con semplici calcoli si ottiene
\[u(x,t)=\frac{u_0(x+vt)+u_0(x-vt)}{2}+\frac{1}{2v}\int_{x-vt}^{x+vt}\dot{u}_0(z)\dif z\]
L'ovvia generalizzazione dell'equazione d'onda a tre dimensioni spaziali è
\[\lap u-\frac{1}{v^2}\der[2]{u}{t}=0\]
\subsection{Equazione delle onde per il campo elettromagnetico}
Supponiamo di essere in una regione di spazio priva di cariche e correnti. Prendiamo il rotore della quarta equazione di Maxwell
\[\nabla\times(\nabla\times\vec{B})=\frac{1}{c}\der{}{t}\nabla\times\vec{E}=-\frac{1}{c^2}\der[2]{\vec{B}}{t}\]
Ossia
\[\lap\vec{B}-\frac{1}{c^2}\der[2]{\vec{B}}{t}=0\]
Il campo magnetico si propaga dunque sotto forma di onda, con velocità $c$. Analogamente, prendendo il rotore della seconda equazione di Maxwell si ottiene
\[\lap\vec{E}-\frac{1}{c^2}\der[2]{\vec{E}}{t}=0\]
\newpage
\section{25 gennaio 2018}
\subsection{Onde piane}
Consideriamo la funzione
\[h(\vec{r},t)=h_0e^{i(\vec{k}\cdot\vec{r}-\omega t)}\]
Cerchiamo delle condizioni affinchè $h$ soddisfi l'equazione delle onde. Notiamo che le derivate agiscono su $h$ "per moltiplicazione", vale a dire
\[\der{h}{x_j}=ik_jh\]
\[\der{h}{t}=-i\omega h\]
Di conseguenza, si ha
\[\lap h-\frac{1}{v^2}\der[2]{h}{t}=\left(\frac{\omega^2}{v^2}-k^2\right)h\]
Dunque se $h$ è soluzione, allora si ha
\[\vec{k}=\frac{\omega}{c}\hat{k}\]
Il vettore $\vec{k}$ prende il nome di vettore d'onda, e chiaramente indica la direzione di propagazione dell'onda. Il suo modulo, detto numero d'onda, è legato alla lunghezza d'onda $\lambda$ da
\[k=\frac{2\pi}{\lambda}\]
Analogamente, la frequenza angolare è legata al periodo $T$ e alla frequenza dell'onda da
\[\omega=\frac{2\pi}{T}=2\pi f\]
Infine, $v$ è detta velocità di fase dell'onda.

Cerchiamo ora delle onde piane che soddisfino le equazioni di Maxwell, vale a dire cerchiamo dei campi elettrici e magnetici della forma
\[\vec{E}(\vec{r},t)=\vec{E}_0e^{i(\vec{k}\cdot\vec{r}-\omega t)}\]
\[\vec{B}(\vec{r},t)=\vec{B}_0e^{i(\vec{k}\cdot\vec{r}-\omega t)}\]
Non è restrittivo supporre che i due campi abbiano lo stesso numero d'onda (e di conseguenza la stessa pulsazione): tale fatto è implicato, ad esempio, dalla legge di Faraday.
Osserviamo inoltre che, a rigore, i campi cercati sono reali: utilizziamo la notazione complessa per alleggerire i calcoli, sottintendendo sempre che stiamo prendendo la parte reale delle espressioni scritte. Sicuramente i campi cercati soddisfano l'equazione delle onde. Imponiamo ora ulteriori condizioni affinchè soddisfino le equazioni di Maxwell. Partiamo dalle equazioni di Maxwell sulla divergenza dei campi
\[\nabla\cdot\vec{E}=i\vec{k}\cdot\vec{E}_0e^{i(\vec{k}\cdot\vec{r}-\omega t)}=0\]
\[\nabla\cdot\vec{B}=i\vec{k}\cdot\vec{B}_0e^{i(\vec{k}\cdot\vec{r}-\omega t)}=0\]
Ne deduciamo che $\vec{E}_0$ e $\vec{B}_0$ sono ortogonali a $\vec{k}$, ossia alla direzione di propagazione. Ciò significa che un'onda elettromagnetica piana è trasversale. Passiamo adesso alle equazioni sui rotori. In particolare, si ha
\[\nabla\times\vec{E}=i\vec{k}\times\vec{E}_0e^{i(\vec{k}\cdot\vec{r}-\omega t)}\]
\[\nabla\times\vec{B}=i\vec{k}\times\vec{B}_0e^{i(\vec{k}\cdot\vec{r}-\omega t)}\]
Quindi otteniamo
\[i\vec{k}\times\vec{E}_0e^{i(\vec{k}\cdot\vec{r}-\omega t)}=i\frac{\omega}{c}\vec{B}_0e^{i(\vec{k}\cdot\vec{r}-\omega t)}\]
\[=i\vec{k}\times\vec{B}_0e^{i(\vec{k}\cdot\vec{r}-\omega t)}=-i\frac{\omega}{c}\vec{E}_0e^{i(\vec{k}\cdot\vec{r}-\omega t)}\]
Ossia
\[\hat{k}\times\vec{E}_0=\vec{B}_0\]
\[\vec{B}_0\times\hat{k}=\vec{E}_0\]
Tali relazioni ci dicono che $\hat{k}$, $\vec{E}_0$ e $\vec{B}_0$ formano una terna ortogonale destrorsa e che $|\vec{E}_0|=|\vec{B}_0|$.
Notiamo infine che onde di tale forma si propagano nel vuoto e in assenza di cariche, inoltre dipendono da cinque parametri indipendenti: tre parametri fissano $\vec{k}$, due parametri fissano $\vec{E}_0$.

Supponiamo ora per semplicità $\vec{k}=k\hat{z}$. Se $\vec{E}_0=(E_x,E_y,0)$,\footnote{Ovviamente supponiamo che $E_x$ e $E_y$ siano reali.} allora il campo elettrico (dopo aver preso la parte reale) è
\[\vec{E}(r,t)=(E_x\hat{x}+E_y\hat{y})\cos(kz-\omega t)\]
Ciò significa che le componenti del campo elettrico oscillano armonicamente sempre nella stessa direzione. Diciamo che un'onda del genere è polarizzata linearmente (e, più in generale, indichiamo la direzione in cui oscilla il campo elettrico come polarizzazione). In realtà, nulla vieta di considerare valori complessi per $\vec{E}_0$. Ad esempio, se $\vec{E}_0=(E_x,E_y,0)e^{i\varphi}$, il campo corrispondente sarà sfasato di $\varphi$, ossia
\[\vec{E}(\vec{r},t)=(E_x\hat{x}+E_y\hat{y})\cos(kz-\omega t+\varphi)\]
In tal caso la polarizzazione è ancora lineare. Possiamo considerare casi più esotici, come $\vec{E}_0=(E_x,iE_y,0)$. Il campo elettrico è
\[\vec{E}(r,t)=E_x\hat{x}\cos(kz-\omega t)-E_y\hat{y}\sin(kz-\omega t)\]
Ovvero $\vec{E}$ si muove lungo un'ellisse. Una polarizzazione di questo tipo viene detta circolare ed è possibile distinguere tra polarizzione circolare destra e sinistra, a seconda del verso in cui ruota il campo.

Concludiamo con un'onda stazionaria. In particolare, consideriamo un'onda piana progressiva
\[\vec{E}_+(\vec{r},t)=(E,0,0)e^{i(kz-\omega t)}\]
\[\vec{B}_+(\vec{r},t)=(0,E,0)e^{i(kz-\omega t)}\]
E un'onda piana regressiva\footnote{Per convenzione, in un onda piana la pulsazione $\omega$ ha sempre segno negativo, dunque un'onda regressiva si ottiene cambiando di segno al vettore d'onda. Fisicamente è sensato, data l'interpretazione di $\hat{k}$, anche se alcuni (specialmente gli ingegneri) indicano un'onda regressiva cambiando di segno a $\omega$.}
\[\vec{E}_-(\vec{r},t)=(E,0,0)e^{i(-kz-\omega t)}\]
\[\vec{B}_-(\vec{r},t)=(0,-E,0)e^{i(-kz-\omega t)}\]
I campi dati dalla sovrapposizione di tali onde sono
\[\vec{E}(\vec{r},t)=(2E,0,0)\cos kz\cos\omega t\]
\[\vec{E}(\vec{r},t)=(0,2E,0)\sin kz\sin\omega t\]
In particolare, i nodi e i ventri di queste onde rimangono fissi nel tempo (dunque è davvero un'onda stazionaria) e in particolare a un massimo o minimo di $\vec{E}$ corrisponde un nodo di $\vec{B}$ e viceversa.
\subsection{Onde sferiche}
Cerchiamo una soluzione dell'equazione delle onde a simmetria sferica, vale a dire
\[h(\vec{r},t)=h(r,t)\]
Scrivendo il laplaciano in coordinate sferiche otteniamo
\[\frac{1}{r}\der[2]{}{r}\left(rh\right)-\frac{1}{v^2}\der[2]{h}{t}=0\]
Cerchiamo ad esempio un'onda della forma
\[h(r,t)=\frac{h_0}{r}e^{i(kr-\omega t)}\]
Inserendo tale funzione nell'equazione d'onda, troviamo
\[k=\frac{\omega}{c}\]
\subsection{Energia dei campi e vettore di Poynting}
Vogliamo trovare una qualche espressione del principio di conservazione dell'energia per i campi elettromagnetici. Consideriamo una regione di spazio $V$, delimitata dall'area chiusa $A$. Supponiamo che l'energia associata ai campi in tale regione sia $U_{\mathrm{em}}$. I campi possono sicuramente scambiare energia con cariche e correnti. Inoltre, ci può essere un flusso di energia attraverso $A$. Scriviamo quindi
\[\frac{\mathrm{d}U_{\mathrm{em}}}{\mathrm{d}t}=-\oint_{A}\vec{S}\cdot\dif\vec{A}-W\]
Dove l'integrale modellizza il trasporto di energia attraverso il bordo e $W$ l'energia scambiata con cariche e sorgenti. Se $u_{\mathrm{em}}$ e $w$ sono rispettivamente la densità di $U_{\mathrm{em}}$ e $W$, allora localmente abbiamo
\[\der{u_{\mathrm{em}}}{t}+\nabla\cdot\vec{S}+w=0\]
Con la convenzione che abbiamo scelto sui segni, sappiamo scrivere
\[w=\rho\vec{v}\cdot\left(\vec{E}+\frac{\vec{v}}{c}\times\vec{B}\right)=\vec{J}\cdot\vec{E}\]
Partiamo da questa relazione cercando di ricostruire $u_{\mathrm{em}}$ e $\vec{S}$.
Dalla quarta equazione di Maxwell sappiamo che
\[w=\left(\frac{c}{4\pi}\nabla\times\vec{B}-\frac{1}{4\pi}\der{\vec{E}}{t}\right)\cdot\vec{E}=\frac{c}{4\pi}\vec{E}\cdot(\nabla\times\vec{B})-\frac{1}{8\pi}\der{E^2}{t}\]
D'altro canto, osserviamo che si ha
\[\nabla\cdot(\vec{E}\times\vec{B})=\partial_i\varepsilon_{ijk}E_jB_k=\varepsilon_{ijk}B_k\partial_iE_j+\varepsilon_{ijk}E_j\partial_iB_k=\vec{B}\cdot(\nabla\times\vec{E})-\vec{E}\cdot(\nabla\times\vec{B})\]
Usando ora la legge di Faraday
\[\vec{E}\cdot(\nabla\times\vec{B})=-\frac{1}{c}\vec{B}\cdot\der{\vec{B}}{t}-\nabla\cdot(\vec{E}\times\vec{B})=-\frac{c}{2}\der{B^2}{t}-\nabla\cdot(\vec{E}\times\vec{B})\]
Infine otteniamo
\[\frac{1}{8\pi}\der{}{t}\left(E^2+B^2\right)+\frac{c}{4\pi}\nabla\cdot(\vec{E}\times\vec{B})+w=0\]
Sembra abbastanza naturale definire
\[u_{\mathrm{em}}=\frac{E^2+B^2}{8\pi}\]
\[\vec{S}=\frac{c}{4\pi}\vec{E}\times\vec{B}\]
La prima relazione è soddisfacente (ad esempio, è in accordo con il caso statico, inoltre l'energia è definita a meno di una costante). La seconda è meno legittima, ma nella maggior parte dei casi non ci sono problemi. Il vettore $\vec{S}$ si chiama vettore di Poynting.
\newpage
\section{1 febbraio 2018}
\subsection{Impulso dei campi e tensore degli sforzi di Maxwell}
In maniera analoga a quanto visto nella lezione precedente, supponiamo che un certo impulso $\vec{G}$ sia associato ai campi. Preso un volume $V$ delimitato da una superficie $A$, in componenti avremo una legge di conservazione della forma
\[\frac{\dif G_i}{\dif t}=-F_i+\oint_A T_{ij}n_j\dif A\]
Il primo termine a secondo membro è la forza (cambiata di segno) che i campi esercitano sulle sorgenti. Il secondo termine rappresenta un generico flusso di impulso, e dato che quest'ultimo è un vettore abbiamo dovuto introdurre un tensore a due indici, $T_{ij}$. $n_j$ è ovviamente la componente $j$-esima della normale uscente. In forma locale, la relazione precedente si scrive come
\[\der{g_i}{t}=-f_i+\partial_jT_{ij}\]
dove $\vec{g}$ è la densità di $\vec{G}$ e \[\vec{f}=\rho\vec{E}+\frac{\vec{J}}{c}\times\vec{B}\]
è la densità di $\vec{F}$. L'ultimo termine può essere interpretato come la divergenza di $T_{ij}$\footnote{Vale il seguente abuso: un integrale superficiale in cui compare il termine $n_i\dif S$ può essere trasformato in un integrale di volume, a patto di effettuare la sostituzione $n_i\dif S\mapsto\dif^3x\partial_i$, con la derivata che agisce su tutta la funzione integranda.}. Scriviamo ora la componente $i$-esima di $\vec{f}$
\[f_i=\rho E_i+\frac{1}{c}\varepsilon_{ijk}J_jB_k=\frac{1}{4\pi}E_i\partial_jE_j+\frac{1}{4\pi}\varepsilon_{ijk}\left(\varepsilon_{jlm}\partial_lB_m-\frac{1}{c}\der{E_j}{t}\right)B_k\]
Simmetrizziamo l'espressione precedente aggiungendo $\vec{B}(\nabla\cdot\vec B)$ e utilizzando la legge di Faraday
\[f_i=\frac{1}{4\pi}E_i\partial_jE_j+\frac{1}{4\pi}B_i\partial_jB_j+\frac{1}{4\pi}\varepsilon_{ijk}\left(\varepsilon_{jlm}\partial_lB_m-\frac{1}{c}\der{E_j}{t}\right)B_k+\frac{1}{4\pi}\varepsilon_{ijk}\left(\varepsilon_{jlm}\partial_lE_m+\frac{1}{c}\der{B_j}{t}\right)E_k\]
Ricordando ora che
\[\varepsilon_{ijk}\varepsilon_{jlm}=\delta_{kl}\delta_{im}-\delta_{km}\delta_{il}\]
Otteniamo
\[f_i=-\frac{1}{4\pi c}\der{}{t}\varepsilon_{ijk}E_jB_k+\frac{1}{4\pi}\partial_j\left[E_iE_j+B_iB_j-\frac{1}{2}\delta_{ij}\left(E^2+B^2\right)\right]\]
Di conseguenza otteniamo
\[\vec{g}=\frac{\vec{E}\times\vec{B}}{4\pi c}=\frac{\vec{S}}{c^2}\]
\[T_{ij}=\frac{1}{4\pi}\left[E_iE_j+B_iB_j-\frac{1}{2}\delta_{ij}\left(E^2+B^2\right)\right]\]
Tale tensore è detto tensore degli sforzi di Maxwell. Notiamo infine che è possibile trovare una legge di conservazione analoga per il momento angolare $\vec{L}$: in particolare, si ha
\[(\partial_t\vec{r}\times\vec{g}+\vec{r}\times\vec{f})\cdot\hat{x}_i=\varepsilon_{ijk}r_j\partial_lT_{kl}\]
Il termine $\vec{r}\times\vec{g}$ è la densità di momento angolare trasportato dai campi, il termine $\vec{r}\times\vec{f}$ è il momento torcente per unità di volume agente sulle sorgenti. Infine, il termine a secondo membro è il prodotto vettoriale tra il vettore posizione e la divergenza del tensore degli sforzi.
\subsection{Pressione di radiazione}
Consideriamo un'onda piana che si propaga lungo l'asse $z$ polarizzata linearmente lungo $x$, vale a dire
\[\vec{E}=E\hat{x}e^{i(kz-\omega t)}\]
\[\vec{B}=E\hat{y}e^{i(kz-\omega t)}\]
Immaginiamo di avere una parete completamente assorbente parallela al piano $xy$. La forza agente sulla parete è lungo $z$ e, presa una porzione della stessa di area $\sigma$, può essere calcolata come
\[F=-\sigma T_{zz}=\sigma\frac{A^2}{8\pi}\]
dove con $A$ si è indicata l'ampiezza media dei campi (in generale, il periodo dell'oscillazione dei campi è molto minore della prontezza degli strumenti che abbiamo a disposizione: in tal senso il valor medio di $F$ è più significativo del valore istantaneo). Sulla parete viene esercitata una pressione, detta pressione di radiazione, pari a
\[p=\frac{A^2}{8\pi}\]
Si ottiene lo stesso risultato considerando un cilindro di area di base $\sigma$ e altezza $c\Delta t$ e calcolando la quantità di moto assorbita dalla parete nel tempo $\Delta t$. Questa seconda interpretazione ci permette anche di dire che, in caso di parete perfettamente riflettente, la pressione di radiazione è
\[p=\frac{A^2}{4\pi}\]
In effetti, se consideriamo l'onda riflessa dalla parete (cioè l'onda che si ottiene invertendo $\vec{k}$ e $\vec{E}$, ma non $\vec{B}$), all'interfaccia il campo elettrico è nullo e il campo magnetico raddoppia, quindi complessivamente la forza agente sulla parete viene moltiplicata per un fattore 2. Consideriamo infine il caso in cui $\vec{k}$ appartenga al piano $xz$ e sia $\theta$ l'angolo tra il vettore d'onda e l'asse $z$. In tal caso, su una parete completamente assorbente le componenti della forza sono
\[F_z=-\sigma T_{zz}=\frac{A^2}{8\pi}\cos^2\theta\]
\[F_y=0\]
\[F_x=-\sigma T_{xz}=\frac{A^2}{8\pi}\sin\theta\cos\theta\]
\newpage
\section{8 febbraio 2018}
\subsection{Generazione di campi elettromagnetici}
Finora abbiamo studiato i campi elettromagnetici senza preoccuparci troppo delle sorgenti che li generano: di fatto, ci siamo limitati a studiare le equazioni di Maxwell nel caso in cui $\rho\equiv0$ e $\vec{J}\equiv0$. Ci proponiamo ora di risolvere le equazioni di Maxwell in tutto lo spazio. L'equazione sulla divergenza di $\vec{B}$ è equivalente all'esistenza di un potenziale vettore $\vec{A}$ tale che
\begin{equation}
\vec{B}=\nabla\times\vec{A}\label{vecpot}
\end{equation}
Inseriamo il potenziale vettore nella legge di Faraday:
\[\nabla\times\vec{E}=-\frac{1}{c}\nabla\times\der{\vec{A}}{t}\]
\[\nabla\times\left(\vec{E}+\frac{1}{c}\der{\vec{A}}{t}\right)=0\]
L'ultima equazione ci dice che esiste un potenziale scalare $V$ tale che
\begin{equation}\vec{E}=-\nabla V-\frac{1}{c}\der{\vec{A}}{t}\label{scalpot}\end{equation}
Inseriamo la (\ref{vecpot}) e la (\ref{scalpot}) nelle rimanenti equazioni di Maxwell: sviluppando i calcoli, si ottiene
\[-\lap V-\frac{1}{c}\der{}{t}\nabla\cdot\vec{A}=4\pi\rho\]
\[\nabla(\nabla\cdot\vec{A})-\lap\vec{A}+\frac{1}{c}\der{}{t}(\nabla V)+\frac{1}{c^2}\der[2]{\vec{A}}{t}=\frac{4\pi}{c}\vec{J}\]
Le equazioni precedenti sono molto accoppiate, ma ricordiamoci che abbiamo una buona libertà di scelta dei potenziali. A tal proposito, consideriamo una trasformazione del tipo
\begin{align*}
	\vec{A}&\mapsto\vec{A}'=\vec{A}+\nabla\psi\\
	V&\mapsto V'=V-\frac{1}{c}\der{\psi}{t}
\end{align*}
dove $\psi$ è una qualunque funzione scalare della posizione e del tempo. Una trasformazione di questo tipo viene detta trasformazione di gauge e si verifica tramite calcolo diretto che $\vec{A}'$ e $V'$ sono potenziali per gli stessi campi di $\vec{A}$ e $V$. Il nostro problema ora è ridotto a cercare una particolare scelta di gauge in cui le equazioni precedenti sono trattabili abbastanza facilmente. Riportiamo di seguito i due gauge più noti.
\subsubsection{Gauge di Coulomb}
Nel gauge di Coulomb, o gauge trasverso, si richiede
\[\nabla\cdot\vec{A}=0\]
Mostriamo che questa scelta di gauge è effettivamente possibile: se $\nabla\cdot\vec{A}=\varphi$, con la notazione del paragrafo precedente si ha
\[\nabla\cdot\vec{A}'=\varphi+\lap\psi\]
Dunque è sufficiente scegliere $\psi$ in modo che risolva l'equazione di Poisson
\[\lap\psi+\varphi=0\]
Con questa scelta di gauge, le equazioni per i potenziali si riducono a
\[\lap V=-4\pi\rho\]
\[\frac{1}{c^2}\der[2]{\vec{A}}{t}-\lap\vec{A}=\frac{4\pi}{c}\vec{J}-\frac{1}{c}\der{}{t}(\nabla V)\]
La prima equazione ha l'ovvia soluzione
\[V(\vec{r},t)=\int\frac{\rho(\vec{r}',t)}{|\vec{r}-\vec{r}'|}\dif^3r'\]
Questo potenziale, oltre a coincidere con quello usuale del caso statico, è "istantaneo", nel senso che un cambiamento di $\rho$ nel tempo si propaga istantaneamente in tutti i punti dello spazio. Ciò non è in contraddizione con il principio di causalità, perchè ciò che ha davvero significato fisico sono i campi, non i potenziali. Inoltre, nell'equazione per $\vec{A}$ il potenziale scalare compare come termine di sorgente, e si può verificare per calcolo diretto che anche $\vec{A}$ ha una componente che si propaga istantaneamente (ma che, per fortuna, elide il contributo dovuto a $V$).
\subsubsection{Gauge di Lorenz}
Nel gauge di Lorenz\footnote{In onore di Ludvig Lorenz, da non confondere con il ben più famoso Hendrik Lorentz.} si impone la cosidetta \emph{condizione di Lorenz}
\[\nabla\cdot\vec{A}+\frac{1}{c}\der{V}{t}=0\]
Tale scelta di gauge è effettivamente possibile: se si ha
\[\nabla\cdot\vec{A}+\frac{1}{c}\der{V}{t}=-\zeta\]
e se utilizziamo la notazione precedente per i campi trasformati, si ha
\[\nabla\cdot\vec{A}'+\frac{1}{c}\der{V'}{t}=-\zeta+\lap\psi-\frac{1}{c^2}\der[2]{\psi}{t}\]
Dunque se $\psi$ soddisfa l'equazione d'onda disomogenea
\[\lap\psi-\frac{1}{c^2}\der[2]{\psi}{t}=\zeta\]
allora la condizione di gauge è soddisfatta. In questo caso, le equazioni per i potenziali si disaccoppiano e diventano equazioni d'onda disomogenee
\[\lap V-\frac{1}{c^2}\der[2]{V}{t}=-4\pi\rho\]
\[\lap\vec{A}-\frac{1}{c^2}\der[2]{\vec{A}}{t}=-\frac{4\pi}{c}\vec{J}\]
Risolviamo solo l'equazione per il potenziale scalare, dato che da questa si ottiene banalmente la soluzione per il potenziale vettore. A tale scopo introduciamo la trasformata di Fourier ambo i membri, ossia scriviamo
\[V(\vec{r},t)=\fourier{\hat{V}(\vec{r},\omega)e^{-i\omega t}}{\omega}\]
\[\rho(\vec{r},t)=\fourier{\hat{\rho}(\vec{r},\omega)e^{-i\omega t}}{\omega}\]
Ricordiamo che si ha
\[\hat{V}(\vec{r},\omega)=\fourier{V(\vec{r},t)e^{i\omega t}}{t}\]
\[\hat{\rho}(\vec{r},\omega)=\fourier{\rho(\vec{r},t)e^{i\omega t}}{t}\]
Se supponiamo che la soluzione sia sufficientemente regolare, allora l'equazione d'onda si riduce a
\[\left(\lap+\frac{\omega^2}{c^2}\right)\hat{V}(\vec{r},\omega)=-4\pi\hat{\rho}(\vec{r},\omega)\]
Risolviamo quest'ultima equazione, detta equazione di Helmoltz, tramite le funzioni di Green. In particolare, cerchiamo una soluzione della forma\footnote{Si potrebbe anche non passare per le funzioni di Green e introdurre un'opportuna trasformata di Fourier lungo le coordinate spaziali. In tal caso si ottiene
$\tilde{V}(\vec{x},\omega)=4\pi\tilde{\rho}(\vec{x},\omega)/(|\vec{x}|^2-\omega^2/c^2)$, avendo posto
\[\tilde{\rho}(\vec{x},\omega)=\frac{1}{(2\pi)^{3/2}}\int_{\R^3}\hat{\rho}(\vec{r},\omega)e^{i\vec{x}\cdot\vec{r}}\dif^3r\]
e analogamnte per $\tilde{V}(\vec{x},\omega)$. Questa soluzione coincide con la soluzione trovata nel testo, ma servono alcuni strumenti più sofisticati per mostrare tale uguaglianza.}
\[\hat{V}(\vec{r},\omega)=\int\hat{\rho}(\vec{r}',\omega)G(\vec{r},\vec{r}',\omega)\dif^3r'\]
Ovviamente $G$ deve risolvere
\begin{equation}\left(\lap+\frac{\omega^2}{c^2}\right)G(\vec{r},\vec{r}',\omega)=-4\pi\delta(\vec{r}-\vec{r}')\label{green}\end{equation}
La relazione precedente ha una chiara interpretazione fisica: stiamo vedendo la distribuzione di sorgenti come tante cariche puntiformi infinitesime. Vogliamo trovare il potenziale generato da ciascuna di queste cariche puntiformi e poi utilizzare il principio di sovrapposizione per ricavare la soluzione generale. Dato che una sorgente puntiforme è chiaramente a simmetria sferica, ci aspettiamo che anche il potenziale da essa generato goda della stessa simmetria. Per questo motivo, mostriamo ora che la soluzione è effettivamente un'onda sferica del tipo
\[G(\vec{r},\vec{r}',\omega)=\frac{e^{i\omega|\vec{r}-\vec{r}'|/c}}{|\vec{r}-\vec{r}'|}\]
Non è restrittivo supporre $\vec{r}'=0$. Sotto tale ipotesi, per $\vec{r}\neq0$ e in coordinate sferiche, otteniamo
\[\left(\lap+\frac{\omega^2}{c^2}\right)G(\vec{r},0,\omega)=\frac{1}{r}\der[2]{r G}{r}+\frac{\omega^2}{c^2}G=0\]
dove si è posto $r=|\vec{r}|$. Per mostrare che la soluzione è valida anche per $\vec{r}=0$, integriamo il primo membro della (\ref{green}) su una palla di raggio $\varepsilon$ centrata nell'origine. Otteniamo
\[\int_{B_\varepsilon}\left(\lap+\frac{\omega^2}{c^2}\right)G(\vec{r},0,\omega)\dif^3r=\int_{\partial B_\varepsilon}\nabla G(\vec{r},0,\omega)\cdot\dif\vec{S}+\frac{\omega^2}{c^2}\int_{B_{\varepsilon}}G(\vec{r},0,\omega)\dif^3r\]
Il secondo integrale si annulla quando $\varepsilon\to0$. Il primo vale invece
\[\int_{\partial B_\varepsilon}\nabla G(\vec{r},0,\omega)\cdot\dif\vec{S}=-4\pi e^{i\omega\varepsilon/c}+\frac{4\pi i\omega\varepsilon}{c}e^{i\omega\varepsilon/c}\]
Quando $\varepsilon\to0$, l'integrale converge a $-4\pi$, dunque $G$ è soluzione della (\ref{green}). Antitrasformando ricaviamo infine
\[V(\vec{r},t)=\fourier{\left(\int \hat{\rho}(\vec{r}',\omega)G(\vec{r},\vec{r}',\omega)\dif^3r'\right)e^{-i\omega t}}{\omega}=\fourier{\int \hat{\rho}(\vec{r}',\omega)\frac{e^{i\omega(|\vec{r}-\vec{r}'|/c-t)}}{|\vec{r}-\vec{r}'|}\dif^3r'}{\omega}=\]\begin{equation}=\int\frac{\rho(\vec{r}',t-|\vec{r}-\vec{r}'|/c)}{|\vec{r}-\vec{r}'|}\dif^3r'\label{vrit}\end{equation}
In maniera del tutto analoga si ottiene
\begin{equation}\vec{A}(\vec{r},t)=\frac{1}{c}\int\frac{\vec{J}(\vec{r}',t-|\vec{r}-\vec{r}'|/c)}{|\vec{r}-\vec{r}'|}\dif^3r'\label{arit}\end{equation}
Come ultima cosa, verifichiamo che i potenziali soddisfano effettivamente la condizione di gauge. Abbiamo
\begin{align*}\der{V}{t}&=\frac{1}{\sqrt{2\pi}}\der{}{t}\int\int_{-\infty}^{+\infty}\hat{\rho}(\vec{r}',\omega)\frac{e^{i\omega(|\vec{r}-\vec{r}'|/c-t)}}{|\vec{r}-\vec{r}'|}\dif\omega\,\dif^3r'=\\&=\frac{1}{\sqrt{2\pi}}\int\int_{-\infty}^{+\infty}-i\omega\hat{\rho}(\vec{r}',\omega)\frac{e^{i\omega(|\vec{r}-\vec{r}'|/c-t)}}{|\vec{r}-\vec{r}'|}\dif\omega\,\dif^3r'\\\nabla\cdot\vec{A}&=\frac{1}{c\sqrt{2\pi}}\nabla\cdot\int\int_{-\infty}^{+\infty}\hat{\vec{J}}(\vec{r}',\omega)\frac{e^{i\omega(|\vec{r}-\vec{r}'|/c-t)}}{|\vec{r}-\vec{r}'|}\dif\omega\,\dif^3r'=\\&=\frac{1}{c\sqrt{2\pi}}\int\int_{-\infty}^{+\infty}\hat{\vec{J}}(\vec{r}',\omega)\cdot\nabla\frac{e^{i\omega(|\vec{r}-\vec{r}'|/c-t)}}{|\vec{r}-\vec{r}'|}\dif\omega\,\dif^3r'=\\&=-\frac{1}{c\sqrt{2\pi}}\int\int_{-\infty}^{+\infty}\left[\nabla'\cdot\frac{\hat{\vec{J}}(\vec{r}',\omega)e^{i\omega(|\vec{r}-\vec{r}'|/c-t)}}{|\vec{r}-\vec{r}'|}-\frac{e^{i\omega(|\vec{r}-\vec{r}'|/c-t)}}{|\vec{r}-\vec{r}'|}\nabla'\cdot\hat{\vec{\vec{J}}}(\vec{r}',\omega)\right]\dif\omega\,\dif^3r'\end{align*}
Nell'ultima espressione, il primo termine si annulla sotto le solite opportune ipotesi di localizzione. Per il secondo, sappiamo dall'equazione di continuità che
\[\der{\rho}{t}(\vec{r},t)+\nabla\cdot\vec{J}(\vec{r},t)=0\]
Trasformando in Fourier rispetto al tempo otteniamo quindi
\begin{equation}\label{continuitafourier}-i\omega\hat{\rho}(\vec{r},\omega)+\nabla\cdot\hat{\vec{J}}(\vec{r},\omega)=0\end{equation}
Da cui si conclude facilmente che i potenziali trovati soddisfano effettivamente la condizione di gauge.
Diamo un'interpretazione fisica dei risultati trovati: un cambio nel tempo delle sorgenti $\rho$ e $\vec{J}$ provoca una variazione nei campi, ma questa variazione si propaga a velocità finita. Di fatto, in un punto $\vec{r}$ e ad un istante $t$ il potenziale scalare è influenzato dalla densità di carica posta in $\vec{r}'$ all'istante $t-|\vec{r}-\vec{r}'|/c$. Per questo motivo le (\ref{vrit}) e (\ref{arit}) prendono il nome di potenziali ritardati.

Come ultima osservazione, notiamo che a rigore la (\ref{green}) ammette anche la soluzione
\begin{equation}G(\vec{r},\vec{r}',\omega)=\frac{e^{-i\omega|\vec{r}-\vec{r}'|/c}}{|\vec{r}-\vec{r}'|}\label{avanzati}\end{equation}
Questa soluzione corrisponde a un'onda sferica entrante dall'infinito in $\vec{r}'$, e pertanto non ha un vero significato fisico. In altre parole, quando specifichiamo le condizioni al bordo in un problema di elettrodinamica, dobbiamo anche specificare le condizioni sul "bordo" temporale: sembra sensato accettare la condizione di Sommerfeld, secondo cui in tempi remoti non erano presenti perturbazioni. Tra l'altro, se scriviamo i potenziali associati alla (\ref{avanzati}), otteniamo
\begin{equation*}V(\vec{r},t)=\int\frac{\rho(\vec{r}',t+|\vec{r}-\vec{r}'|/c)}{|\vec{r}-\vec{r}'|}\dif^3r'\end{equation*}
\begin{equation*}\vec{A}(\vec{r},t)=\frac{1}{c}\int\frac{\vec{J}(\vec{r}',t+|\vec{r}-\vec{r}'|/c)}{|\vec{r}-\vec{r}'|}\dif^3r'\end{equation*}
Questi potenziali, che per ovvi motivi sono detti potenziali avanzati, cozzano abbastanza con il principio di causalità: i potenziali in $\vec{r}$ all'istante $t$ sono influenzati dalla carica che in $\vec{r}'$ sarà presente nel futuro, e più precisamente all'istante $t+|\vec{r}-\vec{r}'|/c$.
\newpage
\section{15 febbraio 2018}
\subsection{Irraggiamento}
Consideriamo una sorgente localizzata di dimensione caratteristica $a$ e supponiamo, per semplicità, che la sorgente sia monocromatica di frequenza $\omega$. Vogliamo studiare i campi all'ordine più basso a una distanza $R\gg a$ da questa sorgente. Questa regione di spazio prende il nome di zona di radiazione (aggiungendo l'ulteriore ipotesi che la lunghezza d'onda $\lambda$ sia molto minore di $R$), mentre nelle vicinanze della sorgente si parla di zona di campo prossimo. Notiamo preliminarmente che sotto tali ipotesi è sufficiente conoscere il solo potenziale vettore per determinare entrambi i campi. Infatti, ovviamente si ha
\[\vec{B}=\nabla\times\vec{A}\]
D'altro canto, la monocromaticità e la quarta equazione di Maxwell ci danno 
\[\vec{E}=\frac{ic}{\omega}\nabla\times\vec{B}\]
Riprendendo i calcoli della lezione precedente e scrivendo, come d'uopo,
\[\vec{J}(\vec{r},t)=\vec{J}_0(\vec{r})e^{-i\omega t}\]
otteniamo il potenziale vettore
\[\vec{A}(\vec{r},t)=\frac{e^{-i\omega t}}{c}\int\frac{\vec{J}_0(\vec{r}')e^{i\omega|\vec{r}-\vec{r}'|/c}}{|\vec{r}-\vec{r}'|}\dif^3r'\]
Facciamo ora le seguenti approssimazioni: posto $R=|\vec{r}|$ e $\hat{n}=\vec{r}/R$, al primo ordine si ha $|\vec{r}-\vec{r}'|\simeq R-\vec{r}'\cdot\hat{n}$. Supponiamo di poter essere ancora più brutali e di sostituire direttamente $R$ nel denominatore dell'integrale, ottenendo
\[\vec{A}(\vec{r},t)=\frac{e^{i\omega(R/c-t)}}{cR}\int\vec{J}_0(\vec{r}')e^{-i\omega\vec{r}'\cdot\hat{n}}/c\dif^3r'\]
Facciamo ora l'ulteriore ipotesi $a\ll\lambda$. Questa assunzione equivale a richiedere che il tempo impiegato dalla luce a percorrere la distanza $a$ sia molto minore del periodo con cui oscilla l'onda generata, ossia equivale a richiedere di essere in regime non relativistico. Questa assunzione permette di approssimare l'esponenziale all'interno dell'integrale con 1, dato che $k\propto\lambda^{-1}$ e $|\vec{r}'|\sim a$. Otteniamo infine
\[\vec{A}(\vec{r},t)=\frac{e^{i\omega(R/c-t)}}{cR}\int\vec{J}_0(\vec{r}')\dif^3r'\]
Questa espressione non ci è nuova, avendola già incontrata in magnetostatica. In quell'occasione questo integrale era nullo, ora non lo è più. Consideriamone ad esempio la componente lungo l'asse $x$
\[\int J_{0,x}(\vec{r}')\dif^3r'=\int\left(\nabla'\cdot(x'\vec{J}_0(\vec{r}'))-x'\nabla'\cdot\vec{J}_0(\vec{r}')\right)\dif^3r'\]
Il primo termine, trasformato con il teorema della divergenza, si annulla perchè la sorgente è localizzata. Il secondo termine sappiamo già trattarlo grazie alla (\ref{continuitafourier}), e in definitiva otteniamo
\[\int\vec{J}_0(\vec{r}')\dif^3r'=-i\omega\vec{p}\]
\[\vec{A}(\vec{r},t)=-\frac{ik\vec{p}}{R}e^{i(k\vec{r}\cdot\hat{n}-\omega t)}\]
dove $\vec{p}$ è il momento di dipolo della distribuzione. Da questo è possibile determinare i campi (noi ci limitiamo al termine dominante, dato che i risultati ottenuti finora sono già piuttosto approssimativi)
\[\vec{B}=\frac{k^2}{R}e^{ik\vec{r}\cdot\hat{n}}(\hat{n}\times\vec{p})\]
\[\vec{E}=\frac{k^2}{R}e^{ik\vec{r}\cdot\hat{n}}(\hat{n}\times\vec{p})\times\hat{n}\]
dove la dipendenza tempporale dei campi è implicita in $\vec{p}$. Per completezza, i campi di radiazione di un dipolo monocromatico oscillante sono
\[\vec{B}=\frac{k^2e^{ikr}}{r}\left(1-\frac{1}{ikr}\right)\hat{n}\times\vec{p}\]
\[\vec{E}=\frac{e^{ikr}}{r}\left[k^2(\hat{n}\times\vec{p})\times\hat{n}+\frac{1-ikr}{r^2}\left(3(\vec{p}\cdot\hat{n})\hat{n}-\vec{p}\right)\right]\]
Discutiamo ora un caso semplice: supponiamo $\vec{p}=p\hat{z}e^{-i\omega t}$ e sia $\theta$ l'angolo tra $\vec{r}$ e l'asse $z$. In tal caso il vettore di Poynting è 
\[\vec{S}=\frac{ck^4p^2\sin^2\theta\hat{n}}{4\pi R^2}\cos^2(kr-\omega t)\]
La media su un periodo è quindi
\[\langle\vec{S}\rangle=\frac{ck^4p^2\sin^2\theta\hat{n}}{8\pi R^2}\]
Ciò significa che la potenza irraggiata per unità di angolo solido è
\[\frac{\dif I}{\dif\Omega}=\frac{ck^4p^2}{8\pi}\sin^2\theta\]
Tale potenza è nulla lungo l'asse del dipolo ed è massima lungo il piano equatoriale. Infine, la potenza totale irraggiata è
\[P=\int\frac{\dif I}{\dif\Omega}\dif\Omega=\frac{ck^4p^2}{3}=\frac{p^2\omega^4}{3c^3}\]
Questa relazione si chiama formula di Larmor (in approssimazione non relativistica).
\subsection{Diffusione}
Consideriamo una particella di massa $m$ e carica $q$ soggetta a una forza di richiamo di tipo elastico di frequenza $\omega_0$. Facciamo incidere su tale particella un'onda piana di ampiezza $E_0$ e di frequenza $\omega$. Supponiamo che l'ampiezza d'oscillazione della particella sia molto minore della lunghezza d'onda $\lambda=2\pi c/\omega$, in modo da poter trascurare la dipendenza spaziale nell'onda piana. In queste approssimazioni l'equazione del moto è
\[m\ddot{z}=-m\omega_0^2z+qE_0e^{-i\omega t}\]
La soluzione a regime è ovviamente
\[z(t)=\frac{qE_0}{m}\frac{1}{\omega_0^2-\omega^2}e^{-i\omega t}\]
La dipendenza dal rapporto $q/m$ implica che facendo incidere un'onda piana su un atomo l'elettrone oscilla molto più del nucleo. Inoltre, dalla formula di Larmor otteniamo la potenza diffusa
\[P=\left(\frac{q^2}{mc^2}\right)^2\frac{cE_0^2}{3}\frac{\omega^4}{(\omega_0^2-\omega^2)^2}\]
La quantità fra parentesi è chiamata raggio classico dell'elettrone. Infine, definendo la sezione d'urto totale di diffusione $\sigma$ come il rapporto tra la potenza diffusa e l'intensità dell'onda incidente, otteniamo
\[\sigma=\frac{8\pi}{3}\left(\frac{q^2}{mc^2}\right)^2\frac{\omega^4}{(\omega_0^2-\omega^2)^2}\]
Distinguiamo ora due regimi: se $\omega\ll\omega_0$ siamo nel regime Rayleigh e la potenza diffusa è approssimativamente proporzionale a $\omega^4$. Questa dipendenza pronunciata dalla frequenza spiega perché il cielo è azzurro di giorno\footnote{In realtà, questo modello dice che il colore del cielo dovrebbe essere in prevalenza viola. Purtroppo l'occhio umano è più sensibile alle frequenze dell'azzurro che a quelle del violetto.} e rosso all'alba e al tramonto. Al contrario, se $\omega\gg\omega_0$ siamo nel regime di Thomson e la potenza diffusa è pressoché indipendente dalla frequenza.
\newpage
\section{22 febbraio 2018}
\subsection{Trasformazioni di Galileo e problemi con l'elettrodinamica}
Sappiamo che, secondo la relatività galileiana, il cambio di coordinate tra due sistemi di riferimento inerziali $S$ e $S'$ in cui l'origine di $S'$ si muove a velocità $\vec{u}$ costante rispetto a $S$ è
\[\vec{r}'=\vec{r}-\vec{u}t\]
Più esplicitamente, se gli assi dei due sistemi sono paralleli e $\vec{u}=u\hat{x}$, le trasformazioni di Galileo sono
\[\begin{cases}
x'=x-ut\\y'=y\\z'=z\\t'=t\end{cases}\]
In particolare, l'ultima relazione implica l'esistenza di un tempo assoluto, cioè uguale per tutti gli osservatori inerziali. Come conseguenza, la velocità e l'accelerazione di un corpo trasformano come
\[\begin{cases}
\vec{v}'=\vec{v}-\vec{u}\\\vec{a}'=\vec{a}
\end{cases}\]
Inoltre, se supponiamo che le forze di interazione tra due punti materiali posti in $\vec{r}_1$ e $\vec{r}_2$ dipendano unicamente $|\vec{r}_1-\vec{r}_2|$, la forza in $S'$ è uguale alla forza in $S$
\[\vec{F}'=\vec{F}\]
Questo significa che la seconda legge di Newton è invariante sotto trasformazioni di Galileo.
Al contrario, le leggi dell'elettromagnetismo non sono invarianti sotto trasformazioni di Galileo.
Consideriamo a tal proposito l'equazione d'onda (per semplicità, limitiamoci al caso unidimensionale) e vediamo come trasforma sotto trasformazioni galileiane. Per prima cosa, vediamo come una derivazione viene trasformata passando da $S$ a $S'$
\[\der{}{t}=\der{}{t'}\der{t'}{t}+\der{}{x'}\der{x'}{t}=\der{}{t'}-u\der{}{x'}\]
\[\der{}{x}=\der{}{t'}\der{t'}{x}+\der{}{x'}\der{x'}{x}=\der{}{x'}\] Dunque se in $S$ si ha
\[\left(\der[2]{}{x}-\frac{1}{c^2}\der[2]{}{t}\right)h(x,t)=0\]
allora in $S'$ si ha
\[\left[\left(1-\frac{u^2}{c^2}\right)\der[2]{}{x'}-\frac{1}{c^2}\der[2]{}{t'}+\frac{2u}{c^2}\frac{\partial^2}{\partial x'\partial t'}\right]h(x',t')=0\]
Il termine problematico è quello misto, dato che se questo fosse assente potremmo definire $c'=c\sqrt{1-u^2/c^2}$ e l'equazione delle onde risulterebbe invariante in forma sotto trasformazioni galileiane. In realtà, il fatto che l'equazione di D'Alembert non sia invariante sotto queste trasformazioni non è sorprendente, dato che per i fenomeni ondulatori nei mezzi è stata ricavata nel sistema in cui il mezzo è a riposo. I fisici di fine Ottocento si chiesero dunque in quale mezzo $-$ il famigerato etere luminifero $-$ si propagasse la luce, in quale sistema inerziale questo fosse a riposo e, infine, in quali sistemi di riferimento sono valide le equazioni di Maxwell. Queste domande furono ulteriormente stimolate da alcuni esperimenti storici.
\subsection{Esperimenti di Bradley, Michelson-Morley e Fizeau}
Già nel 1728 Bradley osservò l'aberrazione stellare. Questo effetto, da non confondere con la parallasse, è dovuto al moto della Terra intorno al Sole e alla finitezza di $c$. Se consideriamo ad esempio una stella lontana, per osservarla dobbiamo inclinare un telescopio rispetto alla congiungente con la stella, come si può osservare in figura
\begin{figure}[h]
	\centering
	\scalebox{1}{\begin{tikzpicture}
		\draw[-stealth](-3,0)--(-3,-1);		
		\draw[-stealth](-1.5,0)--(-1.5,-1);
		\draw[-stealth](0,0)--(0,-1);
		\draw[-stealth](1.5,0)--(1.5,-1);
		\draw[-stealth](3,0)--(3,-1);
		\draw node at(0,0.5){Luce dalle stelle lontane};
		\draw (2.5,-3) arc [start angle=0, end angle=360, x radius=2.5, y radius=0.5];
		\draw node at(0,-3.5)[circle, fill]{};
		\draw [-stealth] (0,-3.5)--(1,-3.5)node[below]{$\vec{u}$};
		\draw [thick] (0,-3.5)--(0.2,-3);
		\draw node at(0,-2.5)[circle, fill]{};
		\draw [-stealth] (0,-2.5)--(-1,-2.5)node[above]{$\vec{u}$};
		\draw [thick] (0,-2.5)--(-0.2,-2);		
	\end{tikzpicture}}
\end{figure}
La sbarretta sul pianeta rappresenta appunto la posizione di un telescopio che intercetta la luce proveniente dalle stelle lontane. La sua inclinazione rispetto al piano orbitale è data da
\[\alpha\approx\tan\alpha=\frac{u}{c}\sim10^{-4}\]
Di conseguenza, il sistema delle stelle fisse era un ottimo candidato come sistema dell'etere. A questo punto ci si poneva il problema di calcolare la velocità relativa della Terra rispetto a tale sistema. Michelson e Morley realizzarono a tal proposito il seguente apparato sperimentale
\begin{figure}[h]
	\centering
	\scalebox{1.3}{\begin{tikzpicture}
			\draw node at(-3.1,-0.1){\LARGE *};
			\draw [-stealth](-3,0)--(-1.5,0);
			\draw (-1.5,0)--(0,0);
			\draw [thick](-0.5,-0.5)--(0.5,0.5);
			\draw [-stealth](0,0)--(1.5,0);
			\draw [-stealth](0,0)--(0,1);
			\draw (0,1)--(0,2);
			\draw (1.5,0)--(3,0);
			\draw [very thick](3,-0.5)--(3,0.5);
			\draw (3,0.5)--(3.15,0.35);
			\draw (3,0.25)--(3.15,0.1);
			\draw (3,0)--(3.15,-0.15);
			\draw (3,-0.25)--(3.15,-0.4);
			\draw [very thick](-0.5,2)--(0.5,2);
			\draw (0.5,2)--(0.35,2.15);
			\draw (0.25,2)--(0.1,2.15);
			\draw (0,2)--(-0.15,2.15);
			\draw (-0.25,2)--(-0.4,2.15);
			\draw (-0.75,-2)--(0.75,-2);
			\draw (-0.1,-0.1)--(1.5,-0.1);
			\draw [-stealth](0.1,2)--(0.1,1);
			\draw (0.1,1)--(0.1,0);
			\draw [-stealth](3,-0.1)--(1.5,-0.1);
			\draw [-stealth](-0.1,-0.1)--(-0.1,-1);
			\draw (-0.1,-1)--(-0.1,-2);
			\draw [-stealth](0.1,0.1)--(0.1,-1);
			\draw (0.1,-1)--(0.1,-2);
			\draw node at(-0.5,1){\tiny $L_y$};
			\draw node at (1.5,-0.5){\tiny $L_x$};
			\draw [-stealth] (2,1)--(2.5,1)node[above left]{\tiny $\vec{v}$};
	\end{tikzpicture}}
\end{figure}

\noindent dove $\vec{v}$ è la velocità dell'etere rispetto alla Terra. Il tempo $T_x$ impiegato dalla luce a percorre il tratto orizzontale si calcola facilmente, notando che lungo tale braccio la velocità della luce è $c\pm v$. Di conseguenza
\[T_x=\frac{L_x}{c-v}+\frac{L_x}{c+v}=\frac{2L}{c}\frac{1}{1-v^2/c^2}\]Il tempo $T_y$ impiegato dalla luce a percorrere il tratto verticale è
\[c^2T_y^2=\frac{L_y^2}{4}+v^2T_y^2\]
Dato che nel sistema dell'etere la traiettoria del raggio è ovviamente
\begin{figure}[h]
	\centering
	\scalebox{1.3}{\begin{tikzpicture}
			\draw [very thick](-0.5,2)--(0.5,2);
			\draw (0.5,2)--(0.35,2.15);
			\draw (0.25,2)--(0.1,2.15);
			\draw (0,2)--(-0.15,2.15);
			\draw (-0.25,2)--(-0.4,2.15);
			\draw (0,0)--(0,2);
			\draw (-3,0)--(3,0);
			\draw [-stealth](-2,0)--(-1,1);
			\draw (-1,1)--(0,2);
			\draw [-stealth](0,2)--(1,1);
			\draw (1,1)--(2,0);
			\draw node at(1,-0.2){\tiny $vT_y/2$};
			\draw node at(-1,-0.2){\tiny $vT_y/2$};
			\draw node at(0.25,0.9){\tiny $L_y$};
			\draw node at(1+0.3,1+0.2){\tiny $cT_y/2$};
			\draw node at(-1+0.3,1-0.2){\tiny $cT_y/2$};
	\end{tikzpicture}}
\end{figure}
Otteniamo quindi
\[T_y=\frac{2L_y}{c}\frac{1}{\sqrt{1-v^2/c^2}}\]
La differenza di fase tra i due raggi è allora
\[\Delta\Phi=\frac
{4\pi}{\lambda\sqrt{1-v^2/c^2}}\left(\frac{L_x}{\sqrt{1-v^2/c^2}}-L_y\right)\]
A questo punto Michelson e Morley regolarono le lunghezze dei bracci in modo che $\Delta\Phi=0$, ossia
\[L_x=L_y\sqrt{1-v^2/c^2}\]
In seguito ruotarono il banco ottico (che galleggiava su una vasca di mercurio) di $90^\circ$. La differenza di fase prevista in tale configurazione è dunque
\[\Delta\Phi\frac{4\pi L_x}{\lambda\sqrt{1-v^2/c^2}}\left(\frac{1}{1-v^2/c^2}-1\right)\approx\frac{4\pi L_x}{\lambda}\frac{v^2}{c^2}\]
I valori sperimentali erano $L_x\sim 5$ m, $\lambda\sim 500$ nm, dunque il valore atteso per la differenza di fase era $\Delta\Phi\sim1$. Il valore misurato invece fu $\Delta\Phi=0$. A quel punto vennero fatte varie ipotesi per interpretare questa discrepanza. Ad esempio, si suppose che l'etere venisse trascinato dalla Terra, un po' come accade per l'atmosfera. A questo punto, Fizeau propone un esperimento per valutare quantitativamente il trascinamento dell'etere. L'apparato era costituito da un tubo a U in cui veniva fatta scorrere dell'acqua a velocità $v$.
\begin{figure}[h]
	\centering
	\scalebox{1.4}{\begin{tikzpicture}
			\draw(-2,0.5)--(-1.5,1);
			\draw [gray, thick](-1,3)--(-1,2)--(2,2)--(2,1)--(-1,1)--(-1,0);
			\draw [gray,thick] (-0.5,3)--(-0.5,2.5)--(2.5,2.5)--(2.5,0.5)--(-0.5,0.5)--(-0.5,0);
			\fill [cyan, nearly transparent](-1,3)--(-1,2)--(2,2)--(2,1)--(-1,1)--(-1,0)--(-0.5,0)--(-0.5,0.5)--(2.5,0.5)--(2.5,2.5)--(-0.5,2.5)--(-0.5,3)--cycle;
			\draw [very thick] (-2,2)--(-1.5,2.5);
			\draw [very thick] (3,2.5)--(3.5,2);
			\draw [very thick] (3,0.5)--(3.5,1);
			\draw (-2.2,-0.5)--(-1.3,-0.5);
			\draw [-stealth](-0.75,3)--(-0.75,3.5) node[below right]{\tiny $\vec{v}$};\draw [-stealth](-0.75,-0.5)--(-0.75,0) node[below right]{\tiny $\vec{v}$};
			\draw (-2,2)--(-2,2.2);
			\draw (-1.9,2.1)--(-1.9,2.3);
			\draw (-1.8,2.2)--(-1.8,2.4);
			\draw (-1.7,2.3)--(-1.7,2.5);
			\draw (-1.6,2.4)--(-1.6,2.6);
			\draw (-1.5,2.5)--(-1.5,2.7);
			\draw (3,2.5)--(3.2,2.5);			
			\draw (3.1,2.4)--(3.3,2.4);
			\draw (3.2,2.3)--(3.4,2.3);
			\draw (3.3,2.2)--(3.5,2.2);
			\draw (3.4,2.1)--(3.6,2.1);
			\draw (3.5,2)--(3.7,2);			
			\draw (3,0.5)--(3.2,0.5);			
			\draw (3.1,0.6)--(3.3,0.6);
			\draw (3.2,0.7)--(3.4,0.7);
			\draw (3.3,0.8)--(3.5,0.8);
			\draw (3.4,0.9)--(3.6,0.9);
			\draw (3.5,1)--(3.7,1);
			\draw node at(-3.1,0.75){\LARGE*};
			\draw (-3,0.85)[-stealth]--(-2.25,0.85);
			\draw [-stealth](-2.25,0.85)--(-1.65,0.85)--(-1.65,1.65);
			\draw [-stealth](-1.65,1.65)--(-1.65,2.35)--(1,2.35);
			\draw [-stealth](1,2.35)--(3.1,2.35)--(3.1,1.5);
			\draw [-stealth](3.1,1.5)--(3.1,0.65)--(1,0.65);
			\draw [-stealth](1,0.65)--(-1.85,0.65)--(-1.85,0);
			\draw (-1.85,0)--(-1.85,-0.5);
			\draw [-stealth](-1.65,0.85)--(1,0.85);
			\draw [-stealth](1,0.85)--(3.3,0.85)--(3.3,1.5);
			\draw [-stealth](3.3,1.5)--(3.3,2.15)--(1,2.15);
			\draw [-stealth](1,2.15)--(-1.8,2.15)--(-1.8,1.5);
			\draw [-stealth](-1.8,1.5)--(-1.8,0);
			\draw (-1.8,0)--(-1.8,-0.5);
 	\end{tikzpicture}}
 \end{figure}

\noindent Se l'etere viene trascinato parzialmente dall'acqua, allora nei due tratti orizzontali la velocità della luce deve essere
\[v_{\pm}=\frac{c}{n}\pm \alpha v\]
dove $\alpha$ è un opportuno coefficiente compreso tra 0 e 1  che tiene conto del trascinamento (e in particolare ci aspettiamo ha $\alpha=1$ per un trascinamento completo). Fizeau ottenne
\[\alpha=1-\frac{1}{n^2}\]
Effettivamente, se trasformiamo le velocità attraverso le trasformazioni di Lorentz otteniamo una differenza di fase pari a
\[\Delta\Phi=\frac{8\pi Lv}{\lambda c}(n^2-1)\]
dove $L$ è la lunghezza del tubo in cui viaggia la luce.
\subsection{Postulati di Einstein e trasformazioni di Lorentz}
In questo contesto si inserisce il lavoro di Einstein, che assume i seguenti postulati
\begin{enumerate}[label=(\roman*)]
	\item $c$ è una costante della natura uguale in tutti i sistemi di riferimento;
	\item le leggi della fisica sono invarianti in forma quando si passa da un sistema inerziale a un altro.
\end{enumerate}
Inoltre, si assume che lo spazio e il tempo siano omogenei e isotropi e che la relatività galileiana sia il limite per piccole velocità della relatività ristretta. 

\noindent Per prima cosa dobbiamo definire un modo per misurare le lunghezze e i tempi. Per le misure di spazio, possiamo prendere dei regoli campione tutti uguali e confrontare un oggetto a riposo con un regolo. Ovviamente, questo metodo permette anche di confrontare i regoli tra di loro, quindi possiamo effettivamente sceglierli tutti uguali. Per le misure di tempo, non confrontare direttamente due intervalli temporali relativi a due eventi separati spazialmente. Possiamo però immaginare di avere tanti orologi nell'origine del nostro sistema di riferimento. In questo punto possiamo sincronizzarli, poi possiamo iniziare a spostarli fino a un evento da misurare. Ovviamente, non possiamo essere certi a priori che dopo lo spostamento gli orologi siano ancora sincronizzati. Per fare ciò, supponiamo di aver lasciato un orologio campione nell'origine e di voler verificare che un altro orologio, posto in $\vec{r}$, sia sincronizzato con questo. Per fare questo, supponiamo di avere due osservatori, ciascuno solidale con uno dei due orologi, che si accordino prima del movimento di uno degli orologi di inviare un segnale luminoso verso $\vec{r}/2$ dopo che sia passato un tempo $\Delta t$ (ovviamente, ciascuno misurerà il proprio $\Delta t$ con il proprio orologio). Se in $\vec{r}/2$ i due segnali arrivano contemporaneamente, allora gli orologi sono ancora sincronizzati.
Troviamo ora il modo in cui trasformare le coordinate passando da un sistema inerziale a un altro. L'omogeneità dello spazio implica\footnote{Si veda la sezione successiva. Alternativamente, basta osservare che un moto rettilineo deve essere trasformato in un moto rettilineo.} che le trasformazioni devono essere lineari e l'isotropia implica che non è restrittivo limitarsi a studiare due sistemi inerziali $S$ e $S'$ con assi paralleli e tali che la velocità di $S'$ rispetto a $S$ sia rivolta lungo l'asse $x$. Le coordinate trasverse devono rimanere invariate, ossia
\[\begin{cases}
y'=y\\z'=z
\end{cases}\]
Infatti, se ad esempio un regolo campione in $S'$ parallelo a $y'$ fosse più corto di un regolo campione in $S$ parallelo a $y$, potremmo osservare i due regoli quando si sovrappongono e l'osservatore in $S'$ dedurrebbe che è il regolo dell'osservatore in $S$ ad essere più corto, che è assurdo. Per le altre coordinate, notiamo che l'origine di $S'$ ha coordinate $x=ut$, $y=z=0$ in $S$, dunque la trasformazione sarà della forma
\[ct'=ax+bct\]
\[x'=\gamma(x-\beta ct)\]
avendo posto $\beta=u/c$. $\gamma$, $a$ e $b$ sono funzioni di $\beta$ tali che
\begin{itemize}
	\item $\lim\limits_{\beta\to0}a=0$, $\lim\limits_{\beta\to0}b=1$. Infatti per basse velocità vogliamo ritrovare il caso classico;
	\item $b>0$ e $\gamma>0$, dato che possiamo assumere che $t'$ e $x'$ abbiano gli stessi versi rispettivamente di $t$ e $x$.
\end{itemize}
Consideriamo ora un impulso luminoso che nel sistema $S$ si propaga lungo l'asse $x$. Allora si deve avere
\[x^2=c^2t^2\]
\[x'^2=c^2t'^2\]
In altri termini, esiste una funzione $\lambda$ della velocità tale che
\[\lambda(v)(x^2-c^2t^2)=x'^2-c^2t'^2=(\gamma^2-a^2)x^2+(\gamma^2\beta^2-b^2)c^2t^2-2ctx(\gamma^2\beta+ab)\]
Si deve quindi imporre
\[\begin{cases}
\gamma^2-a^2=\lambda\\b^2-\gamma^2\beta^2=\lambda\\\gamma^2\beta+ab=0
\end{cases}\]
Ossia
\[\begin{cases}
\gamma=\frac{\sqrt{\lambda}}{\sqrt{1-\beta^2}}\\
a=-\gamma\beta\\
b=\gamma\end{cases}\]
Notiamo che la trasformazione inversa, cioè quella da $S'$ a $S$, si deve ottenere cambiando il segno di $\beta$. Ciò richiede $\lambda=1$, quindi in definitiva le trasformazioni di Lorentz per sistemi che traslano lungo l'asse $x$ sono
\[\begin{cases}
ct'=\gamma(ct-\beta x)\\
x'=\gamma(x-\beta ct)\\
y'=y\\
z'=z
\end{cases}\]
O, in forma matriciale
\[\left(\begin{array}{c}
ct'\\x'\\y'\\z'
\end{array}\right)=\left(\begin{array}{c c c c}
\gamma&-\gamma\beta&0&0\\-\gamma\beta&\gamma&0&0\\0&0&1&0\\0&0&0&1
\end{array}\right)\left(\begin{array}{c}
ct\\x\\y\\z
\end{array}\right)\]
Inoltre, la quantità
\[c^2t^2-x^2-y^2-z^2=\left(\begin{array}{c c c c}
ct&x&y&z
\end{array}\right)\left(\begin{array}{c c c c}
1&0&0&0\\0&-1&0&0\\0&0&-1&0\\0&0&0&-1
\end{array}\right)\left(\begin{array}{c}
ct\\x\\y\\z
\end{array}\right)\]
è un invariante sotto trasformazioni di Lorentz. Questo fatto segue banalmente da
\[\left(\begin{array}{c c c c}
\gamma&-\gamma\beta&0&0\\-\gamma\beta&\gamma&0&0\\0&0&1&0\\0&0&0&1
\end{array}\right)\left(\begin{array}{c c c c}
1&0&0&0\\0&-1&0&0\\0&0&-1&0\\0&0&0&-1
\end{array}\right)\left(\begin{array}{c c c c}
\gamma&-\gamma\beta&0&0\\-\gamma\beta&\gamma&0&0\\0&0&1&0\\0&0&0&1
\end{array}\right)=\left(\begin{array}{c c c c}
1&0&0&0\\0&-1&0&0\\0&0&-1&0\\0&0&0&-1
\end{array}\right)\]
\subsection{Derivazione alternativa delle trasformazioni di Lorentz}
Concludiamo con un'altra derivazione delle trasformazioni di Lorentz\footnote{Da una rielaborazione delle dispense di M. D'Elia.}. Come nella sezione precedente, limitiamoci al caso in cui i due sistemi $S$ e $S'$ hanno assi paralleli e supponiamo che l'origine $O'$ di $S'$ abbia una velocità $\vec{v}=v\hat{x}$ nel sistema $S$. Mostriamo per prima cosa che l'omogeneità dello spazio implica la linearità delle trasformazioni. Infatti, uno spostamento infinitesimo viene trasformato secondo
\[\dif x_i'=\der{x_i'}{x_j}\dif x_j\]
Se lo spazio è omogeneo, tale spostamento non dipende dal punto in cui avviene, ossia tutte le derivate parziali nella relazione precedente sono costanti ovunque, e quindi il cambio di coordinate è lineare. Di conseguenza, avremo
\[\left(\begin{array}{c}
ct'\\x'\\y'\\z'
\end{array}\right)=\left(\begin{array}{c c c c}
\Lambda_{tt}&\Lambda_{tx}&\Lambda_{ty}&\Lambda_{tz}\\
\Lambda_{xt}&\Lambda_{xx}&\Lambda_{xy}&\Lambda_{xz}\\
\Lambda_{yt}&\Lambda_{yx}&\Lambda_{yy}&\Lambda_{yz}\\
\Lambda_{zt}&\Lambda_{zx}&\Lambda_{zy}&\Lambda_{zz}
\end{array}\right)\left(\begin{array}{c}
ct\\x\\y\\z
\end{array}\right)\]
dove tutti gli elementi della matrice sono funzioni continue e universali di $v$. Il fatto che gli assi dei due sistemi siano paralleli implica che una coordinata spaziale non possa dipendere dalle altre, dunque
\[
\Lambda_{xy}=\Lambda_{yx}=\Lambda_{xz}=\Lambda_{zx}=
\Lambda_{yz}=\Lambda_{zy}=0\]
Inoltre, per isotropia dello spazio una rotazione che porta $y$ in $z$ (e di conseguenza $y'$ in $z'$) non deve modificare la trasformazione, dato che il moto di $O'$ rimane invariato sotto tale rotazione. Questo implica dunque
\[
\Lambda_{ty}=\Lambda_{yt}=\Lambda_{tz}=\Lambda_{zt}=0\]
Queste semplici considerazioni ci hanno portato a una matrice di trasformazione a soli 6 parametri, la cui forma è
\[\Lambda=\left(\begin{array}{c c c c}
\Lambda_{tt}&\Lambda_{tx}&0&0\\
\Lambda_{xt}&\Lambda_{xx}&0&0\\
0&0&\Lambda_{yy}&0\\
0&0&0&\Lambda_{zz}
\end{array}\right)\]
Notiamo ora che l'isotropia ci assicura anche che un inversione degli assi $x,x'$ e un cambio di segno di $v$ lasciano la trasformazione invariata. Ciò significa che gli elementi di $\Lambda$ sulla diagonale sono pari in $v$ e quelli fuori dispari in $v$. Inoltre, è ragionevole supporre che per $v\to0$ tutti gli elementi sulla diagonale tendano a 1. Consideriamo ora il moto di $O'$. Le sue coordinate in $S$ e $S'$ sono rispettivamente
\[(x,y,z)=(vt,0,0)\]
\[(x',y',z')=(0,0,0)\]
Dunque si ha
\begin{equation}\label{condizione1}\Lambda_{xt}+\Lambda_{xx}\beta=0\end{equation}
dove, al solito, abbiamo introdotto $\beta=v/c$. Allo stesso modo, le coordinate dell'origine $O$ di $S$ nei due sistemi sono
\[(x,y,z)=(0,0,0)\]
\begin{equation}\label{o'}(x',y',z')=(\Lambda_{xt}ct'/\Lambda_{tt},0,0)=(v't',0,0)\end{equation}
Abbiamo introdotto all'ultimo passaggio la velocità $v'$ di $O$ rispetto a $O'$. L'isotropia ci assicura solamente che tale velocità è sull'asse $x'$, ma a priori non sappiamo nulla sul suo modulo. Mostriamo che, come ci aspettiamo intuitivamente, si ha $v'=-v$. Per chiarezza, sia $\Phi$ la mappa che manda $v$ in $v'$. Chiaramente $\Phi$ gode delle stesse proprietà dei coefficienti della trasformazione, in particolare è una funzione continua e universale di $v$. Inoltre, $\Phi$ è dispari e $\Phi=\Phi^{-1}$, dato che applicando $\Phi$ due volte si deve tornare alla velocità iniziale. Di conseguenza, $\Phi$ è biiettiva e continua, dunque monotona. A meno di considerare $-\Phi$, possiamo assumere $\Phi$ crescente. Ora se $\Phi(v)>v$, allora $v=\Phi(\Phi(v))>\Phi(v)>v$, assurdo. Analogamente si esclude il caso $\Phi(v)<v$, e quindi concludiamo
\[\Phi(v)=\pm v\]
Se fosse $\Phi(v)=v$, allora la (\ref{condizione1}) e la (\ref{o'}) implicano $\Lambda_{tt}=-\Lambda_{xx}$. Questo è assurdo perchè per $v\to0$ entrambi gli elementi devono tendere a 1, di conseguenza $\Phi(v)=-v$ come voluto. A questo punto è chiaro che la trasformazione inversa si deve ottenere cambiando il segno di $v$, cioè
\[\left(\begin{array}{c}
ct\\x\\y\\z
\end{array}\right)=\left(\begin{array}{c c c c}
\Lambda_{tt}&-\Lambda_{tx}&0&0\\
-\Lambda_{xt}&\Lambda_{xx}&0&0\\
0&0&\Lambda_{yy}&0\\
0&0&0&\Lambda_{zz}
\end{array}\right)\left(\begin{array}{c}
ct'\\x'\\y'\\z'
\end{array}\right)\]
Imponendo che tale matrice sia effettivamente l'inversa di quella di partenza otteniamo
\[\Lambda_{yy}=\Lambda_{zz}=1\]
Infine, posto $\Lambda_{xx}=\gamma$, si trova
\[\Lambda_{tt}=\gamma\]
\[\Lambda_{xt}=-\gamma\beta\]
\[\Lambda_{tx}=\frac{1-\gamma^2}{\gamma\beta}\]
Finora non abbiamo mai usato il fatto che $c$ sia costante. A questo punto basta imporre che un raggio luminoso la cui traiettoria in $S$ è
\[(x,y,z)=(ct,0,0)\]
venga mandato in
\[(x',y',z')=(ct',0,0)\]
Questo richiede infine
\[\gamma=\frac{1}{\sqrt{1-\beta^2}}\]
Alternativamente, prima di richiedere l'invarianza di $c$ si può chiedere che le trasformazioni di Lorentz formino un gruppo, con l'operazione data dall'usuale prodotto riga per colonna. Così facendo si ottiene in effetti
\[\gamma=\frac{1}{\sqrt{1-\alpha v^2}}\]
Se $\alpha>0$, allora $1/\sqrt{\alpha}$ è una velocità invariante, da cui $\alpha=c^{-2}$ per il primo postulato di Einstein. In linea di principio è possibile anche $\alpha=0$, e corrisponde alle trasformazioni di Galileo, e $\alpha<0$, ma in quest'ultimo caso sorgerebbe un problema assai fastidioso: non si potrebbe stabilire alcun nesso causale fra due eventi qualunque dello spazio-tempo.
\newpage
\section{5 marzo 2018}
\subsection{Contrazione delle lunghezze e dilatazione dei tempi}
Una misura di lunghezza va fatta confrontando due eventi simultanei: ad esempio, se si vuole misurare la lunghezza di un'asta di estremi $A$ e $B$ posta sull'asse $x$, si deve considerare la differenza $|x_A-x_B|$ quando $t_A=t_B$. 

\noindent Consideriamo ora un'asta sull'asse $x$ in moto con velocità $\vec{u}=\beta c\hat{x}$ rispetto al sistema $S$. Se $S'$ è il sistema solidale con l'asta, ovviamente gli estremi di quest'ultima hanno coordinate
\[A'=(ct'_A,0,0,0)\]
\[B'=(ct'_B,L',0,0)\]
dove $L'$ è la lunghezza dell'asta in $S'$. In questo sistema, una misura di lunghezza va fatta quando $t'_A=t'_B$. Spostandoci in $S$, le trasformazioni di Lorentz ci danno
\[A=(\gamma ct'_A,\gamma\beta ct'_A,0,0)\]
\[B=(\gamma(ct'_B+\beta L'),\gamma(L'+\beta ct'_B),0,0)\]
In questo sistema la misura va fatta quando $\gamma ct_A'=\gamma(ct'_B+\beta L')$.\footnote{Notiamo in particolare che se $t_A=t_B$ allora $t'_A\neq t'_B$, quindi si perde la simultaneità passando da $S$ a $S'$.} In tale istante abbiamo la lunghezza
\[L=x_B-x_A=\gamma(1-\beta^2)L'=L'/\gamma\]
Dunque le lunghezze parallele alla direzione del moto diminuiscono di un fattore $1/\gamma$. Se l'asta fosse invece parallela all'asse $y$, non avremmo contrazione, come si verifica facilmente con un calcolo del tutto analogo.

\noindent Consideriamo ora due sistemi inerziali, $S$ e $S'$, tale che l'origine $O'$ di $S'$ abbia velocità $\vec{v}=\beta c\hat{x}$ rispetto all'origine $O$ di $S$. Se due eventi accadono nello stesso punto in $S'$, a una distanza temporale di $\Delta\tau$, allora gli stessi eventi sono distanziati in $S$ di
\[\Delta t=\gamma\Delta\tau\]
Dunque gli intervalli temporali vengono dilatati.
\subsection{Trasformazione delle velocità}
Consideriamo i nostri due soliti sistemi inerziali $S$ e $S'$, con $S'$ in moto con velocità $\vec{u}=\beta c\hat{x}$ rispetto a $S$. Consideriamo un punto materiale che si muove con velocità $\vec{v}=(v_x,v_y,v_z)$ in $S$. La sua velocità in $S'$ è per definizione
\[\vec{v}'=\left(\frac{\dif x'}{\dif t'},\frac{\dif y'}{\dif t'},\frac{\dif z'}{\dif t'}\right)\]
Applicando le trasformazioni di Lorentz troviamo allora
\[v'_x=\frac{v_x-u}{1-v_xu/c^2}\]
\[v'_y=\frac{v_y}{\gamma(1-v_xu/c^2)}\]
\[v'_z=\frac{v_z}{\gamma(1-v_xu/c^2)}\]
A differenza di quanto accade con le trasformazioni di Galileo, le componenti trasverse della velocità non si conservano passando da $S$ a $S'$. Ciò non deve stupire, poiché cambiando sistema le coordinate $y$ e $z$ non cambiano, ma variano i tempi.
\subsection{Gli esperimenti di Bradley, Michelson-Morley e Fizeau rivisti}
Consideriamo un raggio luminoso che in $S$ si propaga lungo la direzione $-\hat{y}$. La sua velocità in $S'$ è 
\[v'_x=-u\]
\[v'_y=-c\sqrt{1-\frac{u^2}{c^2}}\]
Per piccoli $u$ si ottiene il risultato di Bradley. Notiamo che in ogni caso si ha
\[{v'_x}^2+{v'_y}^2=c^2\]
L'esperimento di Michelson e Morley ha un'ovvia spiegazione, dato che $c$ è invariante. Infine, per l'esperimento di Fizeau notiamo che la luce si propaga a una velocità $c/n$ nel sistema solidale con l'acqua, quindi nel sistema in cui l'acqua si muove la velocità di propagazione è
\[v_{\pm}=\frac{c/n\pm v}{1\pm v/nc}\]
Dunque per $v\ll c$ si ottiene
\[v_\pm\simeq\frac{c}{n}\pm v\left(1-\frac{1}{n^2}\right)\]
In accordo con quanto osservato da Fizeau.
\subsection{Intervallo invariante e diagrammi di Minkowski}
In questa sezione ci limitiamo a considerare moti unidimensionali. Consideriamo l'intervallo
\[s^2=c^2t^2-x^2\]
Abbiamo già visto che questa quantità è invariante per trasformazioni di Lorentz. In particolare, tutti gli osservatori inerziali concorderanno sul segno di $s^2$. Dato che le trasformazioni di Lorentz sono lineari, le stesse considerazioni valgono anche per le differenze di eventi. Per semplicità, supponiamo che uno dei due eventi sia $(0,0)$. Possiamo dunque distinguere tre tipi di intervallo:
\begin{itemize}
	\item se $c^2t^2-x^2>0$, l'intervallo viene detto di tipo tempo. Se $t>0$, allora per ogni osservatore si ha $t'>0$. Infatti, si ha
	\[ct'=\gamma(ct-\beta x)\geq ct-\beta x\geq ct-|\beta x|\geq ct-|x|>0\] 
	Questo significa che in ogni sistema l'evento che in $S$ ha coordinate $(ct,x)$ avviene dopo l'evento che in $S$ ha coordinate $(0,0)$. Possiamo quindi ben definire una successione temporale su cui tutti gli osservatori inerziali concordano. Per di più, è possibile trovare un sistema inerziale in cui i due eventi accadono nello stesso punto, ma a tempi diversi. Infatti, l'ascissa del punto è data in un altro sistema da
	\[x'=\gamma(x-\beta ct)\]
	A questo punto, se
	\[\beta=\frac{x}{ct}\]
	allora $x'=0$. Possiamo scegliere un tale $\beta$ perchè sicuramente $|x/ct|<1$. In maniera del tutto analoga, se $t<0$ allora per ogni osservatore si avrà $t'<0$;
	\item se $c^2t^2-x^2=0$, l'intervallo viene detto di tipo luce. Sappiamo già che questa condizione è rispettata dai raggi luminosi;
	\item infine, se $c^2t^2-x^2<0$, l'intervallo è detto di tipo spazio. In questo caso non è possibile definire un ordinamento temporale valido per tutti gli osservatori inerziali. In altre parole, se $t>0$ è possibile trovare un sistema inerziale in cui $t'<0$, e anche uno in cui $t'=0$, cioè in cui i due eventi sono simultanei. Infatti, se un sistema si muove con velocità
	\[\beta_c=\frac{ct}{x}\]
	allora $t'=0$. Chiaramente un tale sistema esiste, perchè $|\beta_c|<1$. Se invece il sistema si muove con velocità $|\beta|>|\beta_c|$, allora $t'<0$.
\end{itemize}
\noindent Un diagramma di Minkowski è un diagramma cartesiano in cui l'asse $x$ è l'asse orizzontale e l'asse $ct$ è l'asse verticale. Con questa scelta degli assi, un raggio luminoso si muove necessariamente su una delle due bisettrici. Il piano viene quindi diviso naturalmente in tre regioni:
\begin{itemize}
	\item la regione in cui $c^2t^2-x^2>0$. Questa regione forma il passato fisico (se $t<0$) e il futuro fisico (se $t>0$) dell'osservatore ed è costituito dai soli eventi con intervallo di tipo tempo;
	\item la regione in cui $c^2t^2-x^2=0$. Questa regione, detta cono luce, è la regione che può essere raggiunta dall'osservatore unicamente tramite raggi luminosi. Chiaramente è costituito dai soli eventi di tipo luce;
	\item la regione in cui $c^2t^2-x^2<0$. Questa regione è detta altrove ed è formato dagli eventi di tipo spazio.
\end{itemize}
La traiettoria di una particella in un diagramma di Minkowski è detta linea di universo. Chiaramente, tutta la linea di universo deve essere contenuta nel passato e nel futuro fisico e in ogni punto deve avere pendenza maggiore di 1 (in modulo). Ciò implica anche che il tempo ritardato di una particella è unico. Un possibile diagramma di Minkowski è dunque
\begin{figure}[h]
	\centering
	\scalebox{1}{\begin{tikzpicture}
		\draw [-stealth](-5,0)--(5,0)node[right]{$x$};
		\draw [-stealth](0,-4)--(0,4)node[right]{$ct$};
		\draw [dashed](-3.5,-3.5)--(3.5,3.5);
		\draw [dashed](-3.5,3.5)--(3.5,-3.5);
		\node at(-1.5,3.5){Futuro};
		\node at(1.5,-3.5){Passato};
		\node at(4,0.5){Altrove};
		\node at(-4,-0.5){Altrove};
		\draw [-stealth](-0.5,-3)..controls (-0.1,-2)..(-0.2,-1);
		\draw (-0.2,-1)..controls(-0.25,-0.5)..(0,0);
		\draw [-stealth](0,0)..controls(0.1,0.3)..(0.2,1);
		\draw (0.2,1)..controls(0.15,2)..(0.5,3);
		\end{tikzpicture}}
\end{figure}

\noindent Vogliamo ora rappresentare su un diagramma di Minkowski gli assi di un sistema $S'$ in moto a velocità $\beta c$ lungo $x$. L'asse $ct'$ corrisponde agli eventi in cui $x'=0$, ossia nel sistema $S$ corrisponde agli eventi
\[x=\beta ct\]
Ossia una retta che forma un angolo $\theta$ dato da $\tan\theta=\beta$ rispetto all'asse verticale. Analogamente, l'asse $x'$ corrisponde agli eventi $t'=0$, ossia alla retta
\[ct=\beta x\]
che forma un angolo $\theta$ con l'asse orizzontale.
\begin{figure}[h]
	\centering
	\scalebox{1}{\begin{tikzpicture}
		\draw [-stealth](-5,0)--(5,0)node[right]{$x$};
		\draw [-stealth](0,-4)--(0,4)node[right]{$ct$};
		\draw [dashed](-3.5,-3.5)--(3.5,3.5);
		\draw [dashed](-3.5,3.5)--(3.5,-3.5);
		\draw [-stealth](-1.4,-4)--(1.4,4)node[right]{$ct'$};
		\draw [-stealth](-4,-1.4)--(4,1.4)node[right]{$x'$};
		\draw (1,0) arc[start angle=0,end angle=39, x radius=0.5, y radius=0.5];
		\draw (0,1) arc[start angle=90,end angle=51, x radius=0.5, y radius=0.5];
		\draw node at(1.2,0.2){$\theta$};
		\draw node at(0.2,1.2){$\theta$};
		\end{tikzpicture}}
\end{figure}
Come ci aspettiamo, le linee luce di $S$ coincidono con le linee luce di $S'$. Inoltre, supponiamo di avere una sbarra in $S'$ di lunghezza unitaria (in $S'$) con un estremo coincidente con l'origine. Per misurarla in $S$, dobbiamo tracciare la retta pararella a $ct'$ e passante per il secondo estremo e misurare la distanza tra il punto in cui questa retta interseca l'asse $x$ e l'origine. D'altro canto, l'invarianza dell'intervallo ci dice che la lunghezza ottenuta va confrontata con la distanza dell'origine dall'intersezione dell'asse $x$ con l'iperbole $c^2t^2-x^2=-1$, che corrisponde a una sbarra di lunghezza unitaria in $S$. Si osserva allora che la sbarra appare contratta in $S$, come previsto. Analogamente, possiamo osservare "geometricamente" la dilatazione dei tempi proiettando un evento sull'asse $ct'$ sull'asse $ct$ e confrontando la lunghezza ottenuta con la distanza dell'origine dall'intersezione tra l'asse $ct$ e l'iperbole $c^2t^2-x^2=1$.
\begin{figure}[h]
	\centering
	\scalebox{1}{\begin{tikzpicture}
		\draw [-stealth](-5,0)--(5,0)node[right]{$x$};
		\draw [-stealth](0,-4)--(0,4)node[right]{$ct$};
		\draw [dashed](-3.5,-3.5)--(3.5,3.5);
		\draw [dashed](-3.5,3.5)--(3.5,-3.5);
		\draw [-stealth](-2.5,-4)--(2.5,4)node[right]{$ct'$};
		\draw [-stealth](-4,-2.5)--(4,2.5)node[right]{$x'$};
		\draw [smooth, samples=100,domain=1:3] plot(\x,{sqrt(\x*\x-1)});
		\draw [smooth, samples=100,domain=1:3] plot(\x,-{sqrt(\x*\x-1)});
		\draw [smooth, samples=200,domain=-3:3] plot(\x,{sqrt(\x*\x+1)});
		\draw [dotted](-2.5+0.78,-4)--(2.5+0.78,4);
		\draw [dotted](0,1.28)--(0.78,1.28);
		\node at(-3,0.5){\small Dilatazione dei tempi};
		\draw [-stealth](-1,0.6)--(-0.2,0.75);
		\node at(4,-1){\small Contrazione delle lunghezze};
		\draw [-stealth](1.8,-0.8)--(0.8,-0.15);
		\end{tikzpicture}}
\end{figure}
\newpage
\subsection{Quadrivettori}
Introduciamo la seguente notazione per le coordinate di una particella
\[x^\mu=\left(\begin{array}{c}
ct\\ \vec{x}
\end{array}\right)\]
$x^\mu$ è detto quadrivettore posizione. $ct$ è detta componente temporale e $\vec{x}$ è detta componente spaziale. L'indice $\mu$ può valere $0,\dots,3$ e per convezione la componente temporale ha indice $\mu=0$. In maniera analoga, diciamo che una quaterna
\[y^\mu=\left(\begin{array}{c}
y^0\\y^1\\y^2\\y^3
\end{array}\right)=\left(\begin{array}{c}
y^0\\ \vec{y}
\end{array}\right)\]
è un quadrivettore se trasforma come il quadrivettore posizione. In particolare, la componente spaziale $\vec{y}$ deve essere un vettore (cioè ruota come il vettore posizione $\vec{x}$). Inoltre, se applichiamo un boost lungo $\hat{x}$ con velocità $\beta c$ le componenti di $y^\mu$ devono trasformare come
\[{y'}\,^0=\gamma(y^0-\beta y^1)\]
\[{y'}\,^1=\gamma(y^1-\beta y^0)\]
\[{y'}\,^2=y^2\]
\[{y'}\,^3=y^3\]
Definiamo il tensore metrico $g_{\mu\nu}$ come
\[g_{\mu\nu}=\left(\begin{array}{c c c c}
1&0&0&0\\0&-1&0&0\\0&0&-1&0\\0&0&0&-1
\end{array}\right)\]
Se $y^\mu$ è un quadrivettore, poniamo
\[y_\mu=g_{\mu\nu}y^\nu\]
Esplicitamente, si ha
\[y_\mu=\left(\begin{array}{c}
y^0\\-\vec{y}
\end{array}\right)\]
Possiamo ora definire un prodotto interno tra quadrivettori nel seguente modo
\[x^\mu y_\mu=x^\mu g_{\mu\nu}y^\nu=x^0y^0-\vec{x}\cdot\vec{y}\]
Mostriamo che il prodotto interno è invariante sotto una qualunque trasformazione di Lorentz. Supponiamo che $\Lambda\indices{^\mu_\nu}$ sia la matrice associata a una certa trasformazione di Lorentz, ossia tale che
\[{x'}\,^\mu=\Lambda\indices{^\mu_\nu}x^\nu\]
Cerchiamo una forma generale per $\Lambda\indices{^\mu_\nu}$: sappiamo che per boost lungo $x$ e tali che gli assi dei due sistemi sono paralleli si ha
\[\Lambda\indices{^\mu_\nu}=\left(\begin{array}{c c c c}
\gamma&-\gamma\beta&0&0\\-\gamma\beta&\gamma&0&0\\0&0&1&0\\0&0&0&1
\end{array}\right)\]
Per tale trasformazione si mostra immediatamente che il prodotto è invariante. Se gli assi non sono paralleli e la velocità di $S'$ non è diretta lungo uno degli assi di $S$, possiamo ricondurci al boost elementare in questo modo: ruotiamo gli assi di $S$ fino a che il boost non sia lungo $x$, facciamo il boost e poi ruotiamo gli assi fino a che non coincidano con gli assi di $S'$. In altre parole, la forma generale di $\Lambda\indices{^\mu_\nu}$ è
\[\Lambda\indices{^\mu_\nu}=\left(\begin{array}{c c c c}
1&0&0&0\\0&&&\\0&&\mathcal{R}&\\0&&&
\end{array}\right)\left(\begin{array}{c c c c}
\gamma&-\gamma\beta&0&0\\-\gamma\beta&\gamma&0&0\\0&0&1&0\\0&0&0&1
\end{array}\right)\left(\begin{array}{c c c c}
1&0&0&0\\0&&&\\0&&\mathcal{R}'&\\0&&&
\end{array}\right)\]
dove $\mathcal{R},\mathcal{R}'$ sono opportune matrici di rotazione. A questo punto è sufficiente notare che $\vec{x}\cdot\vec{y}$ è invariante se ruotiamo gli assi (dato che le matrici di rotazione conservano la lunghezza dei vettori e gli angoli) e che le componenti temporali rimangono invariate sotto tali rotazioni.
\newpage
\section{12 marzo 2018}
\subsection{Quadrivelocità, quadriaccelerazione}
Definiamo la quadrivelocità come
\[u^\mu=\frac{\d x^\mu}{\d\tau}=\left(\gamma c,\gamma\vec{v}\right)\]
Per costruzione è un quadrivettore, di modulo invariante
\[u^\mu u_\mu=c^2\]
Analogamente, definiamo la quadriaccelerazione come
\[a^\mu=\frac{\d u^\mu}{d\tau}\left(\gamma c\frac{d\gamma}{\d t},\gamma \frac{d}{\d t}\gamma\vec{v}\right)\]
Se deriviamo $u^\mu u_\mu=c^2$ rispetto al tempo proprio, troviamo che la quadrivelocità e la quadriaccelerazione sono ortogonali, ovvero
\[a_\mu u^\mu=0\]
\subsection{Dinamica relativistica}
\newpage
\section{15 marzo 2018}
\subsection{Formulazione covariante dell'elettromagnetismo}
Un elemento di volume dello spaziotempo è dato da
\[\dif\Gamma=\dif x^0\dif x^1\dif x^2\dif x^3\]
Dato che una matrice di Lorentz ha determinante 1, l'elemento di volume si conserva. Inoltre, sappiamo che 
\[\partial_\mu=\left(\frac{1}{c}\der{}{t},\nabla\right)\]
trasforma come un quadrivettore controvariante. Da tale quadrivettore possiamo costruire il d'Alembertiano
\[\square=\partial_\mu\partial^\mu=\frac{1}{c^2}\der[2]{}{t}-\lap\]
Partiamo ora dall'evidenza sperimentale che la carica è un invariante di Lorentz. Ciò significa che se in un sistema $S$ misuriamo una densità di carica $\rho$ e in un sistema $S'$ una densità di carica $\rho'$, allora si ha
\[\rho\d x^1\d x^2\d x^3=\rho'\d x'^1\d x'^2\d x'^3\]
Consideriamo ora la quantità in $S$
\[c\rho\frac{\d x^\mu}{\d x^0}\]
Nel sistema $S'$ tale quantità sarà
\[c\rho'\frac{\d x'^\mu}{\d x'^0}=c\rho'\d x'^1\d x'^2\d x'^3\frac{\d x'^\mu}{\d x'^0\d x'^1\d x'^2\d x'^3}=c\rho\d\vec{x}\frac{\d x'^\mu}{\d x^0\d x^1\d x^2\d x^3}=c\rho\frac{\d x'^\mu}{\d x^0}=c\rho\Lambda\indices{^\mu_\nu}\frac{\d x^\nu}{\d x^0}\]
dove si è usata l'invarianza della carica e dell'elemento di volume. Allora la quantità introdotta trasforma come $\d x^\mu$, dunque è un quadrivettore, chiamato quadricorrente. Se lo indichiamo con $J^\mu$, le sue componenti sono
\[J^0=c\rho\]
\[J^i=\rho v^i\]
Ossia
\[J^\mu=(c\rho,\vec{J})\]
L'invariante associato alla quadricorrente è
\[J^\mu J_\mu=\rho^2(c^2-v^2)=\frac{c^2\rho^2}{\gamma^2}\]
Sappiamo ora che nella gauge di Lorenz le equazioni per i potenziali sono
\[\square V=-4\pi\rho\]
\[\square\vec{A}=-\frac{4\pi}{c}\vec{J}\]
e possono essere riscritte come
\[\square\left(\begin{array}{c}
V/c\\\vec{A}
\end{array}\right)=-\frac{4\pi}{c}J^\mu\]
Di conseguenza, possiamo introdurre il quadripotenziale
\[A^\mu=\left(\begin{array}{c}
V/c\\\vec{A}
\end{array}\right)\]
Le equazioni per i potenziali ci assicurano che è effettivamente un quadrivettore. La sua divergenza è
\[\partial_\mu A^\mu=\frac{1}{c}\der{V}{t}+\nabla\cdot\vec{A}=0\]
Quindi la condizione di gauge corrisponde alla richiesta di indivergenza del quadripotenziale.

\noindent Introduciamo ora il tensore elettromagnetico
\[F\indices{^{\mu\nu}}=\partial^\mu A^\nu-\partial^\nu A^\mu\]
Tale tensore è chiaramente antisimmetrico. Ricordiamo ora l'espressione dei campi in funzione dei potenziali
\[\vec{E}=-\nabla V-\frac{1}{c}\der{\vec{A}}{t}\]
\[\vec{B}=\nabla\times\vec{A}\]
Le componenti fuori diagonale del tensore elettromagnetico sono allora
\[F^{0i}=\partial^0A^i-\partial^iA^0=\partial^0A^i+\partial_iA^0=-E^i\]
\[F^{ij}=\partial^iA^j-\partial^jA^i=-(\partial_iA^j-\partial_jA^i)=-\varepsilon\indices{^{ijk}}B^k\]
Più esplicitamente, si ha
\[F^{\mu\nu}=\left(\begin{array}{c|c c c}
0&-E_x&-E_y&-E_z\\\hline E_x&0&-B_z&B_y\\E_y&B_z&0&-B_x\\E_z&-B_y&B_x&0
\end{array}\right)\]
Da questo possiamo costruire altri due tensori
\[F_{\mu\nu}\equiv g_{\mu\alpha}g_{\nu\beta}F^{\alpha\beta}=\left(\begin{array}{c|c c c}
0&E_x&E_y&E_z\\\hline -E_x&0&-B_z&B_y\\-E_y&B_z&0&-B_x\\-E_z&-B_y&B_x&0
\end{array}\right)\]
\[\mathcal{F}^{\mu\nu}\equiv\frac{1}{2}\varepsilon^{\mu\nu\alpha\beta}F_{\alpha\beta}=\left(\begin{array}{c|c c c}
0&-B_x&-B_y&-B_z\\\hline B_x&0&E_z&-E_y\\B_y&-E_z&0&E_x\\B_z&E_y&-E_x&0
\end{array}\right)\]
In particolare, $\mathcal{F}^{\mu\nu}$ è detto tensore duale di $F^{\mu\nu}$. Gli invarianti associati a questi tensori sono
\[F^{\mu\nu}F_{\mu\nu}=-\mathcal{F}^{\mu\nu}\mathcal{F}_{\mu\nu}=2\left(B^2-E^2\right)\]
\[\mathcal{F}^{\mu\nu}F_{\mu\nu}=-4\vec{E}\cdot\vec{B}\]
Osserviamo ora che si ha
\[\partial_\mu F^{\mu\nu}=\left(\nabla\cdot\vec{E},-\frac{1}{c}\der{\vec{E}}{t}+\nabla\times\vec{B}\right)\]
\[\partial_\mu \mathcal{F}^{\mu\nu}=\left(\nabla\cdot\vec{B},-\frac{1}{c}\der{\vec{B}}{t}-\nabla\times\vec{E}\right)\]
Le equazioni di Maxwell si scrivono allora come
\[\partial_\mu F^{\mu\nu}=\frac{4\pi}{c}J^\nu\]
\[\partial_\mu\mathcal{F}^{\mu\nu}=0\]
In particolare, sono già covarianti. Facendo la divergenza della prima equazione e ricordando l'antisimmetria del tensore elettromagnetico, abbiamo inoltre
\[0=\partial_\nu\partial_\mu F^{\mu\nu}=\frac{4\pi}{c}\partial_\nu J^\nu\]
Questa è l'equazione di continuità, infatti
\[\der{\rho}{t}+\nabla\cdot\vec{J}=\partial_\nu J^\nu=0\]
Infine, cerchiamo di trovare la forza di Lorentz. Per definizione, tale forza deve rispettare
\[f^\mu=\frac{\d p^\mu}{\d \tau}\]
In particolare, deve essere ortogonale alla quadrivelocità. Se $\lambda$ è uno scalare qualsiasi, una tale proprietà è rispettata dal quadrivettore
\[f^\mu=\lambda F^{\mu\nu}u_\nu\]
sempre a causa dell'antisimmetria del tensore $F^{\mu\nu}$. Vediamo come scegliere $\lambda$ in modo da ottenere l'espressione corretta. Le componenti di $f^\mu$ sono
\[f^\mu=\gamma\lambda(\vec{E}\cdot\vec{v},c\vec{E}+\vec{v}\times\vec{B})\]
Di conseguenza, poniamo $\lambda=q/c$ e ricaviamo l'espressione per la forza di Lorentz
\[\frac{\d p^\mu}{\d \tau}=\frac{q}{c}F^{\mu\nu}u_\nu\]
Per unità di volume, tale forza è
\[\mathscr{F}^\mu=\frac{1}{c}F^{\mu\nu}J_\nu\]
Consideriamo infine il tensore
\[T^{\mu\nu}=\frac{1}{4\pi}\left(F^{\mu\alpha}F\indices{^\nu_\alpha}-\frac{1}{4}g^{\mu\nu}F^{\alpha\beta}F_{\alpha\beta}\right)\]
Tale tensore è detto tensore energia-impulso. Le sue componenti sono
\[T^{\mu\nu}=\left(\begin{array}{c |c c c}
-u&&-\vec{S}/c&\\\hline&&&\\-\vec{S}/c&&T_{ij}&\\&&&
\end{array}\right)\]
dove $T_{ij}$ è il tensore degli sforzi di Maxwell. Il teorema di Poynting e conservazione dell'impulso si scrivono allora come
\[\partial_\mu T^{\mu\nu}+\mathscr{F}^\mu=0\]
\subsection{Trasformazioni relativistiche dei campi}
Dal tensore elettromagnetico è possibile trovare il modo in cui trasformano i campi nel passaggio da un sistema di riferimento inerziale a un altro. Infatti, si ha banalmente
\[F'^{\mu\nu}=\Lambda\indices{^\mu_\alpha}\Lambda\indices{^\nu_\beta}F^{\alpha\beta}\]
Esplicitamente, in un boost la componente parallela dei campi rimane invariata, mentre quella trasversa viene modificata come segue
\[\vec{E}'_\perp=\gamma(\vec{E}_\perp+\vec{\beta}\times\vec{B}_\perp)\]
\[\vec{B}'_\perp=\gamma(\vec{B}_\perp-\vec{\beta}\times\vec{E}_\perp)\]
Alternativamente, i campi sotto boost trasformano nel seguente modo
\[\vec{E}'=(\vec{E}\cdot\hat{\beta})\hat{\beta}+\gamma(\vec{E}-(\vec{E}\cdot\hat{\beta})\hat{\beta}+\vec{\beta}\times\vec{B})\]
\[\vec{B}'=(\vec{B}\cdot\hat{\beta})\hat{\beta}+\gamma(\vec{B}-(\vec{B}\cdot\hat{\beta})\hat{\beta}-\vec{\beta}\times\vec{E})\]
\newpage \noindent Con questa lezione si intende terminata la parte del corso per matematici. Il capitolo successivo, dedicato ai campi nei mezzi e all'interazione radiazione-materia è per fisici e chimici.
\chapter{Elettromagnetismo nella materia}
\section{9 aprile 2018}
\subsection{Scale tipiche}
Questa parte del corso sarà dedicata allo studio dell'interazione tra i campi elettromagnetici e la materia. La presenza di quest'ultima implica lo studio di sistemi a molti gradi di libertà, dunque è utile\footnote{Come d'altronde si fa usualmente in fisica statistica e termodinamica.} definire opportune scale di lunghezza dei fenomeni:
\begin{itemize}
	\item scala microscopica: questa scala è dell'ordine delle dimensioni atomiche, ossia pressapoco dell'ordine di $a_0\sim10^{-8}$ cm. A questo livello sono presenti tutti i dettagli dell'interazione, ma noi non ci interesseremo a questo approccio;
	\item scala di variazione dei campi: studieremo principalmente la luce visibile e al più l'ultravioletto, dunque una scala tipica è $\lambda\sim 10^{-5}$ cm;
	\item scala mesoscopica: questa scala è intermedia tra le due precedenti. In particolare, se $R$ è la sua lunghezza caratteristica, vogliamo che $R$ sia sufficientemente grande rispetto a $a_0$ in modo da poter introdurre delle opportune medie delle grandezze che studieremo. Ad esempio, non parleremo più del campo elettrico esattamente definito in un punto $\vec{r}$ dello spazio e a un istante $t$, ma piuttosto lo sostituiremo con la sua media
	\[\langle\vec{E}(\vec{r},t)\rangle=\frac{1}{R^3}\int_{[-R/2,R/2]^3}\vec{E}(\vec{r}+\vec{r}',t)\d^3r'\]
	su un cubo\footnote{La scelta di un volume di forma cubica è in realtà ininfluente.} di lato $R$. Rinunciamo quindi a una descrizione esatta dei fenomeni, che sarebbe d'altronde impossibile visto il numero di gradi di libertà del sistema in esame (dell'ordine del numero di Avogadro). Dobbiamo però anche richiedere che $R$ sia sufficientemente piccolo rispetto a $\lambda$, in maniera da poter comunque utilizzare gli strumenti del calcolo differenziale.
\end{itemize}
 A questo punto, ci si può chiedere se non si debba anche fare una media temporale delle grandezze di interesse. Questa media può essere evitata, poichè la media sulle coordinate spaziali elimina le fluttuazioni microscopiche dei campi (dato che $R$ è grande rispetto alla scala microscopica).
 \subsection{Equazioni di Maxwell nella materia}
 Sappiamo che nel vuoto valgono le equazioni di Maxwell
 \begin{align*}
 \nabla\cdot\vec{E}&=4\pi\rho\\\nabla\cdot\vec{B}&=0\\\nabla\times\vec{E}&=-\frac{1}{c}\der{\vec{B}}{t}\\\nabla\times\vec{B}&=\frac{4\pi}{c}\vec{J}+\frac{1}{c}\der{\vec{E}}{t}\end{align*}
 Dobbiamo ora mediare queste equazioni. Le omogenee non hanno particolari problemi, quindi avremo
 \begin{align*}
 	\nabla\cdot\langle\vec{B}\rangle&=0\\\nabla\times\langle\vec{E}\rangle&=-\frac{1}{c}\der{\langle\vec{B}\rangle}{t}
 \end{align*}
 Le equazioni non omogenee sono più problematiche. Infatti, chiaramente si deve avere
 \begin{align*}
 \nabla\cdot\langle\vec{E}\rangle&=4\pi\langle\rho\rangle\\\nabla\times\langle\vec{B}\rangle&=\frac{4\pi}{c}\langle\vec{J}\rangle+\frac{1}{c}\der{\langle\vec{E}\rangle}{t}\end{align*}
 Tuttavia, nelle medie $\langle\rho\rangle$ e $\langle\vec{J}\rangle$ si deve tener conto anche della risposta del mezzo. In altre parole, la materia può polarizzarsi e magnetizzarsi, contribuendo ai termini di sorgente nelle equazioni di Maxwell disomogenee. Trattiamo separatamente il caso elettrostatico e magnetostatico, per poi arrivare al caso elettrodinamico. Per brevità di notazione, da questo momento non indichiamo più le parentesi angolari nelle medie: se non detto esplicitamente, ogni grandezza si riferisce già al valore mediato.
 \subsection{Polarizzazione}
 Definiamo la polarizzazione $\vec{P}$ come il momento di dipolo elettrico medio per unità di volume, ossia
 \[\vec{P}=\frac{\d \vec{p}}{\d V}\]
 Osserviamo incidentalmente che per definizione $\vec{P}$ è una grandezza macroscopica. Se $\rho_{\textrm{ext}}$ è la densità di carica esterna, ossia la densità di carica che non è dovuta alla risposta del mezzo, il potenziale elettrostatico è semplicemente
 \[\varphi(\vec{r})=\int\frac{\rho_{\textrm{ext}}(\vec{r}')}{|\vec{r}-\vec{r}'|}\d^3r'+\int\frac{\vec{P}(\vec{r}')\cdot(\vec{r}-\vec{r}')}{|\vec{r}-\vec{r}'|^3}\d^3r'\]
 Interessiamoci al secondo termine: può essere scritto come
 \[\int\frac{\vec{P}(\vec{r}')\cdot(\vec{r}-\vec{r}')}{|\vec{r}-\vec{r}'|^3}\d^3r'=\int\vec{P}(\vec{r}')\cdot\nabla'\left(\frac{1}{|\vec{r}-\vec{r}'|}\right)\d^3r'=\int\left[\nabla'\cdot\frac{\vec{P}(\vec{r'})}{|\vec{r}-\vec{r}'|}-\frac{\nabla'\cdot\vec{P}(\vec{r}')}{|\vec{r}-\vec{r}'|}\right]\d^3r'\]
 Sotto le opportune ipotesi di localizzazione, il primo termine è nullo. Viceversa, alla superficie del mezzo può essere interpretato come una densità superficiale di carica di polarizzazione $\sigma_\textrm{p}=\vec{P}\cdot\hat{n}$, dove $\hat{n}$ è la normale uscente dal mezzo. Il secondo termine invece dà un contributo al potenziale che può essere associato a una densità di carica di polarizzazione $\rho_\textrm{p}=-\nabla\cdot\vec{P}$. Di conseguenza, la media della densità di carica che compare nella prima equazione di Maxwell è
 \[\rho=\rho_\textrm{ext}+\rho_\textrm{p}\]
 Ciò significa che la legge di Gauss si scrive nella forma
 \[\nabla\cdot\vec{E}=4\pi(\rho_{\textrm{ext}}+\rho_{\textrm{p}})=4\pi\rho_\textrm{ext}-4\pi\nabla\cdot\vec{P}\]
 Introduciamo il vettore spostamento elettrico $\vec{D}=\vec{E}+4\pi\vec{P}$. La legge di Gauss è allora
 \[\nabla\cdot\vec{D}=4\pi\rho_\textrm{ext}\]
 \subsection{Magnetostatica}
 In maniera del tutto analoga, definiamo la magnetizzazione $\vec{M}$ come il momento di dipolo magnetico medio per unità di volume, ossia
 \[\vec{M}=\frac{\d \vec{m}}{\d V}\]
 Anche $\vec{M}$ è per definizione una grandezza macroscopica. Inoltre, detta $\vec{J}_\textrm{ext}$ la corrente esterna, il potenziale vettore risultante sarà
 \[\vec{A}(\vec{r})=\frac{1}{c}\int\frac{\vec{J}_\textrm{ext}(\vec{r}')}{|\vec{r}-\vec{r}'|}\d^3r'+\int\frac{\vec{M}(\vec{r}')\times(\vec{r}-\vec{r}')}{|\vec{r}-\vec{r}'|^3}\d^3r'\]
 Ricordiamo ora l'identità vettoriale
 \[\nabla\times(a\vec{v})=\nabla a\times\vec{v}+a\nabla\times\vec{v}\]
 Il secondo termine può essere allora riscritto come
 \[\int\frac{\vec{M}(\vec{r}')\times(\vec{r}-\vec{r}')}{|\vec{r}-\vec{r}'|^3}\d^3r'=\int\vec{M}(\vec{r}')\times\nabla'\left(\frac{1}{|\vec{r}-\vec{r}'|}\right)\d^3r'=\int\left[-\nabla'\times\frac{\vec{M}(\vec{r}')}{|\vec{r}-\vec{r}'|}+\frac{\nabla'\times\vec{M}(\vec{r}')}{|\vec{r}-\vec{r}'|}\right]\d^3r'\]
 Il primo termine può essere trasformato in
 \[-\int\nabla'\times\frac{\vec{M}(\vec{r}')}{|\vec{r}-\vec{r}'|}\d^3r'=\int\frac{\vec{M}\times\hat{n}}{|\vec{r}-\vec{r}'|}\d S\]
 dove $\hat{n}$ è la normale uscente dalla superficie. Per mezzi localizzati è nullo, mentre sulla superficie del mezzo può essere interpretato come una densità superficiale di corrente di magnetizzazione $\vec{K}_\textrm{m}=c\vec{M}\times\hat{n}$. Il secondo termine può essere interpretato come una densità di corrente di magnetizzazione $\vec{J}_\textrm{m}=c\nabla\times\vec{M}$. La media della densità di corrente nel caso magnetostatico è quindi
 \[\vec{J}=\vec{J}_\textrm{ext}+\vec{J}_\textrm{m}\]
 Di conseguenza la legge di Ampère diventa
 \[\nabla\times\vec{B}=\frac{4\pi}{c}(\vec{J}_\textrm{ext}+\vec{J}_\textrm{m})=\frac{4\pi}{c}\vec{J}_\textrm{ext}+4\pi\nabla\times\vec{M}\]
 Introduciamo il campo $\vec{H}=\vec{B}-4\pi\vec{M}$. Per motivi storici, $\vec{H}$ si chiama campo magnetico, mentre $\vec{B}$ si dovrebbe chiamare induzione magnetica. Per semplicità, continueremo a chiamare $\vec{B}$ campo magnetico e ci riferiremo a $\vec{H}$ con l'altisonante appellativo di "acca".
 \subsection{Caso elettrodinamico}
 Passiamo ora alle equazioni di Maxwell complete. Come nel vuoto, abbiamo problemi solo per la legge di Ampère. Definiamo la densità di corrente di polarizzazione come
 \[\vec{J}_\textrm{p}=\der{\vec{P}}{t}\]
 Dalla definizione segue l'equazione di continuità per la carica polarizzazione
 \[\der{\rho_\textrm{p}}{t}+\nabla\cdot\vec{J}_\textrm{p}=0\]
 Vale anche l'equazione di continuità per le sorgenti esterne
 \[\der{\rho_\textrm{ext}}{t}+\nabla\cdot\vec{J}_\textrm{ext}=0\]
 e in più la corrente di magnetizzazione è indivergente, ossia
 \[\nabla\cdot\vec{J}_\textrm{m}=0\]
 D'altronde, deve valere anche l'equazione di continuità globale, da cui
 \[\der{}{t}(\rho_\textrm{ext}+\rho_\textrm{p})+\nabla\cdot\vec{J}=0\]
 dunque a meno di campi vettoriali indivergenti si ha
 \[\vec{J}=\vec{J}_\textrm{ext}+\vec{J}_\textrm{p}+\vec{J}_\textrm{m}\]
 Osserviamo che di per sè la corrente di magnetizzazione è indivergente, ma va sicuramente aggiunta a $\vec{J}$ affinchè la quarta equazione di Maxwell sia valida anche nel caso magnetostatico. In definitiva si ha
 \[\nabla\times\vec{H}=\frac{4\pi}{c}\vec{J}_\textrm{ext}+\frac{1}{c}\der{\vec{E}}{t}+\frac{4\pi}{c}\vec{J}_\textrm{p}=\frac{4\pi}{c}\vec{J}_\textrm{ext}+\frac{1}{c}\der{\vec{D}}{t}\]
 Un modo alternativo per ottenere l'equazione precedente è richiedere che si abbia
 \[\nabla\times\vec{H}=\frac{4\pi}{c}\vec{J}_\textrm{ext}+\frac{1}{c}\der{\vec{E}}{t}+\vec{\Lambda}\]
 dove $\vec\Lambda$ è un opportuno vettore tale che
 \[\nabla\cdot\left(\frac{4\pi}{c}\vec{J}_\textrm{ext}+\frac{1}{c}\der{\vec{E}}{t}+\vec{\Lambda}\right)=-\frac{4\pi}{c}\der{\rho_\textrm{ext}}{t}+\frac{4\pi}{c}\der{}{t}(\rho_\textrm{ext}+\rho_\textrm{p})+\nabla\cdot\vec{\Lambda}=0\]
 In conclusione, le equazioni di Maxwell nella materia (dette anche equazioni di Maxwell macroscopiche) sono
 \begin{align*}
 	\nabla\cdot\vec{D}&=4\pi\rho_\textrm{ext}\\\nabla\cdot\vec{B}&=0\\\nabla\times\vec{E}&=-\frac{1}{c}\der{\vec{B}}{t}\\\nabla\times\vec{H}&=\frac{4\pi}{c}\vec{J_\textrm{ext}}+\frac{1}{c}\der{\vec{D}}{t}
 \end{align*}
 Osserviamo che abbiamo introdotto due nuovi campi, $\vec{D}$ e $\vec{H}$, ma abbiamo sempre otto equazioni. Dobbiamo a questo punto specificare delle equazioni costitutive per il particolare mezzo in esame che descrivano nel dettaglio la materia, ossia che permettano di esprimere i due nuovi campi in funzione di $\vec{E}$ e $\vec{B}$. Queste equazioni possono in linea di principio essere complicate a piacere e dipendono fortemente non solo dal materiale in esame, ma anche dalla storia del campione che abbiamo effettivamente in mano (si pensi ad esempio all'isteresi di un ferromagnete). Noi studieremo solo i mezzi più semplici, limitandoci al caso in cui $\vec{D}$ dipenda unicamente da $\vec{E}$ e $\vec{H}$ unicamente da $\vec{B}$ (ignorando quindi un possibile accoppiamento magnetoelettrico) e, in genere, dando meno importanza alle proprietà magnetiche rispetto a quelle elettriche. Questa scelta è dettata dal fatto che, per il teorema di Bohr-van Leeuwen, una teoria soddisfacente del magnetismo deve obbligatoriamente chiamare in causa la meccanica quantistica; inoltre, alle frequenze a cui ci interesseremo gli effetti magnetici saranno usualmente trascurabili rispetto a quelli elettrici.
 \subsection{Condizioni di raccordo}
 Le equazioni di Maxwell nella materia implicano banalmente che le condizioni di raccordo per i campi su una superficie di separazione sono
 \begin{align*}
 	\Delta\vec{E}_\parallel&=0\\
 	\Delta\vec{D}_\perp&=4\pi\sigma_{\textrm{ext}}\\\Delta\vec{H}_\parallel &=\frac{4\pi}{c}\vec{K}_\textrm{ext}\times\hat{n}\\\Delta\vec{B}_\perp&=0
 \end{align*}
 Le condizioni di raccordo permettono anche di dare una definizione "operativa" di $\vec{D}$ e $\vec{H}$: ad esempio, per misurare $\vec{D}$ si può considerare un cilindro molto basso e largo di materiale e misurare il campo elettrico appena sopra la superficie (nel vuoto valgono chiaramente le equazioni di Maxwell microscopiche). Se sulla superficie non sono presenti cariche libere, allora il valore misurato corrisponde al valore di $\vec{D}$ nel mezzo. Per $\vec{H}$ si dà una definizione operativa analoga, ma usando per la misura un cilindro molto alto e stretto.
 \subsection{Regime lineare}
 Osserviamo preliminarmente che è sufficiente dare la polarizzazione $\vec{P}$ in funzione del campo elettrico $\vec{E}$ per trovare $\vec{D}$. Il caso più semplice è quello in cui la dipendenza è lineare, ossia quello in cui
 \[P_i(\vec{r},t)=\int\d t'\int\d^3r'g_{ij}(\vec{r},\vec{r}',t,t')E_j(\vec{r}',t')\]
 Le funzioni $g_{ij}$ codificano tutte le informazioni necessarie per studiare la risposta del mezzo. Possiamo ora fare diverse assunzioni e studiare vari casi. Per ora assumiamo che
 \begin{itemize}
 	\item il mezzo sia omogeneo nello spazio e nel tempo. Questo significa che si deve avere
 	\[g_{ij}=g_{ij}(\vec{r}-\vec{r}',t-t')\]
 	\item il mezzo sia isotropo. Allora il tensore di risposta deve essere multiplo dell'identità, ossia
 	\[g_{ij}=\delta_{ij}g(\vec{r}-\vec{r}',t-t')\]
 	dove $g$ è un'opportuna funzione. In tal caso, si ha
 	\[\vec{P}(\vec{r},t)=\int\d t'\int d^3r'g(\vec{r}-\vec{r}',t-t')\vec{E}(\vec{r}',t')\]
 	\item supponiamo inoltre \[g(\vec{r}-\vec{r}',t-t')=\delta(\vec{r}-\vec{r}')f(t-t')\]
 	Questa approssimazione è giustificata se il campo elettrico in un punto $\vec{r}$ influenza solo i punti estremamente vicini a $\vec{r}$. In un caso realistico, è probabile che $g$ abbia un picco più o meno ripido per $\vec{r}'=\vec{r}$. In tal caso, avremo
 	\[\vec{P}(\vec{r},t)=\int_{-\infty}^{t}\d t'f(t-t')\vec{E}(\vec{r},t')\]
 	Gli estremi di integrazione sono stati scelti in modo che sia rispettato il principio di causalità e in modo da tener conto di tutta la storia del campione.
 	\item infine, possiamo fare un'ipotesi analoga alla precedente, ma sulla coordinata temporale. In questo caso avremo
 	\[\vec{P}(\vec{r},t)=\chi\vec{E}(\vec{r},t)\]
 	\[\vec{D}(\vec{r},t)=\varepsilon\vec{E}(\vec{r},t)\]
 	La costante $\chi$ è detta suscettività elettrica del mezzo, mentre la costante $\varepsilon=1+4\pi\chi$ è detta permeabilità relativa del mezzo. Questa assunzione in genere è buona se la frequenza di oscillazione dei campi è molto minore della frequenza caratteristica di oscillazione del mezzo (che corrisponde grossomodo all'inverso del tempo di risposta del mezzo a perturbazioni esterne).
 \end{itemize}
\newpage
\section{12 aprile 2018}
\subsection{Energia elettrostatica}
Consideriamo un mezzo nel caso elettrostatico. Le equazioni di Maxwell si riducono a 
\begin{align*}
	\nabla\cdot\vec{D}&=4\pi\rho_\textrm{ext}\\
	\nabla\times\vec{E}&=0
\end{align*}
Si può ancora introdurre un potenziale elettrostatico $V$ per $\vec{E}$, ma in generale esso non soddisfa l'equazione di Poisson con termine di sorgente $-4\pi\rho_\ext$. Ciò accade solo a patto che $\varepsilon$ sia uniforme in tutto il volume di interesse, e in tal caso abbiamo le soluzioni tipiche del vuoto, se si effettua la sostituzione $\rho\to\rho_\ext/\varepsilon$. Torniamo ora al caso generale e supponiamo di avere un elettrodo metallico delimitato da una superficie $S$ su cui è distribuita una carica libera $q$. Supponiamo di aggiungere una piccola carica $\delta q$ su $S$ e di voler calcolare la variazione di energia elettrostatica. Notiamo che la carica libera al di fuori dell'elettrodo non varia, ma in generale l'aggiunta di $\delta q$ polarizza in maniera diversa il mezzo. La variazione di energia è
\[\delta U=V\delta q\]
dove $V$ è il potenziale sulla superficie dell'elettrodo. Osserviamo che per la prima equazione di Maxwell abbiamo
\[4\pi\delta q=\oint_S\delta\vec{D}\cdot\d\vec{S}\]
Di conseguenza, la variazione di energia elettrostatica è
\[\delta U=\frac{1}{4\pi}\oint_S V\delta\vec{D}\cdot\d\vec{S}\]
Sotto le opportune ipotesi di localizzazione, abbiamo
\[-\oint_S V\delta\vec{D}\cdot d\vec{S}=\int\nabla\cdot(V\delta\vec{D})\d^3x\]
dove il segno negativo è dovuto alla diversa orientazione del vettore normale e dove l'integrale a secondo membro è esteso a tutto il volume esterno all'elettrodo. Osserviamo ora che
\[\nabla\cdot\delta\vec{D}=0\]
nel volume esterno, proprio perchè $\delta q$ è stata aggiunta solo sul conduttore. Allora si ha 
\[\delta U=-\frac{1}{4\pi}\int\nabla \cdot(V\delta\vec{D})\d^3x=\frac{1}{4\pi}\int\vec{E}\cdot\delta\vec{D}\d^3x\]
In generale, non si può dire qualcosa di più. Supponiamo quindi che la risposta del mezzo sia lineare e istantanea. Allora si trova facilmente
\[\delta U=\frac{1}{4\pi}\int E_i\varepsilon_{ij}\delta E_j\d^3x=\delta\int\frac{E_iE_j\varepsilon_{ij}}{8\pi}\d^3x=\delta\int\frac{\vec{E}\cdot\vec{D}}{8\pi}\d^3x\]
dunque la densità di energia elettrostatica è
\[u=\frac{\vec{E}\cdot\vec{D}}{8\pi}\]
Dimostriamo ora che il tensore $\varepsilon_{ij}$ è simmetrico. Supponiamo di passare da una configurazione di carica in cui all'inizio $\vec{E}=0$ e alla fine $\vec{E}=E_x\hat{x}+E_y\hat{y}$. La conservatività di $\vec{E}$ implica che nel calcolo dell'energia possiamo indifferentemente variare prima la componente $\hat{x}$ del campo elettrico e poi la componente $\hat{y}$ o viceversa. In altre parole, si deve avere
\[\int_{(0,0,0)}^{(E_x,0,0)}\varepsilon_{ij}E_i\delta E_j+\int_{(E_x,0,0)}^{(E_x,E_y,0)}\varepsilon_{ij}E_i\delta E_j=\int_{(0,0,0)}^{(0,E_y,0)}\varepsilon_{ij}E_i\delta E_j+\int_{(0,E_y,0)}^{(E_x,E_y,0)}\varepsilon_{ij}E_i\delta E_j\]
Ma allora si ha
\[\underbrace{\frac{1}{2}\varepsilon_{xx}E_x^2}_{\textrm{Primo integrale}}+\underbrace{\varepsilon_{xy}E_xE_y+\frac{1}{2}\varepsilon_{yy}E_y^2}_{\textrm{Secondo integrale}}=\underbrace{\frac{1}{2}\varepsilon_{yy}E_y^2}_{\textrm{Primo integrale}}+\underbrace{\varepsilon_{yx}E_xE_y+\frac{1}{2}\varepsilon_{xx}E_x^2}_{\textrm{Secondo integrale}}\]
\[\varepsilon_{xy}=\varepsilon_{yx}\]
Si può anche procedere notando che un termine antisimmetrico di $\varepsilon_{ij}$ non dà contributi all'energia, quindi non ci sono motivi per introdurlo.

Facciamo ora un ragionamento diverso: supponiamo di avere un volume di interesse $V$ e una certa regione del volume di interesse $V_0$ in cui passiamo da una permittività $\varepsilon_a$ a una permittività $\varepsilon_b$. Negli altri punti sia $\varepsilon$ la permittività, e supponiamo di fissare la carica libera durante il processo. Se $\vec{E}_1$, $\vec{D}_1$ e $\vec{E}_2$, $\vec{D}_2$ sono rispettivamente i campi nella configurazione iniziale e finale (che in generale variano anche al di fuori di $V_0$), la variazione di energia elettrostatica è
\[\Delta U=\frac{1}{8\pi}\int_V(\vec{E}_2\cdot\vec{D}_2-\vec{E}_1\cdot\vec{D}_1
)\d^3x=\frac{1}{8\pi}\int_V[(\vec{D}_2-\vec{D}_1)\cdot(\vec{E}_2+\vec{E}_1)+\vec{D}_1\cdot\vec{E}_2-\vec{D}_2\cdot\vec{E}_1]\d^3x\]
Analizziamo il primo termine nell'integrale: la somma $\vec{E}_1+\vec{E}_2$ può essere scritta come $\nabla\varphi$, per un'opportuna $\varphi$. Ma allora
\[(\vec{D}_2-\vec{D}_1)\cdot(\vec{E}_1+\vec{E}_2)=\nabla\cdot[\varphi(\vec{D}_2-\vec{D}_1)]-\varphi\nabla\cdot(\vec{D}_2-\vec{D}_1)\]
Il primo termine non dà contributi quando viene integrato, sotto le solite opportune ipotesi di localizzazione. Il secondo è identicamente nullo, dato che la carica libera non è stata variata. Allora
\[\Delta U=\frac{1}{8\pi}\int_V(\varepsilon_1-\varepsilon_2)\vec{E}_1\cdot\vec{E}_2\d^3x=\frac{1}{8\pi}\int_{V_0}(\varepsilon_a-\varepsilon_b)\vec{E}_1\cdot\vec{E}_2\d^3x\]
Il secondo integrale è esteso a $V_0$, dato che altrimenti $\varepsilon_1-\varepsilon_2=0$. Di conseguenza, la variazione di energia elettrostatica è dovuta solo alle regioni in cui effettivamente vi è una variazione di permittività. Se ora $\varepsilon_a=1$, $\varepsilon_b=1+4\pi\chi$, si ha
\[\Delta U=-\frac{1}{2}\int_{V_0}\chi\vec{E}_1\cdot\vec{E}_2\d^3x=-\frac{1}{2}\vec{p}_2\cdot\vec{E_1}\]
dove $\vec{p}_2$ è il dipolo indotto dopo lo spostamento. Si noti che il campo $\vec{E}_1$ è quello presente prima dello spostamento. Nei casi pratici, se $\chi\simeq0$ è possibile o meno (a seconda della geometria del sistema) usare indifferentemente $\vec{E}_1$ o $\vec{E}_2$.
\subsection{Esercizi proposti}
\begin{problema}
	Si consideri un atomo alla Thomson costituito da una sfera di raggio $R$ in cui è distribuita uniformemente una carica $q$ e una carica puntiforme $-q$. Si supponga che la massa della sfera sia molto maggiore della massa della carica puntiforme. Calcolare il momento di dipolo indotto $\vec{p}$, quando l'atomo è immerso in un campo elettrico esterno uniforme e costante $\vec{E}_0$.
\end{problema}
\begin{soluzione}
	Il campo elettrico all'interno della sfera è banalmente
	\[\vec{E}=\frac{q}{R^3}\vec{r}\]
	L'ipotesi sulle masse ci permette di considerare la sfera positiva come fissa e di trascurare la deformazione dovuta al campo esterno. Di conseguenza, all'equilibrio l'elettrone si troverà in
	\[\vec{r}_0=-\frac{R^3\vec{E}_0}{q}\]
	che corrisponde a un dipolo indotto e a una polarizzabilità
	\begin{align*}\vec{p}&=R^3\vec{E}_0\\\alpha&=R^3\end{align*}
	A rigore, ciò si verifica se il campo esterno è sufficientemente piccolo. Infatti, se $\vec{E}_0$ è tale che $|\vec{r}_0|\geq R$, l'atomo si ionizza. 
\end{soluzione}
\begin{problema}
	Si consideri un materiale composto da due tipi di strati paralleli, di spessore e permittività rispettivamente $l_1, \varepsilon_1$ e $l_2,\varepsilon_2$, alternati. Studiare le proprietà dielettriche del mezzo a scale grandi rispetto a $l_1+l_2$.
\end{problema}
\begin{soluzione}
	Siano $\vec{E}_1,\vec{D}_1$ e $\vec{E}_2,\vec{D}_2$ il campo elettrico e lo spostamento elettrico nei due strati. A grande scala la media macroscopica è
	\begin{align*}
	\vec{E}&=\frac{l_1\vec{E}_1+l_2\vec{E}_2}{l_1+l_2}\\
	\vec{D}&=\frac{l_1\vec{D}_1+l_2\vec{D}_2}{l_1+l_2}
	\end{align*}
	Introduciamo ora una terna cartesiana con l'asse $x$ ortogonale all'interfaccia tra gli strati. Le condizioni di raccordo sui campi sono
	\begin{align*}D_{1,x}&=D_{2,x}\\
	E_{1,y}&=E_{2,y}\\
	E_{1,z}&=E_{2,z}\end{align*}
	Da ciò si trova facilmente
	\begin{align*}
	D_x&=\frac{l_1+l_2}{l_1\varepsilon_1^{-1}+\varepsilon_2^{-1}}E_x\\
	D_y&=\frac{l_1\varepsilon_1+l_2\varepsilon_2}{l_1+l_2}E_y\\
	D_z&=\frac{l_1\varepsilon_1+l_2\varepsilon_2}{l_1+l_2}E_z
	\end{align*}
	Il tensore di permittività è dunque diagonale, con un blocco $2\times 2$ proporzionale all'identità. Un materiale con tale permittività è detto per ovvi motivi uniassiale, ed è il più semplice esempio di materiale non isotropo. Osserviamo incidentalmente che
	\begin{align*}
	\varepsilon_{xx}&=\langle\varepsilon^{-1}\rangle^{-1}\\
	\varepsilon_{yy}&=\langle\varepsilon\rangle\\
	\varepsilon_{zz}&=\langle\varepsilon\rangle
	\end{align*}
\end{soluzione}
\newpage
\section{16 aprile 2018}
\subsection{Esercizi proposti}
\begin{problema}
	Si consideri una molecola fissa nello spazio formata da due atomi alla Thomson uguali, mantenuti a distanza fissa $D$. Calcolare la polarizzabilità della molecola.
\end{problema}
\begin{soluzione}
	Sia $\alpha_0=R^3$ la polarizzabilità del singolo atomo, quando l'altro non è presente. Scegliamo l'asse $z$ coincidente con l'asse della molecola e gli assi $x$ e $y$ nel piano ortogonale a tale asse. Il tensore di polarizzabilità sarà allora
	\[\alpha_{ij}=\left(\begin{array}{c c c}
	\alpha_{1}&0&0\\0&\alpha_{1}&0\\0&0&\alpha_{2}
	\end{array}\right)\]
	ovvero, il sistema è uniassiale. Ci aspettiamo che se $D\gg R$ si abbia $\alpha_1=\alpha_2=2\alpha_0$, dato che in tal caso i dipoli indotti non interagiscono significativamente. Se invece $D\approx R$, ci aspettiamo una qualche interazione tra i due dipoli.

	\noindent Mettiamo ora un campo esterno $\vec{E}_0=E_0\hat{z}$ per calcolare $\alpha_2$. I dipoli indotti saranno lungo $z$ e saranno uguali: se avessero modulo diverso (ad esempio, il dipolo "di testa" di modulo maggiore), potremmo fare una simmetria rispetto al centro della molecola, che scambierebbe i ruoli dei dipoli. Ora si trova facilmente
	\begin{align*}p&=\alpha_0\left(E_0+2\frac{p}{D^3}\right)\\\alpha_2&=\frac{2\alpha_0}{1-\frac{2\alpha_0}{D^3}}\end{align*}
	Il 2 a denominatore è dovuto al fatto che abbiamo due dipoli. Osserviamo quindi che il campo esterno è rafforzato dai due dipoli. Analogamente, supponiamo di mettere un campo esterno $\vec{E}_0=E_0\hat{x}$. In tal caso, ci aspettiamo dipoli allineati con $\vec{E}_0$ (se fossero più inclinati verso l'interno, un'inversione di $\vec{E}_0$ non lascerebbe il risultato invariato). Allora si ha
	\begin{align*}
		p&=\alpha_0(E_0-\frac{p}{D^3})\\\alpha_1&=\frac{2\alpha_0}{1+\frac{\alpha_0}{D^3}}
	\end{align*}
	Qui i dipoli indotti si indeboliscono a vicenda. Le espressioni trovate hanno il corretto limite per $D\gg R$, inoltre $\alpha_2$ diverge per $D^3=2\alpha_0$. Questo effetto non è sempre un problema del modello, ma suggerisce che possano esistere dei materiali in cui si ha una polarizzazione residua dovuta a effetti autoindotti anche per campi esterni nulli: tali materiali, per analogia con i magneti, sono detti materiali ferroelettrici. In altri casi, delle divergenze nelle funzioni di risposta possono suggerire la presenza di transizione di fase.
\end{soluzione}
\begin{problema}
	Si consideri una sfera di raggio $a$ e permittività $\varepsilon$ immersa in un campo esterno uniforme e costante $\vec{E}_0$. Calcolare il campo elettrico in tutto lo spazio.
\end{problema}
\begin{soluzione}
	Consideriamo un sistema di coordinate sferiche centrate nella sfera, con l'asse azimutale parallelo a $\vec{E}_0$. Per simmetria, il piano ortogonale a tale asse passante per il centro della sfera (ossia il piano $\theta=\pi/2$) è equipotenziale. Poniamo $V=0$ su tale piano. Inoltre, si mostra facilmente che un'eventuale carica si trova sulla superficie della sfera. Dunque, se $V_{in}$ e $V_{out}$ sono i potenziali all'interno e all'esterno della sfera, tali potenziali sono armonici nelle rispettive regioni. Allora si ha
	\begin{align*}V_{in}(r,\theta)&=\sum_{l=1}^{\infty}A_lr^lP_l(\cos\theta)\\ V_{out}(r,\theta)&=\sum_{l=1}^{\infty}\left(A'_lr^l+\frac{B_l}{r^{l+1}}\right)P_l(\cos\theta)\end{align*}
	La somma è fatta da $l=1$ dato che i termini con $l=0$ sono potenziali costanti, dunque nulli per la condizione sul piano $\theta=\pi/2$. A grande distanza la perturbazione dovuta alla sfera si annulla, ovvero
	\[V_{out}(r\gg a,\theta)=-E_0r\cos\theta\]
	Da cui $A'_1=-E_0$ e $A'_l=0$ se $l\neq 1$. Le altre condizioni di raccordo sono
	\begin{align*}
		\varepsilon\left.\frac{\partial V_{in}}{\partial r}\right|_a&=\left.\frac{\partial V_{out}}{\partial r}\right|_a\\
		\left.\frac{\partial V_{in}}{\partial \theta}\right|_a&=\left.\frac{\partial V_{out}}{\partial \theta}\right|_a
	\end{align*}
	La seconda condizione può essere sostituita dalla richiesta $V_{in}(a,\theta)=V_{out}(a,\theta)$. Infatti, la differenza $V_{in}(a,\theta)-V_{out}(a,\theta)$ è costante dalla condizione sulla componente tangenziale di $\vec{E}$, ed è nulla per $\theta=\pi/2$. Allora si ottiene
	\begin{align*}
		A_1&=-E_0\frac{3}{\varepsilon+2}\\B_1&=a^3E_0\frac{\varepsilon-1}{\varepsilon+2}\\A_l&=0\textrm{ se }l\neq 1\\B_l&=0\textrm{ se }l\neq 1
	\end{align*}
	ovvero
	\begin{align*}
		\vec{E}_{in}&=\frac{3}{\varepsilon+2}\vec{E}_0\\\vec{E}_{out}&=\vec{E}_0+\frac{3(\vec{p}\cdot\hat{r})\hat{r}-\vec{p}}{r^3}
	\end{align*}
	dove si è posto
	\[\vec{p}=a^3\vec{E}_0\frac{\varepsilon-1}{\varepsilon+2}\]
	Il campo interno può essere scritto come
	\[\vec{E}_{in}=\vec{E}_0+\vec{E}_P\]
	dove $\vec{E}_P$ è il campo dovuto alla polarizzazione $\vec{P}$. Esplicitamente si ha
	\begin{align*}
		\vec{P}&=\frac{3\vec{p}}{4\pi a^3}=\\&=\frac{3\vec{E}_0}{4\pi}\frac{\varepsilon-1}{\varepsilon+2}\\
		\vec{E}_P&=-\frac{4\pi}{3}\vec{P}
	\end{align*}
	Il segno negativo è dovuto al fatto che $\vec{P}$ è parallela a $\vec{E}_0$ e tende a indebolire quest'ultimo. Inoltre, se $\varepsilon\to\infty$ otteniamo i risultati noti per una sfera conduttrice.
\end{soluzione}
\subsection{Effetti di campo locale e relazione di Clausius-Mossotti}
Usiamo ora i due esercizi precedenti per studiare un reticolo cubico di passo $a$ formato da atomi di polarizzabilità $\alpha$. Il dipolo indotto di ogni atomo è
\[\vec{p}=\alpha\vec{E}_{loc}\]
$\vec{E}_{loc}$ è il campo locale, ossia il campo sull'atomo. In generale, non coincide con il campo di Maxwell macroscopico $\vec{E}$, a causa dell'interazione tra dipoli indotti. In altre parole, si ha
\[\vec{E}_{loc}=\vec{E}+\vec{E}_d-\vec{E}_P\]
Il campo $\vec{E}_d$ è il campo generato da tutti i dipoli indotti, mentre il campo $\vec{E}_P$ è il campo dovuto alla polarizzazione macroscopica. Questo viene sottratto perchè è già inserito all'interno di $\vec{E}$. Calcoliamo ora i due contributi, per semplicità sull'atomo posto nell'origine: ci aspettiamo deviazioni significative dovute ai soli atomi più vicini nel reticolo, quindi fissiamo una sfera mesoscopica di raggio $R$ che sia grande rispetto alla scala microscopica $a$, ma piccola rispetto al scala di variazione di $\vec{E}$. In tal caso si ha semplicemente
\[\vec{E}_P=-\frac{4\pi}{3}\vec{P}\]
Per il calcolo di $\vec{E}_d$ dobbiamo sommare i campi dei vari dipoli, posti in $\vec{r}_{ijk}=ai\hat{x}+aj\hat{y}+ak\hat{z}$. Calcoliamo solo una componente di tale campo, ossia quella lungo $x$:
\begin{align*}
	E_{d,x}&=\sum_{\substack{i,j,k\\i^2+j^2+k^2\leq R^2/a^2}}\frac{3ai(aid_x+ajd_y+akd_z)-a^2(i^2+j^2+k^2)d_x}{a^5(i^2+j^2+k^2)^{5/2}}=\\&=\frac{3}{a^3}\sum_{\substack{i,j,k\\i^2+j^2+k^2\leq R^2/a^2}}\frac{d_yij+d_zik}{(i^2+j^2+k^2)^{5/2}}+\frac{d_x}{a^3}\sum_{\substack{i,j,k\\i^2+j^2+k^2\leq R^2/a^2}}\frac{2i^2-j^2-k^2}{(i^2+j^2+k^2)^{5/2}}=\\&=0
\end{align*}
Infatti, la prima somma è nulla per parità. La seconda è nulla perchè per simmetria si deve avere
\[\sum_{\substack{i,j,k\\i^2+j^2+k^2\leq R^2/a^2}}\frac{i^2}{(i^2+j^2+k^2)^{5/2}}=\sum_{\substack{i,j,k\\i^2+j^2+k^2\leq R^2/a^2}}\frac{j^2}{(i^2+j^2+k^2)^{5/2}}=\sum_{\substack{i,j,k\\i^2+j^2+k^2\leq R^2/a^2}}\frac{k^2}{(i^2+j^2+k^2)^{5/2}}\]
A questo punto, se $n$ è il numero di atomi per unità di volume si ha
\begin{align*}\vec{P}&=n\vec{p}=\\&=n\alpha\left(\vec{E}+\frac{4\pi}{3}\vec{P}\right)\\\vec{P}&=\frac{n\alpha}{1-\frac{4\pi n\alpha}{3}}\vec{E}\end{align*}
La permittività è allora
\begin{align*}
	\varepsilon&=1+\frac{4\pi n\alpha}{1-\frac{4\pi n\alpha}{3}}
\end{align*}
Abbiamo quindi una relazione tra una grandezza macroscopica, $\varepsilon$, e una microscopica, $\alpha$. In particolare, abbiamo
\[n\alpha=\frac{3}{4\pi}\frac{\varepsilon-1}{\varepsilon+2}\]
Tale relazione è detta relazione di Clausius-Mossotti. In particolare, per basse densità (ad esempio, un gas) la permittività è lineare con $n$. All'aumentare della densità compaiono effetti non lineari dovuti all'interazione interatomica.
\newpage
\section{19 aprile 2018}
\subsection{Polarizzazione di un mezzo lineare per campi variabili}
Consideriamo un mezzo la cui risposta è data da
\[\vec{P}(t)=\int_{-\infty}^{+\infty}g(t-t')\vec{E}(t')\d t'\]
dove conveniamo di porre $g(t)\equiv 0$ per $t<0$, per rispettare il principio di causalità. Osserviamo che possiamo anche scrivere
\[\vec{P}(t)=g*\vec{E}(t)\]
Se indichiamo con $\tilde{\chi}$ la trasformata di Fourier di $g$, otteniamo
\[\hat{\vec{P}}(\omega)=\tilde{\chi}(\omega)\hat{\vec{E}}(\omega)\]
In generale, $\tilde{\chi}$ sarà una funzione a valori complessi, ovvero
\[\tilde{\chi}(\omega)=\chi_1(\omega)+i\chi_2(\omega)\]
Osserviamo però che $g$ è reale, dunque $\tilde\chi(\omega)=\tilde{\chi}(-\omega)^*$. Ciò implica che $\chi_1$ sia una funzione pari e $\chi_2$ una funzione dispari. Cerchiamo ora di interpretare la parte reale e la parte immaginaria di $\tilde{\chi}$. Per fare ciò, consideriamo un campo elettrico della forma
\[\vec{E}(t)=2\vec{E}_0\cos\omega t\]
La polarizzazione associata a tale campo è
\begin{align*}\vec{P}(t)&=\vec{E}_0\textrm{Re}(\tilde{\chi}(\omega)e^{-i\omega t}+\tilde{\chi}(-\omega)e^{i\omega t})=\\&=2\vec{E}_0(\chi_1(\omega)\cos\omega t+\chi_2(\omega)\sin\omega t)\end{align*}
Osserviamo che $\chi_1$ è associata a una componente della polarizzazione in fase con il campo elettrico e, come vedremo, sarà in genere associata a fenomeni dispersivi. Al contrario, $\chi_2$ è associata a una componente della polarizzazione in controfase con il campo elettrico e sarà associata a fenomeni dissipativi.
\subsection{Modello microscopico per la suscettività}
Consideriamo un sistema di oscillatori armonici indipendenti identici costituiti da un corpo puntiforme di massa $m$ e carica $q$ legato all'origine da un potenziale armonico di costante elastica $m\omega_0^2$. Se facciamo incidere un'onda piana di frequenza $\omega$ su uno di questi oscillatori, allora l'equazione del moto è
\[m\ddot{\vec{r}}=-m\omega_0^2\vec{r}-m\gamma\dot{\vec{r}}+q\vec{E}_0e^{-i\omega t}\]
dove $\gamma$ è un opportuno coefficiente di smorzamento\footnote{Ad esempio, può modellizzare la perdita di energia per irraggiamento.}. La soluzione a regime è ovviamente una soluzione oscillante a frequenza $\omega$, ossia
\[\vec{r}=\frac{q\vec{E}_0}{m}\frac{e^{-i\omega t}}{\omega_0^2-\omega^2-i\gamma\omega}\]
Osserviamo come prima cosa che se $\vec{E}_0$ è reale, in generale $\vec{r}e^{i\omega t}$ non lo è: ciò significa che la risposta non sarà in generale in fase con la forzante. Dalla soluzione del moto possiamo ricavare la polarizzazione del sistema, una volta che conosciamo la densità di oscillatori $n$:
\[\vec{P}=nq\vec{r}=\frac{nq^2}{m}\frac{1}{\omega_0^2-\omega^2-i\gamma\omega}\vec{E}\]
Ciò implica che
\begin{align*}\tilde{\chi}(\omega)&=\frac{nq^2}{m}\frac{1}{\omega_0^2-\omega^2-i\gamma\omega}\\\chi_1(\omega)&=\frac{nq^2}{m}\frac{\omega_0^2-\omega^2}{(\omega_0^2-\omega^2)^2+\gamma^2\omega^2}\\\chi_2(\omega)&=\frac{nq^2}{m}\frac{\gamma\omega}{(\omega_0^2-\omega^2)^2+\gamma^2\omega^2}\end{align*}
Come previsto, $\chi_1$ è pari e $\chi_2$ è dispari. Inoltre, possiamo distinguere diversi regimi:
\begin{itemize}
	\item Se $\omega_0\neq 0$ e $\omega\ll\omega_0$ si ha
	\begin{align*}
		\chi_1(\omega)&\simeq\frac{nq^2}{m\omega_0^2}\\\chi_2(\omega)&\simeq 0
	\end{align*}
	\item In un metallo, in cui $\omega_0=0$, a basse frequenze si ha
	\begin{align*}
		\chi_1(\omega)&\simeq-\frac{nq^2}{m\gamma^2}\\\chi_2(\omega)&\simeq\frac{nq^2}{m\gamma\omega}
	\end{align*}
	\item Ad alte frequenze, ossia se $\omega\gg\omega_0$, si ha
	\begin{align*}
	\chi_1(\omega)&-\simeq\frac{nq^2}{m\omega^2}\\\chi_2(\omega)&\simeq \frac{nq^2\gamma}{m\omega^3}
	\end{align*}
	\item Infine, se siamo vicini alla risonanza, ossia per $\omega\simeq\omega_0$, si ha
	\begin{align*}
		\chi_1(\omega)&\simeq\frac{nq^2}{2m\omega_0}\frac{\omega_0-\omega}{(\omega_0-\omega)^2+\gamma^2/4}\\\chi_2(\omega)&\simeq\frac{nq^2}{2m\omega_0}\frac{\gamma/2}{(\omega_0-\omega)^2+\gamma^2/4}
	\end{align*}
\end{itemize}
Nei pressi della risonanza, $\chi_2$ è ben approssimata da una lorentziana di ampiezza $\gamma$ e altezza $1/\gamma$. Un grafico di $\chi_1$ e $\chi_2$ è riportato nella figura seguente\newpage
\begin{figure}[h]
	\centering
	\scalebox{1.5}{\begin{tikzpicture}
		\draw [-stealth](0,0)--(8,0)node[below]{\tiny $\omega$};
		\draw [-stealth](0,-1)--(0,2.5);
		\draw [smooth, samples=100,domain=0:7.8] plot(\x,{(16-\x^2)/((16-\x^2)^2+0.15^2*\x^2)});
		\draw [smooth, samples=100,domain=0:7.8] plot(\x,{(0.15*\x)/((16-\x^2)^2+0.15^2*\x^2)});
		\node at(3,0.5){\tiny $\chi_1$};
		\node at(4.5,1){\tiny $\chi_2$};
		\node at(3.8,-0.2){\tiny $\omega_0$};
		\draw [dashed](4,0)--(4,1.45);
		\end{tikzpicture}}
\end{figure}
\noindent Se $\gamma=0$, otteniamo una funzione discontinua per $\chi_1$ e una $\delta$ per $\chi_2$, ovvero
\begin{figure}[h]
	\centering
	\scalebox{1.5}{\begin{tikzpicture}
		\draw [-stealth](0,0)--(8,0)node[below]{\tiny $\omega$};
		\draw [-stealth](0,-1)--(0,2.5);
		\draw [smooth, samples=100,domain=0:3.95] plot(\x,{(16-\x^2)/((16-\x^2)^2});
		\draw [smooth, samples=100,domain=4.12:7.8] plot(\x,{(16-\x^2)/((16-\x^2)^2});
		\draw [-stealth](4,0)--(4,1);
		\node at(3,0.5){\tiny $\chi_1$};
		\node at(4.25,0.75){\tiny $\chi_2$};
		\node at(3.8,-0.2){\tiny $\omega_0$};
		\draw [dashed](4,-1)--(4,2.5);
		\end{tikzpicture}}
\end{figure}

\noindent La permittività del mezzo è data da
\begin{align*}\tilde{\varepsilon}(\omega)&=1+4\pi\tilde{\chi}(\omega)=\\&=1+\frac{\omega_p^2}{\omega_0^2-\omega^2-i\gamma\omega}\end{align*}
dove abbiamo introdotto la frequenza di plasma
\[\omega_p^2=\frac{4\pi q^2n}{m}\]
Da $\tilde{\varepsilon}$ possiamo considerare $\varepsilon_1$ e $\varepsilon_2$ in maniera del tutto analoga a quanto fatto con la suscettività. Il modello che abbiamo costruito, puramente classico, è in realtà un buon modello confermato anche dalla meccanica quantistica (a patto di ridefinire opportunamente alcune quantità). Distinguiamo ora, in modo del tutto arbitrario, tre regioni:
\begin{itemize}
	\item Regione I: questa regione corrisponde al limite quasi statico, ossia al limite in cui $\varepsilon_1$ è pressochè indipendente da $\omega$ e $\varepsilon_2\simeq0$. Il fatto che $\varepsilon_1$ sia costante implica che la risposta del mezzo è istantanea, mentre il fatto che $\varepsilon_2\simeq0$ implica che il mezzo è trasparente ai campi (dato che non c'è dissipazione). In questa regione studieremo anche qualche proprietà magnetica dei mezzi.
	\item Regione II: in questa regione possiamo ancora trascurare la dissipazione (ovvero $\varepsilon_2\simeq 0$), ma non possiamo trascurare la dipendenza di $\varepsilon_1$ dalla frequenza. Il mezzo è ancora trasparente, ma è presente dispersione; nonostante ciò, $\vec{D}$ è ancora in fase con $\vec{E}$.
	\item Regione III: corrisponde alla regione in cui non possiamo più trascurare la dissipazione. A volte, questa regione è chiamata regione di dispersione anomala, dato che $\varepsilon_1$ non è monotona con la frequenza (come invece accade nelle altre regioni).
\end{itemize}
\subsection{Regione I}
Per quanto detto, in tale regione abbiamo
\[\vec{D}(\vec{r},t)=\varepsilon\vec{E}(\vec{r},t)\]
ovvero una risposta istantanea nel tempo e locale nello spazio. Analogamente, per i campi magnetici avremo
\[\vec{B}(\vec{r},t)=\mu\vec{H}(\vec{r},t)\]
Consideriamo ora un mezzo uniforme senza sorgenti esterne. Allora, in maniera del tutto analoga a quanto fatto nel vuoto\footnote{Si noti che $\nabla \cdot  {D}=\varepsilon\nabla\cdot\vec{E}$ e $\nabla\times\vec{B}=\mu\nabla\times\vec{H}$ per un mezzo omogeneo.} si mostra che i campi soddisfano l'equazione d'onda
\begin{align*}
	\lap\vec{E}-\frac{\mu\varepsilon}{c^2}\der[2]{\vec{E}}{t}&=0\\
	\lap\vec{B}-\frac{\mu\varepsilon}{c^2}\der[2]{\vec{B}}{t}&=0\\
\end{align*}
La velocità di propagazione è ridotta rispetto a quella nel vuoto di un fattore
\[n=\sqrt{\mu\varepsilon}\]
detto indice di rifrazione del mezzo. Se ora consideriamo un'onda piana della forma
\begin{align*}
	\vec{E}&=\vec{E}_0e^{i\vec{k}\cdot\vec{r}-i\omega t}\\
	\vec{B}&=\vec{B}_0e^{i\vec{k}\cdot\vec{r}-i\omega t}
\end{align*}
otteniamo
\begin{align*}k^2&=n^2\frac{\omega^2}{c^2}\\\vec{B}_0&=n\hat{k}\times\vec{E}_0\end{align*}
A differenza di quanto accade nel vuoto, i campi non hanno più la stessa ampiezza, anche se continuano a formare una terna ortogonale insieme al vettore d'onda $\vec{k}$. Inoltre, la lunghezza d'onda è ridotta di un fattore $n$ rispetto al vuoto.
\subsection{Vettore di Poynting nella regione I}
Cerchiamo ora una versione del teorema di Poynting in questa regione. Dobbiamo avere
\[\der{u}{t}=-\nabla\cdot\vec{S}-\vec{J}_\ext\cdot\vec{E}\]
Si noti che è stata usata $\vec{J}_\ext$: infatti, siamo interessati agli scambi energetici con le sorgenti libere, dato che quelle di polarizzazione e di magnetizzazioni non sono facilmente governabili da un sistema esterno. Lavorando con le equazioni di Maxwell otteniamo
\begin{align*}
	\vec{J}_\ext\cdot\vec{E}&=\frac{c}{4\pi}\vec{E}\cdot\nabla\times\vec{H}-\frac{1}{4\pi}\der{\vec{D}}{t}\cdot\vec{E}=\\&=-\frac{c}{4\pi}\nabla\cdot\vec{E}\times\vec{H}+\frac{c}{4\pi}\vec{H}\times\vec{E}-\frac{1}{4\pi}\der{\vec{D}}{t}\cdot\vec{E}=\\&=-\frac{c}{4\pi}\nabla\cdot\vec{E}\times\vec{H}-\frac{1}{4\pi}\der{\vec{B}}{t}\cdot\vec{H}-\frac{1}{4\pi}\der{\vec{D}}{t}\cdot\vec{E}=\\&=-\frac{c}{4\pi}\nabla\cdot\vec{E}\times\vec{H}-\frac{1}{8\pi}\der{}{t}\left(\vec{H}\cdot\vec{B}+\vec{D}\cdot\vec{E}\right)
\end{align*}
L'ultimo passaggio è dovuto al fatto che la risposta è locale e istantanea, dunque
\begin{align*}
	\der{\vec{D}}{t}&=\varepsilon\der{\vec{E}}{t}\\
	\der{\vec{B}}{t}&=\mu\der{\vec{H}}{t}
\end{align*}
Deduciamo quindi che in questa regione si ha
\begin{align*}
	\vec{S}&=\frac{c}{4\pi}\vec{E}\times\vec{H}\\u&=\frac{\vec{E}\cdot\vec{D}+\vec{B}\cdot\vec{H}}{8\pi}
\end{align*}
Nelle altre regione l'espressione di $u$ cambierà, mentre $\vec{S}$ sarà sempre lo stesso. Facciamo un esempio che ci permetta di capire almeno intuitivamente perchè ciò accade: consideriamo un mezzo privo di cariche esterne e facciamo incidere su tale mezzo un'onda piana, ad incidenza normale. Fuori dal mezzo sappiamo che il vettore di Poynting è
\[\vec{S}=\frac{c}{4\pi}\vec{E}\times\vec{B}=\frac{c}{4\pi}\vec{E}\times\vec{H}\]
Subito sotto l'interfaccia, l'analogo del vettore di Poynting non può cambiare per la conservazione dell'energia, anche se sono presenti effetti dissipativi. Infatti, la dissipazione è proporzionale al volume del mezzo, quindi immediatamente sotto l'interfaccia non abbiamo variazioni. Inoltre, l'analogo del vettore di Poynting dovrà essere bilineare nei campi. Per quanto detto, dobbiamo combinare opportunamente dei campi le cui componenti ortogonali alla direzione di propagazione (che sono quelle che contribuiscono a Poynting, dato che questo ha la stessa direzione del vettore d'onda) siano continue all'interfaccia: questi campi sono proprio $\vec{E}$ e $\vec{H}$.

\noindent Consideriamo ora un'onda che si propaghi lungo l'asse $z$, con ampiezza del campo elettrico $E_0$. Abbiamo
\begin{align*}
\langle u\rangle&=\frac{E_0^2\varepsilon}{8\pi}\\
\langle S_z\rangle&=\frac{cE_0^2n}{8\pi\mu}=\frac{c}{n}\langle u\rangle
\end{align*}
\newpage
\section{23 aprile 2018}
\subsection{Esercizi proposti}
\begin{problema}
	Si consideri un gas perfetto composto da dipoli permanenti $\vec{p}$ alla temperatura $T$. Se il sistema è posto in un campo esterno $\vec{E}_0$ uniforme, calcolare la polarizzabilità del gas.
\end{problema}
\begin{soluzione}
	Dato che i dipoli non sono interagenti, l'energia di un dipolo è
	\[U=-\vec{E}_0\cdot\vec{p}=E_0p\cos\theta\]
	dove si è assunto $\vec{E}_0=E_0\hat{z}$. Il momento di dipolo medio deve essere orientato lungo $\hat{z}$ per simmetria, dunque 
	\begin{align*}\langle p_z\rangle&=\frac{\int_{0}^{\pi}p\cos\theta \exp\left(E_0p\cos\theta/kT\right)\sin\theta\d\theta}{\int_{0}^{\pi} \exp\left(E_0p\cos\theta/kT\right)\sin\theta\d\theta}=\\&=p\frac{\int_{-1}^{1}x\exp(\alpha x)\d x}{\int_{-1}^{1}\exp(\alpha x)\d x}\end{align*}
	dove $\alpha=E_0p/kT$. Essendo interessati al regime lineare, vorremmo espandere l'esponenziale al primo ordine. Questa approssimazione è in genere buona, ad esempio per l'aria deve sicuramente essere $E_0\leq 30$ kV/cm, mentre una stima grossonala del momento di dipolo è $p\sim e\cdot 1$ \AA, dunque a temperatura ambiente
	\[\frac{30\mathrm{ kV/cm}\cdot 10^{-8}\mathrm{ cm}\cdot e}{\frac{1}{40}\mathrm{ eV}}\sim10^{-2}\]
	Allora otteniamo
	\begin{align*}\langle p_z\rangle&=p\frac{\int_{-1}^{1}x(1+\alpha x)\d x}{\int_{-1}^{1}(1+\alpha x)\d x}=\\&=\frac{E_0p^2}{3kT}\end{align*}
	Se indichiamo con $n$ il numero di dipoli per unità di volume, la polarizzazione e la polarizzabilità sono
	\begin{align*}
		\vec{P}&=\frac{np^2}{3kT}\vec{E}_0\\\chi&=\frac{np^2}{3kT}
	\end{align*}
\end{soluzione}
\subsection{Regione II}
In questa regione non abbiamo più una dipendenza della forma $\vec{D}(t)=\varepsilon\vec{E}(t)$, anzi siamo obbligati a lavorare in trasformata di Fourier. Dato che la dissipazione è comunque trascurare, abbiamo
\[\hat{\vec{D}}(\omega)=\varepsilon(\omega)\hat{\vec{E}}(\omega)\]
dove $\varepsilon(\omega)$ è reale. Analogamente, per i campi magnetici abbiamo
\[\hat{\vec{B}}(\omega)=\mu(\omega)\hat{\vec{H}}(\omega)\]
Per brevità, omettiamo i cappucci in quanto segue. Le equazioni di Maxwell macroscopiche in assenza di sorgente si scrivono allora come
\begin{align*}
	\nabla\cdot\vec{D}&=0\\\nabla\cdot\vec{B}&=0\\\nabla\times\vec{E}&=\frac{i\omega}{c}\vec{B}\\\nabla\times\vec{H}&=-\frac{i\omega}{c}\vec{D}
\end{align*}
Osserviamo che se il mezzo è omogeneo le prime due equazioni sono implicate dalle ultime due. Lavorando su queste in maniera analoga a quanto fatto nel vuoto si trovano le equazioni di Helmoltz
\begin{align*}
	\lap\vec{E}-\frac{\omega^2\varepsilon(\omega)\mu(\omega)}{c^2}\vec{E}&=0\\
	\lap\vec{B}-\frac{\omega^2\varepsilon(\omega)\mu(\omega)}{c^2}\vec{B}&=0\\
\end{align*}
Come nella regione I, possiamo introdurre l'indice di rifrazione $n=\sqrt{\varepsilon\mu}$, da cui si ottiene la velocità di fase delle onde elettromagnetiche
\[v_f=\frac{c}{n}\]
La grande differenza rispetto alla regione I è la relazione di dispersione non banale: cercando soluzioni della forma di onde piane, ossia $\vec{E}=\vec{E}_0\exp(i\vec{k}\cdot\vec{r}-i\omega t)$, si trova
\[k^2=\frac{\omega^2n^2}{c^2}\]
Come nella regione I, si ha ancora per l'ampiezza delle onde piane
\[|\vec{B}_0|=n(\omega)|\vec{E}_0|\]
La presenza di fenomeni dispersivi apre le porte a un vasto argomento: limitiamoci a trovare le velocità di fase e di gruppo.
\subsection{Velocità di un pacchetto}
Consideriamo un pacchetto d'onda (per semplicità, unidimensionale) dato dalla sovrapposizione di un certo numero di onde piane
\[f(x,t)=\int_{k_0-\Delta k}^{k_0+\Delta k}A(k)e^{ikx-i\omega(k)t}\d k\]
Dato che nel seguito saremo interessati al trasporto di energia, limitiamoci a calcolare $|f(x,t)|^2$. Poniamo $q=k-k_0$ e espandiamo al primo ordine la relazione di dispersione
\[\omega(k)=\omega(k_0)+\left.\der{\omega}{k}\right|_{k=k_0}q\]
Si ha
\begin{align*}
	|f(x,t)|^2&=\left|\int_{-\Delta k}^{\Delta k}A(k_0+q)\exp\left(ik_0x+iqx-i\omega(k_0)t-\left.\der{\omega}{k}\right|_{k=k_0}iqt\right)\d q\right|^2\\&=\left|e^{ik_0x.i\omega(k_0)t}\int_{-\Delta k}^{\Delta k}A(k_0+q)\exp\left(iqx-\left.\der{\omega}{k}\right|_{k=k_0}iqt\right)\d q\right|^2\\&=\left|g\left(x-\left.\der{\omega}{k}\right|_{k=k_0}t\right)\right|^2
\end{align*}
dove $g$ è un'opportuna funzione \footnote{Esplicitamente, è $f(x,0)$.}. A questo punto è chiaro che l'inviluppo del pacchetto si propaga alla velocità
\[v_g=\der{\omega}{k}\]
detta, appunto, velocità di gruppo. Se la relazione di dispersione è non banale, $v_g$ è in generale diversa dalla velocità di fase
\[v_f=\frac{\omega}{k}\]
Inoltre, l'approssimazione che abbiamo fatto è valida solo per piccoli tempi: al passare del tempo diventano importanti anche i termini di ordine superiore in $\omega(k)$, che portano anche a una distorsione del pacchetto (che, in genere, tende ad allargarsi e ad appiattirsi).
\subsection{Vettore di Poynting}
Come già detto, il vettore di Poynting è
\[\vec{S}=\frac{c}{4\pi}\vec{E}\times\vec{H}\]
Cerchiamo ora di trovare la densità di energia. Non daremo una derivazione rigorosa, ma piuttosto cercheremo di trovare una forma decente per $u$ per poi controllare che funziona. Sappiamo che nella regione I si ha
\[\langle S\rangle=v_g\langle u\rangle \]
Ci aspettiamo che ciò continui a valere anche nella regione II. Verifichiamo che l'espressione
\[u=\frac{1}{8\pi}\left(\der{(\omega\varepsilon(\omega))}{\omega}E^2+\der{(\omega\mu(\omega))}{\omega}H^2\right)\]
ha la proprietà richiesta. Osserviamo preliminarmente che può essere riscritta nella forma
\[u=\frac{1}{8\pi\omega\mu}\der{(\omega^2\mu\varepsilon)}{\omega}E^2\]
In tal modo si ottiene
\begin{align*}
	\langle u\rangle&=\frac{1}{16\pi\omega\mu}\der{(\omega^2\mu\varepsilon)}{\omega}E_0^2=\\&=\frac{c^2}{16\pi\omega\mu}\der{k^2}{\omega}E_0^2=\\&=\frac{c^2E_0^2k}{8\pi\omega\mu}\der{k}{\omega}=\\&=\frac{cE_0^2}{8\pi v_g}\sqrt{\frac{\varepsilon}{\mu}}=\\&=\frac{\langle S\rangle}{v_g}
\end{align*}
La formula per $u$ è detta formula di Brillouin.
\newpage
\section{26 aprile 2018}
\subsection{Regione III}
In questa regione non consideriamo gli effetti magnetici, ossia poniamo $\vec{H}=\vec{B}$. Per molti materiali è una buona approssimazione. Per quanto riguarda la permittività, in questa regione è complessa. La linearità delle equazioni di Maxwell ci dà ancora l'equazione di Helmoltz
\[\lap\vec{E}+\tilde{\varepsilon}(\omega)\frac{\omega^2}{c^2}\vec{E}=0\]
Se però cerchiamo soluzioni della forma di onde piane, troviamo che il vettore d'onda soddisfa
\[\vec{k}\cdot\vec{k}=\tilde\varepsilon(\omega)\frac{\omega^2}{c^2}\]
cioè sarà, in generale, complesso. Anche l'indice di rifrazione, definito da
\[\tilde{n}^2(\omega)=\tilde{\varepsilon}(\omega)\]
è complesso. Se scriviamo $\vec{k}=\vec{k}_1+i\vec{k}_2$, si ha
\begin{align*}
	|\vec{k}_1|^2-|\vec{k}_2|^2&=\varepsilon_1(\omega)\frac{\omega^2}{c^2}\\2\vec{k}_1\cdot\vec{k}_2&=\varepsilon_2(\omega)\frac{\omega^2}{c^2}
\end{align*}
In particolare, se $\varepsilon_2\neq0$ allora $\vec{k}_2\neq0$ (ma ovviamente l'implicazione opposta è falsa). Mostriamo ora che, come atteso, $\varepsilon_2$ è associato alla dissipazione. Un'onda piana è della forma
\[\vec{E}=\vec{E}_0e^{-\vec{k}_2\cdot\vec{r}}e^{i\vec{k}_1\cdot\vec{r}-i\omega t}\]
Ossia un'onda piana "pura" modulata con un inviluppo che decresce esponenzialmente. In generale, $\vec{k}_1$ e $\vec{k}_2$ non sono paralleli: in questo caso, si parla di onde inomogenee. Come nel vuoto, i piani ortogonali a $\vec{k}_1$ sono i piani con stessa fase. I piani ortogonali a $\vec{k}_2$ sono invece i piani con ampiezza dei campi costante. Consideriamo ora il caso $\vec{k}=(k_1+ik_2)\hat{z}$. Un'onda di questo tipo è detta onda omogenea. Assumiamo inoltre che l'onda sia della forma
\[\vec{E}=E_0\hat{y}e^{-k_2z}e^{ik_1z-i\omega t}\]
con $E_0$ reale. La media del vettore di Poynting è allora
\begin{align*}
	\langle S_z\rangle&=\frac{c}{8\pi}\Re(E_xB_y)=\\&=\frac{cE_x^2}{8\pi}\Re\tilde{n}=\\&=\frac{cE_0^2n_1}{8\pi}e^{-2n_2\omega z/c}
\end{align*}
Il coefficiente
\[\alpha=2n_2\frac{\omega}{c}\]
è detto coefficiente di assorbimento. Consideriamo ora un cilindro con asse parallelo a $\hat{z}$, di area di base $A$ e altezza $\Delta z$. Assumiamo che una base sia a $z=0$ La potenza assorbita per unità di volume è, per la conservazione dell'energia
\begin{align*}
	W&=\frac{S(0)-S(\Delta z)}{\Delta z}=\\&=\frac{cE_0^2n_1}{8\pi}\frac{1-e^{-\alpha\Delta z
	}}{\Delta z}
\end{align*}
Se ora consideriamo il caso $\alpha\Delta z\ll1$, si ottiene
\begin{align*}
	W&\simeq\frac{cE_0^2n_1}{8\pi}\alpha=\\&=\frac{\omega E_0^2}{8\pi}2n_1n_2=\\&=\frac{\omega E_0^2}{8\pi}\varepsilon_2
\end{align*}
Come previsto, la dissipazione è proporzionale a $\varepsilon_2$. Se il mezzo ha proprietà magnetiche, la dissipazione dovuta a tali proprietà è chiaramente
\[W_m=\frac{\omega H_0^2}{8\pi}\mu_2\]
Vediamo un altro modo per ricavare il risultato precedente. Sappiamo che nel vuoto la dissipazione è
\[W=\langle\vec{J}\cdot\vec{E}\rangle\]
In questo caso, $\vec{J}=\vec{J}_\textrm{pol}$, perchè stiamo considerando un mezzo senza sorgenti esterne e senza proprietà magnetiche. Allora si ha
\begin{align*}
	W&=\langle\vec{J}_\textrm{pol}\cdot\vec{E}\rangle=\\&=\langle-i\omega\vec{P}\cdot\vec{E}\rangle=\\&=\langle-i\omega\frac{\tilde{\varepsilon}(\omega)-1}{4\pi}E^2\rangle=\\&=\frac{\omega E^2}{8\pi}\varepsilon_2
\end{align*}
Finora non abbiamo considerato mezzi conduttivi. Se $\vec{J}=\sigma\vec{E}$, la quarta equazione di Maxwell si scrive come
\[\nabla\times\vec{E}=-\frac{i\omega}{c}\left(\tilde{\varepsilon}+i\frac{4\pi\sigma}{\omega}\right)\vec{E}\]
Nulla ci impedisce di scegliere una conduttività complessa, quindi può diventare un po' arbitrario decidere quali termini vadano inclusi nella permittività e quali nella conducibilità. Inoltre, la differenza tra un conduttore e un non conduttore si manifesta solo a basse frequenze, dato che se $\sigma\neq0$ allora $\varepsilon+4\pi i\sigma/\omega$ ha un polo per $\omega=0$. In altre parole, un campo statico in un materiale non conduttore polarizza semplicemente il mezzo (dato che gli elettroni sono legati), mentre in un conduttore il campo elettrico riesce a mantenere delle correnti (dato che gli elettroni non sono legati).
\subsection{Considerazioni energetiche}
Nella regione III non è in generale possibile definire una velocità di gruppo e una densità di energia. Più esplicitamente, se abbiamo un modello (ad esempio, Drude-Lorentz) possiamo definire $v_g$ e $u$, ma non possiamo esprimerle in funzione dei campi macroscopici: modelli differenti portano a velocità di gruppo e densità di energia differenti. Nella maggior parte dei casi è sufficiente considerare $\vec{S}$ e $W$.
\subsection{Riflessione e rifrazione}
Consideriamo un'interfaccia tra due mezzi: per fissare le idee, diciamo che per $z<0$ abbiamo un mezzo trasparente con indice di rifrazione $n_0$ (in alcuni casi sceglieremo il vuoto, ma la generalizzazione ai mezzi trasparenti è immediata) e per $z>0$ abbiamo un mezzo con permittività $\tilde{\varepsilon}$, in generale complessa. Facciamo incidere un'onda elettromagnetica omogenea sull'interfaccia con vettore d'onda $\vec{k}_i$ contenuto nel piano $xz$ e frequenza $\omega$. Ci aspettiamo sia un'onda riflessa che un'onda rifratta e sappiamo che dobbiamo imporre la continuità di $E_\parallel$, $H_\parallel$, $D_\perp$ e $B_\perp$ all'interfaccia. Vedremo che le prime due condizioni sono sufficienti a risolvere il problema. Per prima cosa, l'omogeneità del tempo implica che l'onda riflessa e l'onda rifratta abbiano entrambe frequenza $\omega$. Inoltre, l'invarianza per traslazioni lungo $x$ e $y$ implica che 
\[\vec{k}_{i,\parallel}=\vec{k}_{r,\parallel}=\vec{k}_{t,\parallel}\]
con il pedice $r$ che sta riflessa e $t$ per trasmessa. Avendo supposto che $\vec{k}_i$ sia contenuto nel piano $xz$, otteniamo
\[k_{i,x}=k_{r,x}=k_{t,x}\]
\[k_{r,y}=k_{r,z}=0\]
Ciò significa che i tre vettori d'onda formano un piano, detto piano di incidenza. Notiamo che se i mezzi non sono isotropi, in generale non il piano di incidenza non sarà ben definito. Usiamo ora le relazioni di dispersione nei due mezzi:
\begin{align*}
	k_{i,x}^2+k_{i,z}^2&=n_0^2\frac{\omega^2}{c^2}\\k_{r,x}^2+k_{r,z}^2&=n_0^2\frac{\omega^2}{c^2}\\k_{t,x}^2+k_{t,z}^2&=\tilde{\varepsilon}\frac{\omega^2}{c^2}
\end{align*}
se ne deduce $k_{r,z}=\pm k_{i,z}$. Dato che l'onda riflessa è, appunto, riflessa, scegliamo $k_{r,z}=-k_{i,z}$. Ciò significa che l'angolo $\theta_i$ formato da $\vec{k}_i$ con la normale all'interfaccia è uguale all'angolo $\theta_r$ formato da $\vec{k}_r$ con la stessa, ossia la prima legge della riflessione
\[\theta_i=\theta_r\]
La terza relazione di dispersione ci dà invece
\[k_{t,z}=\pm\sqrt{\tilde{\varepsilon}\frac{\omega^2}{c^2}-n_0^2k_{i,x}^2}\]
Dato che $\tilde{\varepsilon}$ è complesso, anche $\vec{k}_t$ lo sarà, ovvero l'onda rifratta è in generale disomogenea. Scrivendo $\vec{k}_t=\vec{k}_1+i\vec{k}_2$, Abbiamo $\vec{k}_2=\hat{z}\Im k_{t,z}$. Per questo motivo scegliamo la radice con $\Im k_{t,z}>0$. La direzione di $\vec{k}_2$ è ovvia, altrimenti l'onda rifratta avrebbe ampiezza non costante lungo l'interfaccia, e ciò è assurdo per l'invarianza sotto traslazioni.

\noindent Limitiamoci ora a un mezzo con permittività reale (indicata con $\varepsilon$). In questo caso abbiamo
\[k_{t,z}=\frac{\omega}{c}\sqrt{\varepsilon-\varepsilon_0\sin^2\theta_i}\]
con $\varepsilon_0=n_0^2$. Se $\varepsilon_0\leq\varepsilon$ allora $k_{t,z}$ è reale per ogni angolo di incidenza. Se invece $\varepsilon_0>\varepsilon$, esiste un angolo limite $\theta_L$ definito da
\[\sin\theta_L=\sqrt{\frac{\varepsilon}{\varepsilon_0}}=\frac{n}{n_0}\]
Per $\theta_i\geq\theta_L$, $k_{t,z}$ è puramente immaginario, ossia si ha
\[\vec{k}_t=k_i\left(\hat{x}\cos\theta_i+i\hat{z}\sqrt{\frac{\varepsilon_0}{\varepsilon}\sin^2\theta_i-1}\right)\]
che corrisponde a un'onda disomogenea che si propaga lungo $\hat{x}$ e con ampiezza esponenzialmente decrescente lungo $\hat{z}$. Nonostante ciò, $\vec{k}_{t,1}\cdot\vec{k}_{t,2}=0$, come deve essere dato che la permittività è reale (e dunque non c'è dissipazione). Tutta l'energia viene quindi riflessa dall'interfaccia e si parla di riflessione totale. Se al contrario $\theta_i<\theta_L$, $\vec{k}_t$ forma con la normale all'interfaccia un angolo $\theta_r$ dato da
\begin{align*}
	k_{t,x}&=n\frac{\omega}{c}\sin\theta_r=\\=k_{i,x}&=n_0\frac{\omega}{c}\sin\theta_i
\end{align*}
da cui otteniamo la legge di Snell
\[n\sin\theta_r=n_0\sin\theta_i\]
Calcoliamo ora le ampiezze dei campi riflessi e rifratti. Distinguiamo due casi: se il campo elettrico è polarizzato lungo $\hat{y}$ (ossia abbiamo un'onda incidente TE), anche i campi riflessi e rifratti devono avere la stessa polarizzazione. Analogamente, se il campo magnetico è polarizzato lungo $\hat{y}$ (onda TM), anche i campi riflessi e rifratti hanno la stessa polarizzazione. Partiamo dal caso TE. La continuità della componente parallela del campo elettrico si scrive come
\[E_{i}+E_r=E_t\]
La continuità della componente parallela del campo magnetico (ossia quella lungo $\hat{x}$) è invece equivalente, per la legge di Faraday, alla continuità di $k_zE_y$. Di conseguenza otteniamo
\[k_{i,z}E_i+k_{r,z}E_r=k_{t,z}E_t\]
Si ottiene quindi
\begin{align*}
	E_r&=\frac{k_{i,z}-k_{t,z}}{k_{i,z}+k_{t,z}}E_i\\
	E_t&=\frac{2k_{i,z}}{k_{i,z}+k_{t,z}}E_i\\
\end{align*}
I coefficienti che moltiplicano $E_i$ sono detti coefficienti di Fresnel. Introduciamo ora i coefficienti di riflessione e di trasmissione, definiti da
\begin{align*}
	R&=\frac{I_r}{I_i}\\T&=\frac{I_t}{I_i}
\end{align*}
Per la conservazione dell'energia, ci aspettiamo $R+T=1$. Limitandoci al caso in cui l'onda incidente si propaghi nel vuoto\footnote{O, alternativamente, indicando con $\tilde{n}$ l'indice di rifrazione relativo del secondo mezzo rispetto al primo.} e sia ad incidenza normale, si ha effettivamente
\begin{align*}
	R&=\frac{|E_r|^2}{|E_i|^2}=\\&=\left|\frac{k_{i,z}-k_{t,z}}{k_{i,z}+k_{t,z}}\right|^2=\\&=\left|\frac{1-\tilde{n}}{1+\tilde{n}}\right|^2\\T&=\frac{n_1|E_t|^2}{|E_i|^2}=\\&=n_1\left|\frac{2k_{i,z}}{k_{i,z}+k_{t,z}}\right|^2=\\&=\frac{4n_1}{|1+\tilde{n}|^2}\\R+T&=\frac{4n_1+|1-\tilde{n}|^2}{|1+\tilde{n}|^2}=\\&=\frac{4n_1+(1-n_1)^2+n_2^2}{(1+n_1)^2+n_2^2}=\\&=1
\end{align*}
Per le onde TM, la continuità della componente parallela del campo magnetico si scrive come
\[B_i+B_r=B_t\]
Di nuovo, per la legge di Ampère-Maxwell la continuità di $E_x$ è equivalente alla continuità di $k_zB_y/\varepsilon$
\[\frac{k_{i,z}}{\varepsilon_0}B_i+\frac{k_{r,z}}{\varepsilon_0}B_r=\frac{k_{t,z}}{\tilde\varepsilon}B_t\]
Allora si ottiene
\begin{align*}
B_r&=\frac{\tilde{\varepsilon}k_{i,z}-\varepsilon_0k_{t,z}}{\tilde{\varepsilon}k_{i,z}+\varepsilon_0k_{t,z}}B_i\\
B_t&=\frac{2\tilde{\varepsilon}k_{i,z}}{\tilde{\varepsilon}k_{i,z}+\varepsilon_0k_{t,z}}B_i\\
\end{align*}
Ossia per i campi elettrici
\begin{align*}
E_r&=\frac{\tilde{\varepsilon}k_{i,z}-\varepsilon_0k_{t,z}}{\tilde{\varepsilon}k_{i,z}+\varepsilon_0k_{t,z}}E_i\\
E_t&=\frac{\varepsilon_0^2}{\tilde{\varepsilon}^2}\frac{2\tilde{\varepsilon}k_{i,z}}{\tilde{\varepsilon}k_{i,z}+\varepsilon_0k_{t,z}}E_i\\
\end{align*}
Limitiamoci ora al caso in cui il secondo mezzo è trasparente. Il coefficiente di riflessione è
\begin{align*}
	R&=\left[\frac{n_0^2\cos\theta_i-n^2\cos\theta_r}{n_0^2\cos^2\theta_i+n^2\cos^2\theta_r}\right]^2
\end{align*}
Usando la legge di Snell si ottiene
\[R=\left[\frac{\frac{n^2}{n_0^2}\cos^2\theta_i-\sqrt{\frac{n^2}{n_0^2}-\sin^2\theta_i}}{\frac{n^2}{n_0^2}\cos^2\theta_i+\sqrt{\frac{n^2}{n_0^2}-\sin^2\theta_i}}\right]^2\]
Un grafico di $R$ è riportato in figura \ref{brewster}, dove è anche graficato l'andamento di $R$ per onde TE, dato da 
\[R=\left[\frac{\cos^2\theta_i-\sqrt{\frac{n^2}{n_0^2}-\sin^2\theta_i}}{\cos^2\theta_i+\sqrt{\frac{n^2}{n_0^2}-\sin^2\theta_i}}\right]^2\]

\begin{figure}[h]
	\centering 
	\begin{tikzpicture}
		\draw [-stealth](-7.283,0)--(-0.25,0)node[below]{$\theta_i$};
		\draw [-stealth](1,0)--(7.283+0.25,0)node[below]{$\theta_i$};
		\draw [-stealth](-7.283,0)--(-7.283,5)node[left]{$R$};
		\draw [-stealth](1,0)--(1,5)node[left]{$R$};
		
		\draw [smooth, samples=200,domain=-7.283:-1] plot(\x,{4*((4*(cos(14.323*(\x+7.283)))^2-sqrt(4-(sin(14.323*(\x+7.283)))^2))^2)/((4*(cos(14.323*(\x+7.283)))^2+sqrt(4-(sin(14.323*(\x+7.283)))^2))^2)});
		\draw [smooth, samples=200,domain=-7.283:-1] plot(\x,{4*(((cos(14.323*(\x+7.283)))^2-sqrt(4-(sin(14.323*(\x+7.283)))^2))^2)/(((cos(14.323*(\x+7.283)))^2+sqrt(4-(sin(14.323*(\x+7.283)))^2))^2)});
		\draw [dotted](-7.283,4)node[left]{1}--(-1,4)--(-1,0)node[below]{$\pi/2$};
		\node at(-3.85,0)[below]{$\theta_B$};
		\node at(-4,2){TE};
		\node at(-2,1){TM};
		
		\draw [smooth, samples=200,domain=1:3.0951766] plot(\x,{4*(((cos(14.323*(\x-1)))^2-sqrt(0.25-(sin(14.323*(\x-1)))^2))^2)/(((cos(14.323*(\x-1)))^2+sqrt(0.25-(sin(14.323*(\x-1)))^2))^2)});
		\draw [smooth, samples=200,domain=1:3.0951766] plot(\x,{4*((0.25*(cos(14.323*(\x-1)))^2-sqrt(0.25-(sin(14.323*(\x-1)))^2))^2)/((0.25*(cos(14.323*(\x-1)))^2+sqrt(0.25-(sin(14.323*(\x-1)))^2))^2)});
		\draw [dotted](1,4)node[left]{1}--(3.0951766,4)--(3.0951766,0)node[below right]{$\theta_L$};
		\node at(7.283,0)[below left]{$\pi/2$};
		\node at(2.9,0)[below]{$\theta_B$};
		\node at(2.5,2){TE};
		\node at(3.5,1){TM};
		
	\end{tikzpicture}
	\caption{Andamento di $R$ in funzione di $\theta_r$, nei due casi $n_0<n_1$ e $n_0>n_1$.}
	\label{brewster}
\end{figure}
\noindent Nel secondo caso si osserva banalmente che $R$ tende a 1 quando $\theta_i$ tende a $\theta_L$. Inoltre, per le onde TM esiste un angolo $\theta_B$, detto angolo di Brewster, in cui $R=0$. Si parla in questo caso di rifrazione totale.
\newpage
\section{30 aprile 2018}
\subsection{Elementi di analisi complessa}
Su $\C$ c'è una nozione naturale di distanza data da
\[d(z_1,z_2)=|z_1-z_2|\]
dove $z_1,z_2$ sono due complessi qualunque. Questa nozione ci permette di parlare di topologia, e dunque di continuità di una certa funzione $f\colon\C\to\C$ in un punto $z_0\in\C$. Ovviamente, $f$ è continua in tale punto se
\[\lim\limits_{z\to z_0}f(z)=f(z_0)\]
La condizione di continuità è però ben più forte della continuità in $\R$. Di fatto, posto $z=x+iy$ e $f=u+iv$, la continuità di $f$ è equivalente alla continuità del campo vettoriale $(u,v)\colon\R^2\to\R^2$. Analogamente, definiamo la derivata in senso complesso di $f$ nel punto $z_0$ come
\[f'(z_0)=\lim\limits_{\Delta z\to 0}\frac{f(z_0+\Delta z)-f(z_0)}{\Delta z}\]
ammesso chiaramente che il limite a secondo membro esista finito. Come per la continuità, la nozione di derivabilità in senso complesso è ben più delicata che in ambito reale. Vale il seguente
\begin{teorema}[Condizioni di Cauchy-Riemann]
	Sia $f\colon A\to\C$ una funzione, dove $A\subseteq\C$ è un aperto. Allora $f$ è derivabile in senso complesso in $A$ se e solo se esistono e sono continue le derivate parziali di $u$ e $v$ rispetto a $x$ e $y$ e verificano
	\[\begin{cases}
	\partial_xu=\partial_y{v}\\\partial_y{u}=-\partial_x{v}
	\end{cases}\]
\end{teorema}
\begin{proof}
	Mostriamo solo la necessità. Se $f$ è derivabile, allora esistono e coincidono
	\begin{align*}\lim\limits_{\Delta x\to 0}\frac{f(x+\Delta x+iy)-f(x+iy)}{\Delta x}\\\lim\limits_{\Delta y\to0}\frac{f(x+iy+i\Delta y)-f(x+iy)}{i\Delta y}\end{align*}
	Ossia si ha
	\begin{align*}
		\partial_xu+i\partial_xv=-i(\partial_yu+i\partial_yv)
	\end{align*}
	Da cui si ha la tesi, uguagliando parte reale e parte immaginaria.
\end{proof}
Una funzione derivabile in un aperto $A$ è detta analitica in $A$. Consideriamo ora un aperto $A\subset C$, una curva regolare (eventualmente a tratti) $\gamma\colon[0,1]\to A$ e una funzione continua $f\colon A\to\C$. Se $f=u+iv$, definiamo l'integrale di $f$ lungo $\gamma$ come
\[\int_{\gamma}f(z)\d z=\int_{\gamma}(u\d x-v\d y)+i\int_{\gamma}(u\d y+v \d x)\]
Il verso positivo di $\gamma$ è antiorario.
\begin{teorema}[di Cauchy-Goursat]
	Sia $A\subseteq \C$ un aperto semplicemente connesso, $f\colon A\to\C$ una funzione analitica in $A$. Se $\gamma\colon[0,1]\to A$ è una curva chiusa e regolare a tratti, allora
	\[\oint_\gamma f(z)\d z=0\]
\end{teorema}
\begin{proof}
	Daremo la dimostrazione di Cauchy, che richiede anche che le derivate parziali siano continue. Goursat diede una dimostrazione senza questa ipotesi, e permette di avere importanti risultati. Posto $f=u+iv$, consideriamo i campi vettoriali $(u,-v)$ e $(v,u)$. Le condizioni di Cauchy-Riemann implicano che entrambi i campi sono chiusi, dunque esatti dato che $A$ è semplicemente connesso.
\end{proof}
\begin{teorema}[Formula integrale di Cauchy]
	Sia $A\subseteq\C$ un aperto, $f\colon A\to\C$ una funzione analitica, $z_0$ un punto di $A$. Se $\gamma\colon[0,1]\to A$ è una curva chiusa e semplice che contiene $z_0$ al suo interno, allora
	\[f(z_0)=\frac{1}{2\pi i}\oint_\gamma \frac{f(\zeta)}{\zeta-z_0}\d \zeta\]
\end{teorema}
\begin{proof}
	Consideriamo una circonferenza $\tilde{\gamma}\colon[0,1]\to A$ di raggio $r$ con centro in $z_0$ percorsa in senso antiorario. Siano inoltre $z_\gamma=\gamma(0)$ e $z_{\tilde{\gamma}}=\tilde\gamma(0)$, e sia $\xi\colon[0,1]\to A$ la curva
	\[\xi(t)=z_\gamma+t(z_{\tilde{\gamma}}-z_\gamma)\]
	Consideriamo ora la curva $\Gamma\colon[0,4]\to A$ definita da
	\[\Gamma(t)=\begin{cases}
		\gamma(t)&\textrm{ se }0\leq t\leq 1\\
		\xi(t-1)&\textrm{ se }1\leq t\leq 2\\
		\tilde{\gamma}(3-t)&\textrm{ se }2\leq t\leq 3\\
		\xi(4-t)&\textrm{ se }3\leq t\leq 4
	\end{cases}\]
	Dato che $f(\zeta)/(\zeta-z_0)$ è analitica nei punti interni a $\Gamma$, per il teorema di Cauchy-Goursat si ha
	\begin{align*}
		\oint_\Gamma \frac{f(\zeta)}{\zeta-z_0}\d \zeta&=\oint_\gamma \frac{f(\zeta)}{\zeta-z_0}\d \zeta-\oint_{\tilde{\gamma}}\frac{f(\zeta)}{\zeta-z_0}\d \zeta=\\&=0
	\end{align*}
	dove si è tenuto conto dell'orientazione di $\tilde{\gamma}$. Ora si ha
	\begin{align*}\oint_{\tilde{\gamma}}\frac{f(\zeta)}{\zeta-z_0}\d \zeta&=\int_{0}^{2\pi}\frac{f(z_0+re^{i\theta})}{re^{i\theta}}ire^{i\theta}d\theta=\\&=i\int_{0}^{2\pi}f(z_0+re^{i\theta})\d\theta\end{align*} 
	Quando $r\to0$, il secondo membro tende a $2\pi if(z_0)$.
\end{proof}
\noindent Dalla formula integrale di Cauchy si può mostrare che una funzione analitica è derivabile infinite volte, e che si ha
\[f^{(n)}(z_0)=\frac{n!}{2\pi i}\oint_\gamma\frac{f(\zeta)}{(\zeta-z_0)^{n+1}}\d\zeta\]
Inoltre, dalle condizioni di Cauchy-Riemann si deduce anche che $u$ e $v$, viste come funzioni di due variabili, sono armoniche e con gradienti mutualmente ortogonali. Consideriamo ora una funzione $f$ analitica in una corona circolare di centro $z_0$, raggio interno $r$ e raggio esterno $R$. Si può mostrare che $f$ si scrive in maniera unica come serie di Taylor-Laurent
\[f(z)=\sum_{k=-\infty}^{+\infty}c_k(z-z_0)^k\]
Osservando che, presa una curva $\gamma$ chiusa e semplice contenuta nella corona circolare, si ha
\begin{align*}
	\oint_\gamma\frac{1}{(z-z_0)^n}\d z=2\pi i\delta_{n,1}
\end{align*}
si ottiene
\[c_k=\frac{1}{2\pi i}\oint_\gamma\frac{f(z)}{(z-z_0)^{k+1}}\d z\]
\subsection{Esercizi proposti}
\begin{problema}
	Considerando il modello di Drude-Lorentz, la regione nel limite di alte frequenze si può considerare I o II?
\end{problema}
\begin{soluzione}
	In questa regione si ha
	\begin{align*}
		\varepsilon_1&\simeq1-\frac{\omega_p^2}{\omega^2}\\\varepsilon_2&\simeq0
	\end{align*}
	Da qui si deduce che l'indice di rifrazione è minore di 1 e che la velocità di fase è
	\[v_f\simeq\frac{c}{\sqrt{1-\omega_p^2/\omega^2}}>c\]
	Non possiamo quindi essere nella regione I, in cui la velocità di fase e la velocità di gruppo coincidono. Per quest'ultima si ha
	\[v_g\simeq\frac{ck}{\sqrt{c^2k^2+\omega_p^2}}<c\]
	dove si è usata la relazione di dispersione
	\begin{align*}
		c^2k^2&=n^2\omega^2\simeq\\&\simeq\omega^2-\omega_p^2
	\end{align*}
	Notiamo che si ha $v_gv_f=c^2$. Inoltre, la densità di energia è
	\begin{align*}
		u&=\frac{E^2}{8\pi\omega}\der{\omega^2\varepsilon}{\omega}=\\&=\frac{E^2}{8\pi\omega}\der{k^2c^2}{\omega}=\\&=\frac{c^2E^2}{8\pi\omega v_g}\der{k^2}{k}=\\&=\frac{c^2E^2}{4\pi v_gv_f}=\\&=\frac{E^2}{4\pi}
	\end{align*}
	cioè coincide con l'espressione nel vuoto. Per il vettore di Poynting si ha invece
	\begin{align*}
		S_z=\frac{nc E^2}{4\pi}<S_{z,\textrm{vuoto}}
	\end{align*}
\end{soluzione}
\begin{problema}
	Unire gli effetti di campo locale e il modello di Drude-Lorentz.
\end{problema}
\begin{soluzione}
	Tenendo conto degli effetti di campo locale, sappiamo che
	\[\chi=\frac{4\pi n\alpha}{1-\frac{4\pi n\alpha}{3}}\]
	D'altro canto, il modello di Drude-Lorentz ci dà
	\[4\pi n\alpha=\frac{\omega_p^2}{\omega_0^2-\omega^2-i\gamma\omega}\]
	Allora si ha
	\begin{align*}\chi&=\frac{\omega_p^2/(\omega_0^2-\omega^2-i\gamma\omega)}{1-\frac{1}{3}\omega_p^2/(\omega_0^2-\omega^2-i\gamma\omega)}=\\&=\frac{\omega_p^2}{\omega_1^2-\omega^2-i\gamma\omega}
	\end{align*}
	dove si è posto
	\[\omega_1^2=\omega_0^2-\frac{1}{3}\omega_p^2\]
	Dunque, a meno di ridefinire la frequenza di richiamo, possiamo includere gli effetti di campo locale nel modello di Drude-Lorentz.
\end{soluzione}
\begin{problema}
	Trovare il campo elettrico generato da una carica puntiforme $q$ in un mezzo omogeneo e anisotropo.
\end{problema}
\begin{soluzione}
	Poniamo per semplicità la carica nell'origine e orientiamo gli assi in modo che
	\[\varepsilon_{ij}=\left(\begin{array}{c c c}
	\varepsilon_{xx}&0&0\\0&\varepsilon_{yy}&0\\0&0&\varepsilon_{zz}
	\end{array}\right)\]
	Dall'irrotazionalità di $\vec{E}$ segue che $\vec{E}=-\nabla\phi$. Allora la legge di Gauss si scrive come
	\[(\varepsilon_{xx}\partial_{xx}+\varepsilon_{yy}\partial_{yy}+\varepsilon_{zz}\partial_{zz})\phi=-4\pi q\delta^3(\vec{r})\]
	Introduciamo ora le variabili
	\begin{align*}
		x'&=\frac{x}{\sqrt{\varepsilon_{xx}}}\\y'&=\frac{y}{\sqrt{\varepsilon_{yy}}}\\z'&=\frac{z}{\sqrt{\varepsilon_{zz}}}
	\end{align*}
	In termini di queste nuove variabili, abbiamo semplicemente
	\[(\partial_{x'x'}+\partial_{y'y'}+\partial_{z'z'})\phi=-\frac{4\pi q}{\sqrt{\varepsilon_{xx}\varepsilon_{yy}\varepsilon_{zz}}}\delta^3(\vec{r}')\]Allora si ha
	\begin{align*}
		\phi&=\frac{q}{\sqrt{\varepsilon_{xx}\varepsilon_{yy}\varepsilon_{zz}}\sqrt{x'^2+y'^2+z'^2}}=\\&=\frac{q}{\sqrt{\varepsilon_{xx}\varepsilon_{yy}\varepsilon_{zz}}\sqrt{\frac{x^2}{\varepsilon_{xx}}+\frac{y^2}{\varepsilon_{yy}}+\frac{z^2}{\varepsilon_{zz}}}}
	\end{align*}
	da cui si può ottenere il campo elettrico. Osserviamo che le superfici equipotenziali sono ellissoidi.
\end{soluzione}
\newpage
\section{3 maggio 2018}
\subsection{Relazioni di Kramers-Kronig}
Nella trattazione fatta finora, abbiamo sempre considerato frequenze reali. Vediamo cosa accade se invece si ha $\omega=\tilde{\omega}=\omega_1+i\omega_2$, prendendo ad esempio una generica funzione oscillante
\[f(\vec{r},t)=\varphi(\vec{r})e^{i\omega t}\]
Sostituendo si ha
\[f(\vec{r},t)=\varphi(\vec{r})e^{-\omega_2t}e^{i\omega_1t}\]
La parte immaginaria della frequenza è quindi utile per descrivere ampiezze che variano esponenzialmente nel tempo, in maniera simile a quanto visto con vettori d'onda complessi. Consideriamo ora la funzione di risposta $g$ tale che
\[\vec{P}(t)=\int_{-\infty}^{+\infty}g(t-t')\vec{E}(t')\d t'\]
con la condizione $g(\tau)=0$ se $\tau<0$, in modo da rispettare la causalità. Si può mostrare che la causalità è equivalente all'analiticità nel semipiano complesso superiore della funzione
\[\tilde{\chi}(\omega_1+i\omega_2)=\int_{-\infty}^{+\infty}g(t)e^{i\omega_1t}e^{-\omega_2t}\d t\]
Fissiamo ora una certa $\omega$ reale e consideriamo la curva $\Gamma$ nel piano complesso formata da una semicircoferenza $C$ di raggio $R$ centrata nell'origine, una semicirconferenza $c$ di raggio $\varepsilon$ centrata in $\omega$ e due segmenti, percorsa in senso antiorario
\begin{figure}[h]
	\centering
	\scalebox{1}{\begin{tikzpicture}
			\draw[-stealth](-6,0)--(6,0)node[below]{$\omega_1$};
			\draw[-stealth](0,-2)--(0,6)node[left]{$\omega_2$};
			\node at(2,0)[below]{$\omega$};
			\draw [-stealth](1.5,0.1) arc [start angle=180, end angle=90, x radius=0.5, y radius=0.5];
			\draw (2,0.1) arc[start angle=90,end angle=0, x radius=0.5, y radius=0.5]
			\draw [-stealth] (0,0.1) arc[start angle=0, end angle=70, x radius=5, y radius=5];
			\draw [-stealth] (0,0.1) arc[start angle=0, end angle=70, x radius=5, y radius=5];
			
	\end{tikzpicture}}
\end{figure}
\subsection{Esercizi proposti}
\chapter{Esercizi}
\section{2 ottobre 2017}
\begin{problema}
	Si consideri una sfera di raggio $a$ con la seguente distribuzione di carica
	\[\rho(r)=\rho_0\left(\frac{r}{a}\right)^n\]
	Trovare il campo in tutto lo spazio e dire per quali $n$ è ben definito
\end{problema}
\begin{soluzione}
	Per motivi di simmetrica il campo deve essere della forma $\vec{E}(\vec{r})=E(r)\hat{r}$. Notiamo che la carica totale è
	\[Q=\frac{4\pi\rho_0}{a^n}\int_{0}^{a}r^{2+n}\dif r\]
	Tale integrale converge solo per $n>-3$. Inoltre, usando il teorema di Gauss si trova
	\[\vec{E}(\vec{r})=\frac{4\pi\rho_0\hat{r}}{n+3}\left\{\begin{array}{l l}	
	r^{n+1}/a^n&\textrm{ se }r\leq a\\a^3/r^2&\textrm{ se }r>a
	\end{array}\right.\]
\end{soluzione}
\begin{problema}
	Si consideri una sfera di raggio $b$ con una cavità concentrica di raggio $a$, carica uniformemente con una densità $\rho$. Trovare il campo elettrico in tutto lo spazio.
\end{problema}
\begin{soluzione}
	Anche qui, per la simmetria della distribuzione di carica, il campo deve essere radiale e di modulo costante sulle sfere. In particolare, per $r<a$ è nullo, per $r>a$ coincide con il campo generato da una carica puntiforme $Q=\frac{4}{3}\pi\rho\left(b^3-a^3\right)$. Per $a\leq r\leq b$ si trova, usando Gauss
	\[\vec{E}(\vec{r})=\frac{4}{3}\pi\rho\left(r-\frac{a^3}{r^2}\right)\hat{r}\]
	Tale risultato è stato ottenuto grazie al principio di sovrapposizione. In particolare, abbiamo sovrapposto due sfere concentriche, una di raggio $b$ e densità di carica $\rho$ e una di raggio $a$ e densità di carica $-\rho$.
\end{soluzione}
\begin{problema}
	Si consideri una sfera uniformemente carica, con densità di carica $\rho$, al cui interno è presente una cavità sferica non concentrica. Trovare il campo elettrico all'interno della cavità.	
\end{problema}
\begin{soluzione}
	Sia $\vec{d}$ il vettore con coda nel centro della sfera carica e punta nel centro della cavità. Usando il principio di sovrapposizione, possiamo pensare il campo come somma dei campi generati da una sfera carica con densità $\rho$ e da una sfera interna con densità $-\rho$. Allora il campo in un punto della cavità a distanza $\vec{r}$ dal centro è
	\[\vec{E}(\vec{r})=\frac{4}{3}\pi\rho\vec{r}-\frac{4}{3}\pi\rho\left(\vec{r}-\vec{d}\right)=\frac{4}{3}\pi\rho\vec{d}\]
	ovvero il campo elettrico è uniforme in tutta la cavità.
\end{soluzione}
\begin{problema}
	\label{deltae}
	Si consideri una regione di spazio in cui è presente una superficie con densità superficiale di carica $\sigma$. Allora tale regione è divisa in due parti, in cui sono presenti delle densità di carica volumiche rispettivamente $\rho_1$ e $\rho_2$. Se $\vec{E}_1$ e $\vec{E}_2$ sono i campi elettrici nelle due regione, che relazione c'è tra $\vec{E}_1,\vec{E}_2$ e $\sigma$?
\end{problema}
\begin{soluzione}
	Consideriamo un cilindro di area di base $S$ e altezza $2h\ll\sqrt{S}$ che interseca la superficie, con $S$ sufficientemente piccola da poter considerare la porzione di superficie carica intercettata come piana. Allora, se $\hat{n}$ è il verso uscente con verso nella regione con densità $\rho_1$, per il teorema di Gauss si ha
	\[S\left(\vec{E}_1-\vec{E}_2\right)\cdot\hat{n}=4\pi S\left(\sigma+h\rho_1+h\rho_2\right)\]
	Se $h\to0$ e $\Delta E_n=\left(\vec{E}_1-\vec{E}_2\right)\cdot\hat{n}$ è la differenza tra le componenti di $\vec{E}$ normali alla superficie si trova
	\[\Delta E_n=4\pi\sigma\]
	Quindi il campo elettrico è discontinuo solo in presenza di densità superficiali di carica.
\end{soluzione}
\begin{problema}
	Consideriamo una sfera di raggio $R$ con densità superficiale $\sigma$. Trovare $\sigma$ in modo che il campo all'interno della sfera sia uniforme.
\end{problema}
\begin{soluzione}
	Consideriamo due sfere di raggio $R$, con densità volumiche di carica $\pm\rho$. Sia $d<2R$ la distanza tra i centri. Nella cavità comune alle due sfere il campo è uniforme e pari a
	\[\vec{E}=\frac{4}{3}\pi\rho d\hat{x}\]
	Vogliamo ora fare i limiti $d\to0$, $\rho\to\infty$ in modo che il prodotto $\sigma_0=\rho d$ rimanga costante. In tal modo otteniamo la distribuzione cercata, che è
	\[\sigma(\theta)=\sigma_0\cos\theta\]
	dove $\theta$ è la latitudine. All'esterno della sfera, il campo è esattamente quello di un dipolo, con un momento di dipolo $\vec{p}=\frac{4}{3}\pi R^3\sigma_0\hat{x}$. Il potenziale in un punto $\vec{r}$ è allora
	\[V(\vec{r})=\frac{\vec{p}\cdot\hat{r}}{r^2}\]
	Di conseguenza l'energia del sistema è
	\[U=\frac{1}{2}\int\rho V\dif^3x=\frac{8\pi^2\sigma_0^2 R^3}{9}\]
	Alternativamente, possiamo usare
	\[U=\frac{1}{8\pi}\int|\vec{E}|^2\dif^3x\]
	Questi ultimi conti sono più brutti, visto l'andamento del campo di un dipolo.
	Se consideriamo ancora il caso delle due sfere separate, l'energia del sistema è
	\[U=\frac{8\pi^2\rho^2d^2R^3}{9}\]
	Di conseguenza la forza con interagiscono le forze è
	\[F=-\frac{\partial U}{\partial d}=-\frac{16\pi^2\rho^2R^3}{9}d\]
	Quindi le due sfere oscillano con frequenza
	\[\omega^2=\frac{16\pi^2\rho^2R^3}{9m}\]
	dove $m$ è la massa di una sfera. Se supponiamo che una sfera sia un reticolo ionico e l'altra una nuvola di elettroni, allora la sfera di ioni rimane pressochè ferma (dato che è molto più massiva della nuvola di elettroni). Se $n_e$ è la densità di elettroni, si trova
	\[\omega^2=\frac{4\pi e^2n_e}{3m_e}\]
	La grandezza
	\[\omega_p^2=\frac{4\pi n_ee^2}{m_e}\]
	è detta frequenza di plasma, e queste oscillazioni sono dette oscillazioni di Mie.
\end{soluzione}
\begin{problema}
	Si consideri un guscio sferico con densità uniforme $\sigma$. Se togliamo un dischetto sufficientemente piccolo dalla superficie, qual è il campo elettrico al centro del buco creato? Se il dischetto viene inserito successivamente, qual è la forza di cui risente? E la pressione?
\end{problema}
\begin{soluzione}
	Se il tappo è presente, il campo nelle sue immediate vicinanze è quello di una sfera, ossia $4\pi\sigma\hat{z}$ all'esterno, nullo all'interno. Si è indicato con $\hat{z}$ il versore che punta dal centro della sfera al centro del disco. Se consideriamo un tappo con carica $-\sigma$ sufficientemente piccolo da poter essere considerato piano, e se consideriamo un punto sufficientemente vicino al tappo in modo da poter utilizzare l'approssimazione di piano infinito, troviamo che il campo generato dal disco al suo centro è $-2\pi\sigma\hat{z}$ all'esterno, $2\pi\sigma\hat{z}$ all'interno. Allora per il principio di sovrapposizione il campo al centro della cavità è $2\pi\sigma\hat{z}$. La forza di cui risente il tappo si ottiene integrando il campo su tutto il dischetto. Se questo è un cerchio, allora i contributi non diretti lungo $\hat{z}$  si elidono. Dobbiamo quindi integrare la componente lungo $\hat{z}$ del campo, che però per il problema \ref{deltae} è costante e pari a $2\pi\sigma$. Allora se $A$ è l'area del dischetto si trova $\vec{F}=2\pi\sigma^2A\hat{z}$, da cui la pressione $p=2\pi\sigma^2$. Alternativamente, l'energia di un guscio di raggio $R$ senza tappo è
	\[U=\frac{1}{8\pi}\int_{0}^{\pi}\int_{0}^{2\pi}\int_{R}^{\infty}\frac{q^2}{r^2}\sin\theta\dif r\dif\phi\dif\theta=\frac{q^2}{2R}\]
	Se il raggio aumenta di $\delta R$, si ha
	\[\delta U=-\frac{q^2}{2R^2}\delta R\]
	D'altro canto si ha
	\[\delta U=-p\delta V=-4\pi R^2p\delta R\]
	Da cui il risultato precedente.
\end{soluzione}
\newpage
\section{9 ottobre 2017}
\begin{problema}
	Calcolare gradiente, divergenza e laplaciano in coordinate cilindriche.
\end{problema}
\begin{soluzione}
	L'elemento di lunghezza è
	\[\dif\vec{s}=\dif r\hat{r}+r\dif\theta\hat{\theta}+\dif z\hat{z}\]
	Di conseguenza, presa una funzione scalare $f$ si ha
	\[\nabla f\cdot\dif\vec{r}=\left(\nabla f\right)_r\dif r+\left(\nabla f\right)_\theta r\dif\theta+\left(\nabla f\right)_z\dif z=\dif f=\der{f}{r}\dif r+\der{f}{\theta}\dif \theta+\der{f}{z}\dif z\]
	Si ottiene
	\[\nabla f=\left(\der{f}{r},\frac{1}{r}\der{f}{\theta},\der{f}{z}\right)\]
	Prendiamo ora una funzione vettoriale $\vec{A}$ e consideriamo un volume $\dif V$. Per il teorema della divergenza di ha
	\[\nabla\cdot A\dif V=\Phi_{\dif S}(\vec{A})\]
	Dove $\dif S$ è la superficie che circonda $\dif V$. Si ha
	\[\Phi_{\dif S}(\vec{A})=A_r(r+\dif r,\theta,z)(r+\dif r)\dif\theta\dif z-A_r(r,\theta,z)r\dif\theta\dif z+\left(A_z(r,\theta,z+\dif z)-A_z(r,\theta,z)\right)r\dif r\dif\theta+\]\[+\left(A_\theta(r,\theta+\dif\theta,z)-A_\theta(r,\theta,z)\right)\dif r\dif z=\left(\frac{1}{r}\der{(rA_r)}{r}+\frac{1}{r}\der{A_\theta}{\theta}+\der{A_z}{z}\right)r\dif r\dif\theta\dif z\]
	E infine
	\[\nabla\cdot\vec{A}=\frac{1}{r}\der{(rA_r)}{r}+\frac{1}{r}\der{A_\theta}{\theta}+\frac{1}{r\sin\theta}\der{A_\phi}{\phi}\]
	Considerata nuovamente $f$, si ha semplicemente
	\[\lap f=\nabla\cdot\left(\nabla f\right)=\frac{1}{r}\frac{\partial}{\partial r}\left(r\der{f}{r}\right)+\frac{1}{r^2}\der[2]{f}{\theta}+\der[2]{f}{z}\]
\end{soluzione}
\begin{problema}
	Calcolare gradiente, divergenza e laplaciano in coordinate sferiche.
\end{problema}
\begin{soluzione}
	L'elemento di lunghezza è
	\[\dif\vec{s}=\dif r\hat{r}+r\dif\theta\hat{\theta}+r\sin\theta\dif \phi\hat{\phi}\]
	Di conseguenza, presa una funzione scalare $f$ si ha
	\[\nabla f\cdot\dif\vec{s}=\left(\nabla f\right)_r\dif r+\left(\nabla f\right)_\theta r\dif\theta+\left(\nabla f\right)_\phi r\sin\theta\dif \phi=\dif f=\der{f}{r}\dif r+\der{f}{\theta}\dif \theta+\der{f}{\phi}\dif \phi\]
	Si ottiene
	\[\nabla f=\left(\der{f}{r},\frac{1}{r}\der{f}{\theta},\frac{1}{r\sin\theta}\der{f}{\phi}\right)\]
	Prendiamo ora una funzione vettoriale $\vec{A}$ e consideriamo un volume $\dif V$. Per il teorema della divergenza di ha
	\[\nabla\cdot A\dif V=\Phi_{\dif S}(\vec{A})\]
	Dove $\dif S$ è la superficie che circonda $\dif V$. Si ha
	\[\Phi_{\dif S}(\vec{A})=A_r(r+\dif r,\theta,\phi)(r+\dif r)\sin\theta\dif\theta\dif z-A_r(r,\theta,\phi)r\sin\theta\dif\theta\dif \phi+(A_\phi(r,\theta,\phi+\dif \phi)+\]\[-A_\phi(r,\theta,\phi))r\dif r\dif\theta+A_\theta(r,\theta+\dif\theta,\phi)r\sin(\theta+\dif\theta)\dif r\dif\phi-A_\theta(r,\theta,\phi)r\sin\theta\dif r\dif \phi=\]\[=\left(\frac{1}{r}\der{(rA_r)}{r}+\frac{1}{r}\der{A_\theta}{\theta}+\frac{1}{r\sin\theta}\der{A_\phi}{\phi}\right)r^2\sin\theta\dif r\dif\theta\dif \phi\]
	E infine\[\nabla\cdot\vec{A}=\frac{1}{r^2}\der{(r^2A_r)}{r}+\frac{1}{r\sin\theta}\der{A_\theta\sin\theta}{\theta}+\frac{1}{r\sin\theta}\der{A_\phi}{\phi}\]
	
	Considerata nuovamente $f$, si ha semplicemente
	\[\lap f=\nabla\cdot\left(\nabla f\right)=\frac{1}{r^2}\frac{\partial}{\partial r}\left(r^2\der{f}{r}\right)+\frac{1}{r^2\sin\theta}\der{}{\theta}\left(\sin\theta\der{f}{\theta}\right)+\frac{1}{r^2\sin^2\theta}\der[2]{f}{\phi}\]
\end{soluzione}
\newpage
\section{19 ottobre 2017}
\begin{problema}
	Trovare la distribuzione di carica che genera il potenziale
	\[V(x,y,z)=\left\{\begin{array}{l l}
	V_0&\textrm{ se }z\leq0\\
	V_0e^{-z/l}&\textrm{ se }z>0
	\end{array}\right.\]
\end{problema}
\begin{soluzione}
	Il sistema è simmetrico per traslazioni lungo $x$ e $y$, quindi la densità di carica dipenderà unicamente dalla coordinata $z$. Per $z\neq0$ il potenziale è derivabile due volte, dunque per Poisson si ottiene
	\[\rho(z)=\left\{\begin{array}{l l}
	0&\textrm{ se }z<0\\
	-\frac{V_0}{4\pi l^2}e^{-z/l}&\textrm{ se }z>0
	\end{array}\right.\]
	In $z=0$ il campo elettrico, ossia $-V'$, è discontinuo. Ci aspettiamo quindi una distribuzione di carica superficiale su tale piano. Ricordando che attraversando una superficie carica si ha $\Delta E_n=4\pi\sigma$, si ottiene
	\[\sigma=\frac{V_0}{4\pi l^2}\]
	Tale risultato si poteva anche ottenere dal teorema di Gauss, scegliendo come superficie un cilindro con asse parallelo all'asse $z$ e facendo tendere l'altezza del cilindro a infinito. La distribuzione può anche essere scritta nella forma
	\[\rho(z)=\frac{V_0}{4\pi l^2}\left(\delta(z)-e^{-z/l}\theta(z)\right)\]
	Dove $\theta$ è la funzione di Heaviside.
\end{soluzione}
\begin{problema}
	Si consideri un tetraedro le cui facce sono mantenute ai potenziali $V_1,V_2,V_3,V_4$. Qual è il potenziale al centro del tetraedro?
\end{problema}
\begin{soluzione}
	Per il principio di sovrapposizione, è sufficiente trovare il contributo di una singola faccia al potenziale nel centro. Supponiamo che una faccia sia al potenziale $V_0$ e le altre a potenziale nullo. Il contributo al centro non dipende dalla faccia scelta, come si vede ruotando il tetraedro. Inoltre, se consideriamo il tetraedro nel caso in cui tutte le facce sono allo stesso potenziale $V$, allora il potenziale al centro è di nuovo $V$ (dato che in tale volume il potenziale è armonico e costante sul bordo, quindi costante ovunque). Da ciò deduciamo che, nel caso di una sola faccia a potenziale $V_0$ non nullo, il potenziale al centro è $V_0/4$. Allora nel caso generale il potenziale al centro è la media dei quattro potenziali.
\end{soluzione}
\newpage
\section{23 ottobre 2017}
\begin{problema}
	Si consideri un cilindro di raggio $a$ e altezza $h\gg a$ con una densità di carica uniforme $\rho>0$. Nel cilindro è scavato un tubicino di dimensioni trascurabili inclinato di un angolo $\alpha$ rispetto all'asse, passante per quest'ultimo. In esso viene posto un punto materiale di massa $m$ e carica $-q<0$. Se il punto è fermo per $t=0$, qual è la sua legge oraria?
\end{problema}
\begin{soluzione}
	Fatte le opportune considerazioni sulle simmetrie del sistema, e quindi del campo elettrico, in coordinate cilindriche si trova
	\[\vec{E}(r,\phi,z)=2\pi r\rho\hat{r}\]
	L'equazione del moto è allora
	\[m\ddot{x}=-2\pi x\sin^2\alpha\rho q\]
	Ossia un'oscillazione armonica
	\[x(t)=x_0\cos\omega t\]
	Con una frequenza
	\[\omega=\sin\alpha\sqrt{\frac{2\pi\rho q}{m}}\] 
\end{soluzione}
\begin{problema}
	Si consideri il semispazio $z\leq0$ come un conduttore scarico a potenziale nullo. Se si pongono due cariche puntiformi $q$ entrambe a distanza $d$ dal piano e a una distanza reciproca $a$, qual è l'energia elettrostatica del sistema?
\end{problema}
\begin{soluzione}
	Chiaramente l'energia è
	\[U_e=\frac{q^2}{a}-\frac{q^2}{2d}-\frac{q^2}{\sqrt{4d^2+a^2}}\]
\end{soluzione}
\begin{problema}
	Si considerino due sfere conduttrici di raggi $a$ e $b$ (con $b>a$) poste a distanza $d\gg b$ e collegate da un filo conduttore molto sottile. Qual è il campo in prossimità della superficie di ogni sfera, se sul sistema viene posta una carica $Q$?
\end{problema}
\begin{soluzione}
	Siano $q_a,q_b$ le cariche sulle due sfere. Ovviamente $q_a+q_b=Q$, inoltre all'equilibrio elettrostatico le due sfere sono allo stesso potenziale. Data la condizione su $d$, possiamo in prima approssimazione trascurare l'interazione tra le cariche, dunque in tal caso si deve avere
	\[\frac{q_a}{a}=\frac{q_b}{b}\]
	Se invece vogliamo fare un lavoro un po' più di fino, possiamo richiedere
	\[\frac{q_a}{a}+\frac{q_b}{d}=\frac{q_b}{b}+\frac{q_a}{d}\]
	Trovate le cariche nei due casi, i campi sulla superficie sono
	\[E_a=\frac{q_a}{a^2}+\frac{q_b}{d^2}\]
	\[E_b=\frac{q_a}{d^2}+\frac{q_b}{b^2}\]
\end{soluzione}
\begin{problema}
	Consideriamo un piano conduttore scarico a potenziale nullo e si ponga una carica $q$ a distanza $d$ dal piano. Inseriamo ora un disco conduttore di raggio $a\gg d$ parallelo al piano a distanza $d/2$ da quest'ultimo, in modo che il centro del disco appartenga alla retta passante per $q$ ortogonale al piano. Stimare l'energia necessaria per rimuovere il disco.
\end{problema}
\begin{soluzione}
	Quando il disco è inserito, possiamo ipotizzare che nel semispazio $z>d/2$ si abbia il potenziale generato da $q$ e da una carica immagine $-q$ posta simmetricamente rispetto al disco. Questa carica schematizza la distribuzione di carica sul lato del disco affacciato su $q$. Sull'altro lato, possiamo mettere una carica $q$ uniformemente distribuita, quindi otteniamo un condensatore. Allora l'energia elettrostatica è
	\[U_e=-\frac{q^2}{2d}+\frac{q^2d}{a^2}\]
	Senza il disco inserito, l'energia elettrostatica è banalmente
	\[U'_e=-\frac{q^2}{4d}\]
	Da cui si ricava l'energia richiesta.
\end{soluzione}
\newpage
\section{30 ottobre 2017}
\begin{problema}
	Si considerino una sfera di raggio $a$ con centro in $(0,0,-2a)$ e una sfera di raggio $b$ con centro in $(0,0,2b)$. Entrambe le sfere sono uniformemente cariche, con densità di carica rispettivamente $\rho_a<0$ e $\rho_b>0$. Trovare il massimo e il minimo lavoro che bisogna compiere per spostare una particella di carica $q>0$ da un punto della sfera di raggio $a$ a un punto della sfera di raggio $b$.
\end{problema}
\begin{soluzione}
	Siano $Q_a=\frac{4}{3}\pi\rho_aa^3$ e $Q_b=\frac{4}{3}\pi\rho_bb^3$ le cariche sulle due sfere. Il potenziale sulla sfera $a$ è massimo in $(0,0,-a)$ e minimo in $(0,0,-3a)$. In tali punti vale rispettivamente
	\[V_M=\frac{Q_a}{a}+\frac{Q_b}{3b+a}\]
	\[V_M=\frac{Q_a}{a}+\frac{Q_b}{3b+3a}\]
	Analogamente, il massimo e il minimo del potenziale sulla sfera $b$ sono in $(0,0,3b)$ e in $(0,0,b)$ e valgono
	\[V'_M=\frac{Q_b}{b}+\frac{Q_a}{3b+3a}\]
	\[V'_M=\frac{Q_b}{b}+\frac{Q_a}{3b+a}\]
	Allora si ha semplicemente
	\[L_{max}=q(V'_M-V_m)\]
	\[L_{min}=q(V'_m-V_M)\]
\end{soluzione}
\begin{problema}
	Si consideri una sfera di raggio $a$ con una densità di carica a simmetria sferica $\rho(r)$. Sull'asse $x$ è presente un tunnel di dimensioni trascurabili e in $(-a,0,0)$ è posta un corpo puntiforme di massa $m$, carica $q>0$ e velocità iniziale $\vec{v}_0=(v_0,0,0)$, con $v_0>0$. Dire se esiste una velocità minima $v_m$ tale che se $v_0>v_m$ la carica puntiforme attraversa la sfera nei seguenti casi
	\begin{enumerate}
		\item $\rho(r)=\rho_1\sqrt{a/r}$
		\item $\rho(r)=\rho_2(a/r)$
		\item $\rho(r)=\rho_3(a/r)^2$
	\end{enumerate}	
\end{problema}
\begin{soluzione}
	Una tale velocità esiste se l'energia iniziale della carica è sufficiente a raggiungere il centro della sfera. Allora è sufficiente imporre che il potenziale nell'origine sia finito. In particolare, dato che il campo elettrico è per simmetria $\vec{E}(\vec{r})=E(r)\hat{r}$, usando il teorema di Gauss si ottiene nei tre casi
	\begin{enumerate}
		\item $E(r)\propto \sqrt{r}$, dunque $V(r)\propto r^{3/2}$ e $v_m$ esiste.
		\item $E(r)$ è costante, dunque $V(r)\propto r$ e $v_m$ esiste.
		\item $E(r)\propto r^{-1}$, dunque $V(r)\propto \ln r$ e $v_m$ non esiste.
	\end{enumerate}
\end{soluzione}
\begin{problema}
	Si consideri un anello di raggio $a$ nel piano $xy$ con centro nell'origine e densità lineare di carica $\lambda$. Determinare la prima correzione non nulla al potenziale in un intorno dell'origine.
\end{problema}
\begin{soluzione}
	Poniamo $\vec{r}=x\hat{x}+y\hat{y}+<\hat{z}$. Un'idea è passare per i multipoli, o alternativamente espandere in Taylor l'integrale
	\[V(\vec{r})=\int_{0}^{2\pi}\frac{\lambda a\dif \theta}{\sqrt{x^2+y^2-2ax\cos\theta-2ay\sin\theta+z^2+a^2}}\]
	Ovviamente bisognerà espandere al secondo ordine in $r/a$, dato che il momento di dipolo dell'anello è evidentemente nullo. Alternativamente, facciamo qualche considerazione. Per $x=y=0$ il potenziale è pari in $z$, tende a 0 per $|z|\to+\infty$ e vale $2\pi\lambda$ per $z=0$. Viceversa, per $y=z=0$ il potenziale vale $2\pi\lambda$ per $x=0$ e diverge per $x=a$. Di conseguenza $(0,0,0)$ è un punto di sella per il potenziale (come deve essere per il teorema di Earnshaw), quindi la sua espansione al secondo ordine sarà
	\[V(x,y,z)=2\pi\lambda+ax^2+by^2+cz^2+dxy+eyz+fxz+o(x^2+y^2+z^2)\]
	Affinchè il potenziale sia invariante sotto rotazioni intorno a $z$ e sotto simmetrie rispetto al piano $xy$ deve essere $a=b$, $c$, $d=e=f=0$. Inoltre, dato che il potenziale è armonico in un intorno dell'origine deve essere $a+b+c=0$. Tra l'altro, si può pure dire $a,b>0$ e $c<0$. $c$ si calcola banalmente, da cui si ottengono gli altri coefficienti:
	\[V(x,y,z)=2\pi\lambda\left(1+\frac{2x^2+2y^2-z^2}{2a^2}\right)+o(x^2+y^2+z^2)\]
\end{soluzione}
\newpage
\section{12 febbraio 2018}
\begin{problema}
	Si consideri il seguente circuito infinito
	\begin{figure}[h]
		\centering
		\scalebox{1}{
		\begin{tikzpicture}[circuit]
			\draw (0,0)to[R](1.5,0)to[L](3,0)to[C](3,-1.5);
			\draw (3,0)to[R](4.5,0)to[L](6,0)to[C](6,-1.5);
			\draw (6,0)to[R](7.5,0)to[L](9,0)to[C](9,-1.5);
			\draw (0,-1.5)--(10,-1.5);
			\draw(9,0)--(10,0);
			\draw[dashed](-0.75,0)--(0,0);
			\draw[dashed](-0.75,-1.5)--(0,-1.5);
			\draw[dashed](10,0)--(10.75,0);
			\draw[dashed](10,-1.5)--(10.75,-1.5);
			\draw(0.5,-1.5)node[ground]{}(0.5,-1.5);
		\end{tikzpicture}
	}
	\end{figure}

\noindent in cui tutte le resistenze sono pari a $R$, tutte le induttanze pari a $L$ e tutti i condensatori di capacità $C$. Studiare il circuito.
\end{problema}
\begin{soluzione}
	Sia $V_n$ la tensione all'$n$-esimo nodo, $I_n$ la corrente che scorre tra i punti a tensione $V_n$ e $V_{n+1}$ e $Q_n$ la carica sull'$n$-esimo condensatore, ossia
	\begin{figure}[h]
		\centering
		\scalebox{1}{
			\begin{tikzpicture}[circuit]
			\draw [-stealth](4,0.6)--(4.5,0.6)node[above]{$I_n$}--(5,0.6);
			\draw [-stealth](1.7,1)node[above left]{$V_n$}--(2.9,0.3);
			\draw [-stealth](7.3,1)node[above right]{$V_{n+1}$}--(6.1,0.3);
			\draw (0,0)to[R](1.5,0)to[L](3,0)to[C, l=$Q_n$](3,-1.5);
			\draw (3,0)to[R](4.5,0)to[L](6,0)to[C, l=$Q_{n+1}$](6,-1.5);
			\draw (6,0)to[R](7.5,0)to[L](9,0)to[C](9,-1.5);
			\draw (0,-1.5)--(10,-1.5);
			\draw(9,0)--(10,0);
			\draw[dashed](-0.75,0)--(0,0);
			\draw[dashed](-0.75,-1.5)--(0,-1.5);
			\draw[dashed](10,0)--(10.75,0);
			\draw[dashed](10,-1.5)--(10.75,-1.5);
			\draw(0.5,-1.5)node[ground]{}(0.5,-1.5);
			\end{tikzpicture}
		}
	\end{figure}
\end{soluzione}
\appendix
\chapter{Breve formulario}
\section{Analisi vettoriale}
\begin{itemize}
	\item Operatori in coordinate curvilinee: consideriamo una metrica di $\R^3$ data da 
	\[\dif \vec{s}=\sum_{i=1}^{3}h_i\dif x_i\hat{x}_i\]
	Allora si ha, per $f$ funzione scalare e $\vec{v}$ funzione vettoriale
	\[\nabla f=\sum_{i=1}^{3}\frac{1}{h_i}\der{f}{x_i}\hat{x}_i\]
	\[\nabla\cdot\vec{v}=\frac{1}{h_1h_2h_3}\sum_{i=1}^{3}\der{}{x_i}\left(v_i\prod_{j\neq i}h_j\right)\]
	\[\nabla\times\vec{v}=\frac{1}{h_1h_2h_3}\det\left(\begin{array}{c c c}
	h_1\hat{x}_1&h_2\hat{x}_2&h_3\hat{x}_3\\
	\partial_1&\partial_2&\partial_3\\
	h_1v_1&h_2v_2&h_3v_3
	\end{array}\right)\]
	\[\lap f=\frac{1}{h_1h_2h_3}\sum_{i=1}^{3}\der{}{x_i}\left(\frac{1}{h_i}\der{f}{x_i}\prod_{j\neq i}h_j\right)\]
\end{itemize}
\section{Elettrostatica}
\begin{itemize}
	\item Campo elettrico e potenziale di una carica puntiforme
	\[\vec{E}=\frac{q}{r^2}\hat{r}\]
	\[V=\frac{q}{r}\]
	\item Campo elettrico e potenziale di una sfera di raggio $R$ uniformemente carica
	\[\vec{E}=\left\{\begin{array}{l l}
	\frac{q}{r^2}\hat{r} &\textrm{ se }r>R\\
	\frac{4\pi\rho \vec{r}}{3}&\textrm{ se }r<R
	\end{array}\right.\]
	\[V=\left\{\begin{array}{l l}
	\frac{q}{r} &\textrm{ se }r>R\\
	2\pi\rho\left(R^2-\frac{r^2}{3}\right)&\textrm{ se }r<R
	\end{array}\right.\]
	Dove $\rho$ e $q$ sono rispettivamente la densità e la carica totale della sfera.
	\item Campo elettrico e potenziale di un filo uniformemente carico
	\[\vec{E}=\frac{2\lambda}{r}\hat{r}\]
	\[V=-2\lambda\ln\frac{r}{r_0}\]
	Dove $r_0$ è una costante arbitraria.
	\item Campo elettrico di un piano infinito
	\[\vec{E}=2\pi\sigma\textrm{sgn}(z)\hat{z}\]
	Dove $\sigma$ è la densità di carica e si è assunto che il piano sia il piano $xy$.
	\item Campo elettrico all'interno di un condensatore a facce piane e parallele
	\[\vec{E}=4\pi\sigma\]
	Il campo è ortogonale alle piastre e diretto dalla piastra positiva alla piastra negativa.
	\item Campo elettrico e potenziale di un dipolo
	\[\vec{E}=\frac{3(\vec{p}\cdot\hat{r})\hat{r}-\vec{p}}{r^3}\]
	\[V=\frac{\vec{p}\cdot\hat{r}}{r^2}\]
	\item Capacità di un condensatore a facce piane e parallele
	\[C=\frac{S}{4\pi h}\]
	Dove $S$ è l'area delle piastre e $h$ la distanza tra esse.
	\item Capacità di un condensatore sferico
	\[C=\frac{r_2r_1}{r_2-r_1}\]
	Dove $r_2$ e $r_1$ sono i raggi delle due sfere ($r_2>r_1$).
	\item Capacità di un condensatore cilindrico
	\[C=\frac{h}{\ln b/a}\]
	Dove $h$ è l'altezza del condensatore e $a,b$ i raggi dei due cilindri.
	\item Cariche immagine per il piano: si simmetrizza la carica e se ne cambia il segno.
	\item Cariche immagine per una sfera di raggio $R$ messa a terra: se la carica reale è $q$ ed è a distanza $x$ dal centro, si pone una carica $q'=-qR/x$ sulla congiungente tra $q$ e il centro, a distanza $x'=-R^2/x$ da quest'ultimo.
	\item Da aggiungere funzioni di Green e energia.
\end{itemize}
\section{Soluzioni dell'equazione di Laplace e sviluppo in multipoli}
\begin{itemize}
	\item Soluzione in coordinate sferiche con simmetria azimutale
	\[V(r,\theta)=\sum_{r=0}^{\infty}\left(A_lr^l+\frac{B_l}{r^{l+1}}\right)P_l(\cos\theta)\]
	\item Formula di Rodrigues
	\[P_l(x)=\frac{1}{l!2^l}\frac{\dif^l}{\dif x^l}\left((x^2-1)^l\right)\]
	\item Proprietà dei polinomi di Legendre
	\begin{enumerate}
		\item Normalizzazione\[P_l(1)=1\]
		\item Ortogonalità
		\[\int_{-1}^{1}P_l(x)P_k(x)\dif x=\frac{2}{2l+1}\delta_{lk}\]
		\item Formule ricorsive varie
		\[\frac{\dif P_{l+1}}{\dif x}-\frac{\dif P_{l-1}}{\dif x}=(2l+1)P_l\]
		\[(l+1)P_{l+1}-(2l+1)xP_l+lP_{l-1}=0\]
		\[\frac{\dif P_{l+1}}{\dif x}-x\frac{\dif P_l}{\dif x}-(l+1)P_l=0\]
		\[(x^2-1)\frac{\dif P_{l}}{\dif x}-lxP_l+lP_{l-1}=0\]
	\end{enumerate}
	\item Primi polinomi di Legendre
	\begin{eqnarray}
		P_0(x)&=&1\nonumber\\
		P_1(x)&=&x\nonumber\\
		P_2(x)&=&\frac{3\cos^2\theta-1}{2}\nonumber\\
		P_3(x)&=&\frac{5x^3-3x}{2}\nonumber\\
		P_4(x)&=&\frac{35x^4-30x^2+3}{8}\nonumber
	\end{eqnarray}
	\item Soluzione in coordinate sferiche generale
	\[V(r,\theta,\varphi)=\sum_{l=0}^{\infty}\sum_{m=-l}^{l}\left(A_{l,m}r^l+\frac{B_{l,m}}{r^{l+1}}\right)Y_{l,m}(\theta,\varphi)\]
	\item Formula per polinomi di Legendre generalizzati e armoniche sferiche qualunque
	\[P_{l,m}(x)=(-1)^m(1-x^2)^{m/2}\frac{\dif^{m}}{\dif x^{m}}P_l(x)\]
	\[Y_{l,m}(\theta,\varphi)=\sqrt{\frac{2l+1}{4\pi}\frac{(l-m)!}{(l+m)!}}P_{lm}(\cos\theta)e^{im\varphi}\]
	\item Proprietà delle armoniche sferiche
	\begin{enumerate}
		\item $Y_{l,-m}(\theta,\varphi)=(-1)^mY^*_{l,m}(\theta,\varphi)$
		\item Trasformazioni di parità
	\begin{enumerate}
		\item $Y_{l,m}(\pi-\theta,\varphi)=(-1)^{l+m}Y_{l,m}(\theta,\varphi)$
		\item $Y_{l,m}(\theta,\pi+\varphi)=(-1)^mY_{l,m}(\theta,\varphi)$
		\item $Y_{l,m}(\pi-\theta,\pi+\varphi)=(-1)^lY_{l,m}(\theta,\varphi)$
	\end{enumerate}
	\item Ortonormalizzazione
	\[\int Y_{l,m}Y^*_{l',m'}\dif\Omega=\int_{0}^{2\pi}\dif\varphi\int_{0}^{\pi}Y_{l,m}(\theta,\varphi)Y^*_{l',m'}(\theta,\varphi)\sin\theta\dif\theta=\delta_{ll'}\delta_{mm'}\]
	\end{enumerate}
	\item Prime armoniche sferiche
\end{itemize}
\section{Magnetostatica}
\begin{itemize}
	\item Forza di Lorentz e momento agenti su una distribuzione di corrente
	\[\vec{F}=\frac{1}{c}\int\vec{J}\times\vec{B}\dif^3r\]
	\[\vec{\tau}=\frac{1}{c}\int\vec{r}\times(\vec{J}\times\vec{B})\dif^3r\]
	\item Campo magnetico e (un) potenziale vettore di un filo
	\[\vec{B}=\frac{2I}{cr}\hat{\phi}\]
	\[\vec{A}=-\frac{2I\hat{z}}{c}\ln\frac{r}{r_0}\]
	\item Campo magnetico e potenziale vettore di un solenoide di raggio $R$
	\[\vec{B}=\left\{\begin{array}{l l}
	\frac{4\pi nI}{c}\hat{z}&\textrm{ se }r<R\\
	0&\textrm{ se }r>R
	\end{array}\right.\]
	\[\vec{A}=\left\{\begin{array}{l l}
	B\frac{r}{2}\hat{\phi}&\textrm{ se }r<R\\
	B\frac{R^2}{2r}\hat{\phi}&\textrm{ se }r>R
	\end{array}\right.\]
	\item Campo magnetico sull'asse di una spira circolare di raggio $R$
	\[\vec{B}=\frac{2\pi IR^2\hat{z}}{c(z^2+R^2)^{3/2}}\]
	\item Campo magnetico e potenziale vettore di un dipolo
	\[\vec{B}=\frac{3(\vec{m}\cdot\hat{r})\hat{r}-\vec{m}}{r^3}\]
	\[\vec{A}=\frac{\vec{m}\times\hat{r}}{r^2}\]
\end{itemize}
\chapter{Completezza, serie e trasformate di Fourier}

	Questa appendice vuole essere un'introduzione, assai semplificata ed essenziale, agli strumenti matematici utilizzati frequentemente nel corso.
\section{Spazi di Hilbert e serie di Fourier astratta}
\begin{definizione}[Spazio di Hilbert]
	Uno spazio vettoriale $H$ dotato di un prodotto scalare (se lo spazio è reale) o hermitiano (se lo spazio è complesso) viene detto di Hilbert se è completo rispetto alla norma indotta dal prodotto scalare.
\end{definizione}
\noindent D'ora in poi, dato che non ci sono differenze sostanziali tra spazi reali e spazi complessi, parleremo di prodotto scalare anche per spazi di Hilbert complessi (anzi, se non è altrimenti specificato, intenderemo che uno spazio di Hilbert è complesso). Il prodotto scalare è lineare nella seconda componente e antilineare nella prima, ovvero per ogni $v,w\in H$ e per ogni $\lambda\in\C$ si ha
\[(v,\lambda w)=\lambda(v,w)\]
\[(\lambda v,w)=\overline{\lambda}(v,w)\]
\begin{definizione}[Insieme completo, base]
	Sia $H$ uno spazio di Hilbert e consideriamo un suo sottoinsieme $V=\left\{v_i\right\}_{i\in I}$. $V$ si dice insieme completo se combinazioni lineari finite di elementi di $V$ è densa in $H$. $V$ si dice base se è un insieme completo e se è formato da elementi linearmente indipendenti.
\end{definizione}
\begin{lemma}[Disuguaglianza di Bessel]
	Siano $H$ uno spazio di Hilbert e $\left\{e_{n}\right\}_{n\in\mathbb{N}}\subseteq H$ un insieme di vettori ortonormali. Allora per ogni $v\in H$ si ha
	\[\sum_{n=0}^{\infty}|(e_n,v)|^2\leq\norm{v}^2\]
\end{lemma}
\begin{proof}
	Fissiamo $k\in\mathbb{N}$. Allora si ha
	\[0\leq\norm{v-\sum_{n=0}^{k}(e_n,v)e_n}^2=\norm{v}^2-\sum_{n=0}^{k}|(e_n,v)|^2\]
	\[\sum_{n=0}^{k}|(e_n,v)|^2\leq \norm{v}^2\]
	da cui si conclude passando al limite $k\to+\infty$.
\end{proof}
\begin{proposizione}
	Siano $H$ uno spazio di Hilbert, $V=\left\{v_n\right\}_{n\in\mathbb{N}}$. Allora $V$ è completo se e solo se il vettore nullo è l'unico vettore ortogonale a tutti gli elementi di $V$.\label{proposizione}
\end{proposizione}
\begin{proof}
	$\Rightarrow$: sia $v\in H$ tale che $(v,v_n)=0$ per ogni $n\in \mathbb{N}$. Per ipotesi esiste una successione $(w_k)_{k\in\mathbb{N}}$ di combinazioni lineari finite dei vettori di $V$ che converge a $v$. Se $w_k$ è della forma
	\[w_k=\sum_{j=0}^{N_k}\lambda_j^{(k)}v_j\]
	allora dalla continuità del prodotto scalare si ha
	\[\norm{v}^2=(v,v)=(v,\lim\limits_{k\to+\infty}w_k)=\lim\limits_{k\to+\infty}\sum_{j=1}^{N_k}\lambda_j^{(k)}(v,v_k)=0\]
	di conseguenza $v=0$.
	
	\noindent $\Leftarrow$: a meno di ortonormalizzare tramite Gram-Schimdt, possiamo supporre che $V$ sia formato da vettori ortonormali. Fissiamo ora $v\in H$ e consideriamo la successione definita da
	\[w_n=\sum_{j=1}^{n}(v_j,v)v_j\]
	Tale successione è una successione di Cauchy, infatti se $n< m$ si ha
	\[\norm{w_m-w_n}^2\leq\sum_{j=n+1}^{m}|(v_j,v)|^2<\varepsilon\]
	dove la seconda disuguaglianza segue facilmente dalla disuguaglianza di Bessel\footnote{La disuguaglianza di Bessel implica che la serie converge, dunque la successione delle somme parziali è di Cauchy.}. Allora esiste $w=\lim\limits_{n\to+\infty}w_n$. A questo punto si ha, per ogni $k\in\mathbb{N}$,
	\[(v_k,v-w)=\lim\limits_{n\to\infty}(v_k,v-w_n)=\lim\limits_{n\to\infty}\left((v_k,v)-\sum_{j=1}^{n}(v_k,v_j)(v_j,v)\right)=0\]
	dove si è utilizzata l'ortonormalità dei vettori di $V$. Allora per ipotesi $w=v$, dunque $V$ è completo.	
\end{proof}
\begin{corollario}
	Sia $H$ uno spazio di Hilbert e $V=\left\{v_n\right\}_{n\in\mathbb{N}}$ un insieme ortonormale completo. Allora ogni $v\in H$ può essere scritto come la serie
	\[v=\sum_{k=0}^{\infty}(v_k,v)v_k\]
	Tale serie è detta serie di Fourier.
	\label{corollario}
\end{corollario}
\begin{teorema}[di Fischer-Riesz]
	Siano $H$ uno spazio di Hilbert e $V=\left\{v_n\right\}_{n\in\mathbb{N}}$ un insieme di vettori ortonormali. Sono equivalenti:
	\begin{enumerate}
		\item per ogni $v,w\in H$ si ha
		\[(v,w)=\sum_{k=0}^{\infty}\overline{(v_k,v)}(v_k,w)\]
		\item vale l'identità di Parsevall, ovvero per ogni $v\in H$ si ha
		\[\norm{v}^2=\sum_{k=0}^{\infty}|(v_k,v)|^2\]
		\item $V$ è un insieme completo,
		\item il vettore nullo è l'unico vettore ortogonale a tutti i vettori di $V$,
		\item ogni $v\in H$ è esprimibile in serie di Fourier nella base $V$.
	\end{enumerate}
\end{teorema}
\begin{proof}
	1. $\Rightarrow$2. : ovvia.
	
	\noindent2. $\Rightarrow$5. : in tal caso si ha
	\[\lim\limits_{n\to+\infty}\norm{v-\sum_{k=0}^{n}(v_k,v)v_k}^2=\lim\limits_{n\to+\infty}\left(\norm{v}^2-\sum_{k=0}^{n}|(v_k,v)|^2\right)=0\]
	Usando la continuità della norma si ha allora
	\[v=\sum_{k=0}^{\infty}(v_k,v)v\]
	\noindent5. $\Rightarrow$3. : ovvia.
	
	\noindent3. $\Rightarrow$4. : è la proposizione  \ref{proposizione}.
	
	\noindent4. $\Rightarrow$5. : segue dalla proposizione \ref{proposizione} e dal corollario \ref{corollario}.
	
	\noindent5. $\Rightarrow$1. : usando la continuità del prodotto scalare si ha
	\[(v,w)=\lim\limits_{n\to+\infty}\sum_{k=0}^{n}((v_k,v)v_k,w)=\lim\limits_{n\to+\infty}\sum_{k=0}^{n}\overline{(v_k,v)}(v_k,w)=\sum_{k=0}^{\infty}\overline{(v_k,v)}(v_k,w)\]
	
\end{proof}
\section{Serie di Fourier in $L^2$}
\begin{definizione}
	Sia $p\leq1$ un numero reale, $[a,b]\subseteq\R$ un intervallo limitato e consideriamo lo spazio
	\[V^p([a,b])=\left\{f\colon[a,b]\to\C:\int_{a}^{b}|f(x)|^p\dif x<+\infty\right\}\]
	Introduciamo su tale spazio la seguente relazione di equivalenza: $f\sim g$ se e solo se
	\[\int_{a}^{b}|f(x)-g(x)|^p\dif x=0\]
	Definiamo ora lo spazio $L^p([a,b])=V^p([a,b])/\sim$ e introduciamo su $L^p$ la norma $\norm{\cdot}_p\colon L^p\to\R$ data da
	\[\norm{f}_p=\left(\int_{a}^{b}|f(x)|^p\dif x\right)^{1/p}\]
\end{definizione}
\begin{teorema}[di Fischer-Riesz]
		Lo spazio $L^p([a,b])$ con la norma $\norm{\cdot}_p$ è di Banach per ogni $p\geq1$ e per ogni $a,b\in\R$.
\end{teorema}
\begin{teorema}
	Per ogni $a,b\in\R$, $L^p([a,b])$ è uno spazio di Hilbert se e solo se $p=2$.
\end{teorema}
\begin{proof}
	Introduciamo su $L^2([a,b])$ il prodotto scalare
	\[(f,g)=\int_{a}^{b}\overline{f(x)}g(x)\dif x\]
	Tale prodotto è ben definito per la disuguaglianza di H\"older, inoltre si verifica facilmente che induce la norma $\norm{\cdot}_2$. Consideriamo ora le funzioni caratteristiche $f_1=\chi_{[a,(a+b)/2]}$ e $f_2=\chi_{[(a+b)/2,b]}$. Allora si ha
	\[\norm{f_1+f_2}_p=\norm{f_1-f_2}_p=(b-a)^{1/p}\]
	\[\norm{f_1}_p=\norm{f_2}_p=\left(\frac{b-a}{2}\right)^{1/p}\]
	Ricordando che se la norma di uno spazio vettoriale proviene da un prodotto scalare allora rispetta la regola del parallelogramma, si conclude che se $L^p([a,b])$ è di Hilbert allora
	\[2(b-a)^{2/p}=2^{2-2/p}(b-a)^{2/p}\]
	ossia $p=2$.
\end{proof}
\begin{definizione}
	Un polinomio trigonometrico è una funzione della forma
	\[f(x)=\sum_{k=0}^{n}\alpha_ke^{i\beta_kx}\]
	con $\beta_k\in\mathbb{Z}$ per ogni $k$.
\end{definizione}
\begin{lemma}
	Sia $(,)$ il prodotto scalare di $L^2([-\pi,\pi])$. Allora $(e^{inx},e^{imx})=2\pi\delta_{nm}$.
\end{lemma}
\begin{proof}
	Segue da calcolo diretto.
\end{proof}
\begin{teorema}
	I polinomi trigonometrici sono densi in $L^2([-\pi,\pi])$ rispetto alla norma $\norm{\cdot}_2$.
\end{teorema}
\begin{proof}
	Presi $a,b\in\R$ tali che $[a,b]\subseteq(-\pi,\pi)$, in maniera analoga alla dimostrazione del teorema di Stone-Weierstrass si mostra che i polinomi in $\cos x$ e $\sin x$ sono densi in $\mathcal{C}([a,b])$ rispetto alla norma $\norm{\cdot}_\infty$. Chiaramente, tali polinomi sono polinomi trigonometrici. Allora tali polinomi sono densi, sempre rispetto alla norma $\norm{\cdot}_\infty$, anche in
	\[\mathcal{C}_c([-\pi,\pi])=\left\{f\in \mathcal{C}([\pi,\pi]):f(-\pi)=f(\pi)=0\right\}\]
	Inoltre, dato che $[-\pi,\pi]$ è un intervallo limitato, la densità rispetto alla norma $\norm{\cdot}_\infty$ implica la densità rispetto alla norma $\norm{\cdot}_2$. A questo punto si conclude ricordando che $\mathcal{C}_c([-\pi,\pi])$ è denso in $L^2([-\pi,\pi])$, sempre rispetto alla norma $\norm{\cdot}_2$.
\end{proof}
\begin{corollario}
	L'insieme $\left\{(2\pi)^{-1/2}e^{inx}\right\}_{n\in\mathbb{N}}$ è una base ortonormale di $L^2([-\pi,\pi])$.
\end{corollario}
\begin{corollario}
	L'insieme $\left\{(2\pi)^{-1/2}\right\}\cup\left\{\pi^{-1/2}\cos nx,\pi^{-1/2}\sin nx\right\}_{n\in\mathbb{N}}$ è una base ortonormale di $L^2([-\pi,\pi])$.
\end{corollario}

\noindent I risultati precedenti non implicano che la serie di seni e coseni di una funzione $f\in L^2([-\pi,\pi])$ converga puntualmente o uniformemente a $f$. Esistono dei teoremi che permettono di capire quando ciò avviene (ad esempio, il criterio di Dini).
\section{Trasformate di Fourier}
\begin{definizione}
	Definiamo la classe di Schwartz come 
	\[\mathcal{S}=\left\{f\in\mathcal{C}^{\infty}(\R):x^nf^{(m)}(x)\textrm{ è limitato per ogni }n,m\in\mathbb{N}\right\}\]
\end{definizione}
\begin{definizione}
	Presa $f\in\mathcal{S}$, definiamo la sua trasformata di Fourier come
	\[\hat{f}(\omega)=\mathcal{F}[f](\omega)=\frac{1}{\sqrt{2\pi}}\int_{-\infty}^{+\infty}f(x)e^{i\omega x}\dif x\]
\end{definizione}
\begin{lemma}
	Se $f\in\mathcal{S}$, allora $\hat{f}\in\mathcal{S}$, dunque la trasformata di Fourier è un operatore lineare su $\mathcal{S}$.
\end{lemma}
\begin{proof}
	Mostriamo che $\hat{f}$ è derivabile e che $\hat{f}'=\mathcal{F}[ixf]$. Si ha
	\[\frac{\hat{f}(\omega+h)-\hat{f}(\omega)}{h}-\mathcal{F}[ixf](\omega)=\fourier{ixf(x)e^{i\omega x}\left(\frac{e^{ihx}-1}{ihx}-1\right)}{x}\]
	L'integrale a secondo membro tende puntualmente a 0 quando $h\to 0$. Inoltre, esiste $M\in\R$ tale che per ogni $x\in\R$ si ha
	\[\left|\frac{e^{ihx}-1}{ihx}-1\right|<M\]
	Dunque la funzione integranda è dominata da $M|xf(x)|\in L^1(\R)$. Allora per il teorema di convergenza dominata si può passare al limite sotto il segno di integrale e concludere
	\[\frac{\dif \hat{f}}{\dif\omega}(\omega)=\mathcal{F}[ixf](\omega)\] FINISCI
\end{proof}
\begin{lemma}
	Sia $f\in\mathcal{S}$. Allora
	\[f(x)=\fourier{\hat{f}(\omega)e^{-i\omega x}}{\omega}\]
\end{lemma}
\begin{lemma}[Identità di Parsevall] Siano $f,g\in\mathcal{S}$. Allora $(f,g)=(\hat{f},\hat{g})$.
\end{lemma}
\begin{proof}
	
\end{proof}
\begin{corollario}
	$\mathcal{F}\colon\mathcal{S}\to\mathcal{S}$ è biiettiva.
\end{corollario}
\begin{proof}
	
\end{proof}
\end{document}
	